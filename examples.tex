As a motivating example, we will show how our features for stepping through programs can help clarify the execution of a map function.

A Hazel implementation of a map function over lists of integers is given in \autoref{fig:hazel-map}. The full execution trace is shown in \autoref{fig:full-trace}. It shows each of the small-step transitions that have been taken when evaluating this program. 

\begin{figure}
    \centering
    \begin{minipage}{.40\linewidth}
    \begin{subfigure}{\linewidth}
        \includegraphics[scale=0.57]{images/eg11.png}
        \caption{Hazel Code. This code, when run, will evaluate to the list [1,4,9].}
        \label{fig:hazel-map}
    \end{subfigure}
    \end{minipage}
    \hfill
    \begin{minipage}{.57\linewidth}
    \begin{subfigure}{\linewidth}
        \includegraphics[scale=0.57]{images/eg12.png}
        \caption{Stepper output in stepper mode. Strikethroughs indicate that a substitution has taken place.The code highlighted green at the bottom indicates where the next small-step transition can be taken.}
        \label{fig:full-trace}
    \end{subfigure}
    \end{minipage}
    \caption{The stepper used on a list map example.}
\end{figure}

In some circumstances, seeing the full length of the execution trace can be distracting. We may, for example, only want to focus on the recursive calls. The Hazel stepper provides some filter primitives that can specify steps to hide. In order to more clearly show the intuition behind the map function, we can hide all steps except for those which show the application of the map function, by inserting a hide filter and a pause filter as shown in \autoref{fig:hide-pause-map}.

\begin{figure}
    \centering
    \begin{minipage}{.40\linewidth}
    \begin{subfigure}{\linewidth}
        \includegraphics[scale=0.57]{images/eg21.png}
        \caption{The code from \autoref{fig:hazel-map} modified with extra hide and pause statements.}
        \label{fig:hide-pause-map}
    \end{subfigure}
    \end{minipage}
    \hfill
    \begin{minipage}{.57\linewidth}
    \begin{subfigure}{\linewidth}
        \includegraphics[scale=0.57]{images/eg22.png}
        \caption{Stepper output showing only the recursive calls.}
        \label{fig:full-trace}
    \end{subfigure}
    \end{minipage}
    \caption{A filtered stepper for showing recursive calls in the list map example.}
\end{figure}

Each filter has a pattern that triggers the filter. The \verb|$e| pattern matches any expression, and the \verb|$v| pattern matches any fully-evaluated expression. \verb|map($v, $v)| is a pattern that matches any map expression applied to two fully-evaluated expressions. More details on the patterns are given in \TODO{ref}

The \verb|hide| construct hides a step that matches the pattern, thus \verb|hide $e| hides all steps. The \verb|pause| construct shows a step that matches the pattern. Since it comes later, the \verb|pause map($v, $v)| overrides the \verb|hide| expression to show any steps taken from an expression that contains a \verb|map| applied to two values. More details are given on the ordering of filters in \TODO{ref}

The \verb|hide| and \verb|pause| constructs each only hide or show one step. We also provide two more constructs, \verb|eval| and \verb|debug| that hide or show all evaluation steps for a subexpression that matches the given filter. 

For example, in figure \autoref{fig:debug-map} we have added a debug filter that shows the full trace of each application of \verb|f|. We can see in \autoref{fig:debug-trace} that this leads to not only showing steps that apply \verb|f|, but also the multiplication that takes place to evaluate \verb|f(x)|.


\begin{figure}
    \centering
    \begin{minipage}{.40\linewidth}
    \begin{subfigure}{\linewidth}
        \includegraphics[scale=0.57]{images/eg31.png}
        \caption{The code from \autoref{fig:hazel-map} modified with extra hide, pause, and debug statements.}
        \label{fig:debug-map}
    \end{subfigure}
    \end{minipage}
    \hfill
    \begin{minipage}{.57\linewidth}
    \begin{subfigure}{\linewidth}
        \includegraphics[scale=0.57]{images/eg32.png}
        \caption{Stepper output showing only the recursive calls and the steps taken to evaluate squares.}
        \label{fig:debug-trace}
    \end{subfigure}
    \end{minipage}
    \caption{A filtered stepper that demonstrates debug mode.}
\end{figure}

Note that this \verb|debug| statement overrides the \verb|pause| and \verb|eval| statement because of the dynamic scoping of filters, given in \TODO{ref}



% \subsection{Matching}

% Terms that has the same structural form after substituting all bounded variables
% are considered \emph{matched}.

% \begin{verbatim}
% let x = 3 in
% eval x + 3 in
% x + 3
% \end{verbatim}
% shall has the same behavior as the program
% \begin{verbatim}
% let x = 3 in
% eval 3 + 3 in
% 3 + 3
% \end{verbatim}

% \begin{verbatim}
% eval $e in
% let g = fun x -> x + x in
% pause g($v) in # or pause (fun x -> x + x)($v) in
% (fun x -> x + x)(3)
% == (fun x -> x + x)(3) # or g(3) #
% \end{verbatim}

% \begin{figure}[h]
%   \includegraphics[width=0.4\textwidth]{images/match-mod-subst.png}
% \end{figure}

% As in a substitution-based evaluator, the substitution is applied to the
% the body of a function:
% \begin{verbatim}
% eval $e in
% let y = fun x -> x in
% let h = fun x -> (fun y -> y) in
% pause (fun x -> y)($v) in
% h(3)
% == h(3) # or (fun x -> (fun x -> x))(3) #
% \end{verbatim}

% \begin{figure}[h]
%   \includegraphics[width=0.4\textwidth]{images/match-recursive.png}
% \end{figure}

% A immediate corollary of this is that the stepper filter shows that
% the filter cannot be applied to a variable, since the substitution is always
% performed ahead of the matching process, and for the filter it shall not be
% able to match against a variable.

% \begin{verbatim}
% eval $e in
% let x = 3 in
% pause x in
% x + x
% == 6
% \end{verbatim}

% % \TODO{Discuss: whether or not to use alpha-equivalence here.}

% \subsection{Four filters: Hide, Eval, Pause and Debug}

% % \TODO{Discuss: mention skip all, skip one, pause all and all one?}

% Typical use case for the four basic filters.

% \subsubsection{Eval}

% We want to use the \verb|eval| to \emph{evaluate} all sub-expressions that matches the pattern.
% \begin{verbatim}
% eval 1 + 2 in
% 1 + 2
% == 3
% \end{verbatim}

% \subsubsection{Hide}

% The \verb|hide| filter will be \emph{used up} when its body
% expression is actually being evaluated.

% \begin{verbatim}
% hide let   =   in   in
% let x = 3 in
% let y = 4 in
% x + y
% == 3 + 4
% == 7
% \end{verbatim}
% Compare with \verb|eval|
% \begin{verbatim}
% eval let   =   in   in
% let x = 3 in
% let y = 4 in
% x + y
% == 7
% \end{verbatim}

% \subsubsection{Pause}

% During evaluation, we want to be able to examine every instruction
% transitioning steps.

% \begin{verbatim}
% eval 1 + 2 + 3 + 4 in
% pause 3 + 3 in
% 1 + 2 + 3 + 4
% == 3 + 3 + 4
% == 10
% \end{verbatim}

% This would be especially useful when unfolding a higher level
% expression to its individual terms, for example I want to know how
% many terms of \verb|fib(1)| I need to call to evaluate the whole
% \verb|fib(5)|.

% \subsubsection{Debug}

% On the other hand, we want to went through all the evaluation process of a
% sub-expression, for example when debugging the implementation of a function
% \begin{verbatim}
% let (is_even: Int -> Bool, is_odd: Int -> Bool) = ... in
% debug is_odd($v) in
% is_even(5)
% \end{verbatim}

% The difference between \verb|pause| and \verb|debug| is subtle. For
% example, considering the following two piece of code:
% \begin{verbatim}
% eval $e in
% pause (1 + 2) + (3 + 4) + (5 + 6) in
% eval 3 + 7 + (5 + 6) in
% (1 + 2) + (3 + 4) + (5 + 6)
% == (1 + 2) + (3 + 4) + (5 + 6)
% == 21
% \end{verbatim}
% and
% \begin{verbatim}
% eval $e in
% debug (1 + 2) + (3 + 4) + (5 + 6) in
% eval 3 + 7 + (5 + 6) in
% (1 + 2) + (3 + 4) + (5 + 6)
% == (1 + 2) + (3 + 4) + (5 + 6)
% == 3 + (3 + 4) + (5 + 6)
% == 21
% \end{verbatim}
% The first one will immediately evaluate to final value is because it is a \verb|pause| statement, which will be only effective once.

% \subsection{Interaction between filter statements}

% % \TODO{Q: Do we need formalise these properties?}

% We want nested filter statements to behave correctly, i.e.
% \begin{enumerate}
% \item For every pattern \verb|p|, \verb|pause p| cancels the effects
%   of \verb|eval p| and \verb|hide p|, vice versa.
% \item Inner filter statements take precedences.
% \end{enumerate}

% \subsubsection{``Newer'' Expression overrides ``Older'' Expressions}

% \begin{verbatim}
% pause 1 + 2 + 3 + 4 in
% eval 1 + 2 + 3 + 4 in
% 1 + 2 + 3 + 4
% == 10
% \end{verbatim}

% \begin{verbatim}
% pause $e in
% eval 1 + 2 + 3 + 4 in
% pause 1 + 2 + 3 + 4 in
% 1 + 2 + 3 + 4
% == 1 + 2 + 3 + 4
% \end{verbatim}

% \begin{verbatim}
% pause $e in
% hide (let   =   in  ) in
% pause (let   =   in  ) in
% let x = 3 in
% x + 4
% == let x = 3 in x + 4
% \end{verbatim}

% \subsubsection{The nesting properties should works across bindings}

% \begin{verbatim}
% eval $e in
% let x = 1 in
% pause 3 + 3 in
% let y = 2 in
% eval 3 + 3 in
% x + y + 3 + 4
% == 10
% \end{verbatim}

% \begin{verbatim}
% let add = fun x, y -> pause $e in x + y in
% eval $e in
% add(3, 4)
% == [3 + 4]
% \end{verbatim}


% \subsubsection{We want the filter to recover to \emph{older} state when it finish
% evaluating matched sub-expression.}

% \begin{verbatim}
% pause $e in
% eval 1 + 2 + 3 + 4 in
% pause 3 + 3 in
% 1 + 2 + 3 + 4
% == 3 + 3 + 4
% == 10
% \end{verbatim}

% After evaluating \verb|3 + 3|, the stepper falls back to eval mode
% since eval filter matches \verb|1 + 2 + 3 + 4|, so it directly
% evaluates to \verb|10|.

% We also want a \verb|eval| filter to automatically evaluate all sub-expression
% until it cannot proceed.



% \subsection{Handling inconsistency between DHExp and UExp}

% \TODO{Discuss: Move to implementation.tex?}

% There are inconsistencies between the surface expression and expression for
% evaluation in Hazel. For example, fix-points are inserted in the expression
% during elaboration. We want the filters to work with fix-points, with-out user
% acknowledging that they actually needs a fix-point to implement the recursion.

% \begin{verbatim}
% let map : ([Int], Int -> Int) -> [Int] = fun xs, f ->
%   case xs
%   | [] => []
%   | hd :: tl => f(hd)::map(tl, f)
%   end
% in
% let square = fun x -> x * x in
% pause map($v) in
% map([1, 2, 3], square)
% == 1::map([2, 3], f)
% == 1::4::map([3], f)
% \end{verbatim}

% In the example above, we don't want to use to click twice to do unroll and apply, instead we want to merge these two transition
% in one step.

% \subsection{Good but unrealistic for now}

% \TODO{Discuss: Is it better remove in total?}

% There are also something that we think would be intuitive and useful but not possible with current implementation

% \begin{verbatim}
% let fib : Int -> Int =
%   fun n ->
%     if n <= 1 then
%       n
%     else
%       fib(n - 1) + fib(n - 2)
% in
% pause fib($v) + fib($v) in
% fib(5)
% == 3
% \end{verbatim}

% Intuitively we shall see a evaluation trace like this:
% \begin{verbatim}
% ...
% == fib(4) + fib(3)
% == fib(3) + fib(2) + fib(3)
% == fib(3) + fib(2) + fib(2) + fib(1)
% == ...
% \end{verbatim}

% This is no possible since this require us to traverse all possible
% evaluation sequence given a program, which would be super powerful,
% but at the same time super slow.

%%% Local Variables:
%%% mode: latex
%%% TeX-master: "main"
%%% End:
