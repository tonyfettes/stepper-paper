%%
%% This is file `sample-acmsmall.tex',
%% generated with the docstrip utility.
%%
%% The original source files were:
%%
%% samples.dtx  (with options: `acmsmall')
%%
%% IMPORTANT NOTICE:
%%
%% For the copyright see the source file.
%%
%% Any modified versions of this file must be renamed
%% with new filenames distinct from sample-acmsmall.tex.
%%
%% For distribution of the original source see the terms
%% for copying and modification in the file samples.dtx.
%%
%% This generated file may be distributed as long as the
%% original source files, as listed above, are part of the
%% same distribution. (The sources need not necessarily be
%% in the same archive or directory.)
%%
%% Commands for TeXCount
%TC:macro \cite [option:text,text]
%TC:macro \citep [option:text,text]
%TC:macro \citet [option:text,text]
%TC:envir table 0 1
%TC:envir table* 0 1
%TC:envir tabular [ignore] word
%TC:envir displaymath 0 word
%TC:envir math 0 word
%TC:envir comment 0 0
%%
%%
%% The first command in your LaTeX source must be the \documentclass command.
\documentclass[acmsmall,anonymous]{acmart}
%% NOTE that a single column version is required for
%% submission and peer review. This can be done by changing
%% the \doucmentclass[...]{acmart} in this template to
%% \documentclass[manuscript,screen]{acmart}
%%
%% To ensure 100% compatibility, please check the white list of
%% approved LaTeX packages to be used with the Master Article Template at
%% https://www.acm.org/publications/taps/whitelist-of-latex-packages
%% before creating your document. The white list page provides
%% information on how to submit additional LaTeX packages for
%% review and adoption.
%% Fonts used in the template cannot be substituted; margin
%% adjustments are not allowed.
%%
%% \BibTeX command to typeset BibTeX logo in the docs
\AtBeginDocument{%
  \providecommand\BibTeX{{%
    \normalfont B\kern-0.5em{\scshape i\kern-0.25em b}\kern-0.8em\TeX}}}

%% Rights management information.  This information is sent to you
%% when you complete the rights form.  These commands have SAMPLE
%% values in them; it is your responsibility as an author to replace
%% the commands and values with those provided to you when you
%% complete the rights form.
\setcopyright{acmlicensed}
\copyrightyear{2018}
\acmYear{2018}
\acmDOI{XXXXXXX.XXXXXXX}


%%
%% These commands are for a JOURNAL article.
\acmJournal{JACM}
\acmVolume{37}
\acmNumber{4}
\acmArticle{111}
\acmMonth{8}

%%
%% Submission ID.
%% Use this when submitting an article to a sponsored event. You'll
%% receive a unique submission ID from the organizers
%% of the event, and this ID should be used as the parameter to this command.
%%\acmSubmissionID{123-A56-BU3}

%%
%% For managing citations, it is recommended to use bibliography
%% files in BibTeX format.
%%
%% You can then either use BibTeX with the ACM-Reference-Format style,
%% or BibLaTeX with the acmnumeric or acmauthoryear sytles, that include
%% support for advanced citation of software artefact from the
%% biblatex-software package, also separately available on CTAN.
%%
%% Look at the sample-*-biblatex.tex files for templates showcasing
%% the biblatex styles.
%%

%%
%% The majority of ACM publications use numbered citations and
%% references.  The command \citestyle{authoryear} switches to the
%% "author year" style.
%%
%% If you are preparing content for an event
%% sponsored by ACM SIGGRAPH, you must use the "author year" style of
%% citations and references.
%% Uncommenting
%% the next command will enable that style.
%%\citestyle{acmauthoryear}

%%
%% end of the preamble, start of the body of the document source.

\usepackage{mathtools}
\usepackage{amsmath}
\usepackage{ebproof}
\usepackage{xcolor}
\usepackage{listings}
\lstset{
  escapeinside={@}{@},
  captionpos=b,
}
\usepackage{subcaption}
\usepackage{mathpartir}
\usepackage{pgfplots}

% !TEX root = ./main.tex

%%% Editing Convience Macros

\newif\ifdraft
\drafttrue
\newcommand{\TODO}[1]{\ifdraft{\textcolor{red}{\textbf{TODO}: {#1}}}\fi}
\newcommand{\mck}[1]{\ifdraft{\color{purple}[#1 --Matt]}\fi}
\newcommand{\hxf}[1]{\ifdraft{\color{blue}[#1 --Haoxiang]}\fi}
\newcommand{\cyr}[1]{\ifdraft{\color{green}[#1 --Cyrus]}\fi}


%%% Hazel language highlight

\lstdefinelanguage{hazel}{
  columns=fullflexible,
  keepspaces=true,
  basicstyle=\ttfamily\color{black},
  keywords={
    let, in, case, end, type,
    eval, pause, hide, debug, filter, do,
  },
  keywordstyle=\bfseries\color{black},
  commentstyle=\color{gray},
}

%%% Stepper/Filter Semantics

\DeclareMathSymbol{\Gamma}{\mathalpha}{operators}{0}
\DeclareMathSymbol{\Delta}{\mathalpha}{operators}{1}
\DeclareMathSymbol{\Theta}{\mathalpha}{operators}{2}
\DeclareMathSymbol{\Lambda}{\mathalpha}{operators}{3}
\DeclareMathSymbol{\Xi}{\mathalpha}{operators}{4}
\DeclareMathSymbol{\Pi}{\mathalpha}{operators}{5}
\DeclareMathSymbol{\Sigma}{\mathalpha}{operators}{6}
\DeclareMathSymbol{\Upsilon}{\mathalpha}{operators}{7}
\DeclareMathSymbol{\Phi}{\mathalpha}{operators}{8}
\DeclareMathSymbol{\Psi}{\mathalpha}{operators}{9}
\DeclareMathSymbol{\Omega}{\mathalpha}{operators}{10}

%% Kinds

\DeclareMathOperator{\DefPat}{Pat}
\DeclareMathOperator{\DefExp}{Exp}
\DeclareMathOperator{\DefVal}{Val}
\DeclareMathOperator{\DefCtx}{Ctx}
\DeclareMathOperator{\DefAct}{Act}
\DeclareMathOperator{\DefGas}{Gas}
\DeclareMathOperator{\DefFilter}{Filter}
\DeclareMathOperator{\DefTyp}{Typ}
\DeclareMathOperator{\DefTypCtx}{TypCtx}
\DeclareMathOperator{\KeywordFilter}{filter}
\DeclareMathOperator{\KeywordHide}{hide}
\DeclareMathOperator{\KeywordEval}{eval}
\DeclareMathOperator{\KeywordPause}{pause}
\DeclareMathOperator{\KeywordDebug}{debug}
\DeclareMathOperator{\KeywordResidue}{do}

%% Terms

% Pat
\newcommand{\PatExpr}{\mathtt{\$e}}
\newcommand{\PatValue}{\mathtt{\$v}}

% Act
\newcommand{\ActSkip}{\mathsf{skip}}
\newcommand{\ActStep}{\mathsf{step}}

% Gas
\newcommand{\GasOne}{\mathsf{one}}
\newcommand{\GasAll}{\mathsf{all}}

% Expr
\newcommand{\Nat}[1]{\underline{#1}}
\newcommand{\Lam}[2]{\lambda #1 . #2}

\newcommand{\Filter}[2]{\KeywordFilter #1 ~\mathsf{in}~ #2}
\newcommand{\FilterEval}[2]{\KeywordEval #1 ~\mathsf{in}~ #2}
\newcommand{\FilterHide}[2]{\KeywordHide #1 ~\mathsf{in}~ #2}
\newcommand{\FilterDebug}[2]{\KeywordDebug #1 ~\mathsf{in}~ #2}
\newcommand{\FilterPause}[2]{\KeywordPause #1 ~\mathsf{in}~ #2}

\newcommand{\Residue}[4]{\prescript{#1}{#3}{\langle} #4 \rangle^{#2}}
\newcommand{\ResidueEval}[2]{\KeywordEval \# #1 ~\mathsf{in}~ #2}
\newcommand{\ResidueHide}[2]{\KeywordHide \# #1 ~\mathsf{in}~ #2}
\newcommand{\ResiduePause}[2]{\KeywordPause \# #1 ~\mathsf{in}~ #2}
\newcommand{\ResidueDebug}[2]{\KeywordDebug \# #1 ~\mathsf{in}~ #2}

%% Functions

\newcommand{\FSubst}[3]{[#1/#2] #3}
\DeclareMathOperator{\Strip}{strip}
\DeclareMathOperator{\Decay}{decay}

%% Judgments

\newcommand{\FCMark}{\circ}

\newcommand{\Value}[1]{#1~\mathsf{value}}
\newcommand{\FEStep}[2]{#1 \mapsto #2}

\newcommand{\Decompose}[3]{#1 = #2 \{ #3 \}}
\newcommand{\Compose}[3]{#1 = #2 \{ #3 \}}

\newcommand{\Matches}[2]{#1 \mathrel{\mathop{\triangleright}} #2}
\newcommand{\FPatMatchesExpOp}{\mathrel{\mathop{\triangleright}}}
\newcommand{\FPatMatchesExp}[2]{#1 \FPatMatchesExpOp #2}
\newcommand{\FPatNotMatchesExpOp}{\mathrel{\mathop{\not\triangleright}}}
\newcommand{\FPatNotMatchesExp}[2]{#1 \FPatNotMatchesExpOp #2}

\newcommand{\FPatMatchesCtxOpL}{\mathrel{\mathop{\triangleright_\circ}}}
\newcommand{\FPatMatchesCtxOpR}{\mathrel{\mathop{\triangleleft_\circ}}}
\newcommand{\FPatMatchesCtx}[3]{#1 \FPatMatchesCtxOpL #2 \{ #3 \}}

\newcommand{\FInstruct}[3]{#1 \vdash #2 \rightsquigarrow #3}
\newcommand{\Instruct}[6]{(#1, #2, #3, #4) \vdash #5 \rightsquigarrow #6}

\newcommand{\Analyze}[3]{#1 \vdash #2 \dashv #3}

\newcommand{\FTrans}[2]{#1 \rightarrow #2}

\newcommand{\RuleSkip}[2]{#1 \rightarrow^{\ast} #2}
\newcommand{\RuleStep}[2]{#1 \mapsto #2}

\newcommand{\FHasType}[3]{#1 \vdash #2 : #3}
\newcommand{\FInTypeContext}[3]{#1 : #2 \in #3}
\newcommand{\FTNat}{\mathbb{N}}

\newcommand{\Arrow}[2]{#1 \Rightarrow #2}
\newcommand{\Natural}{\mathbb{N}}
\newcommand{\TypeEntails}[3]{#1 \vdash #2 : #3}
\newcommand{\TypeContains}[3]{#1 \ni #2 : #3}
\newcommand{\TypeExtends}[3]{#1 , #2 : #3}

%%% Local Variables:
%%% mode: latex
%%% TeX-master: "main"
%%% End:


\newcommand{\evalsto}{\mathrel{\mathop{\Downarrow}}}
\newcommand{\matches}{\mathrel{\mathop{\blacktriangleright}}}
\newcommand{\fmatches}{\mathrel{\mathop{\triangleright}}}
\newcommand{\smatches}{\mathrel{\mathop{\sim}}}
\newcommand{\hooksto}{\mathrel{\mathop{\hookrightarrow}}}
\newcommand{\entails}{\mathrel{\mathop{\vdash}}}
\newcommand{\steps}{\mathrel{\mathop{\vartriangleright}}}
\newcommand{\skips}{\mathrel{\mathop{\blacktriangleright}}}
\newcommand{\final}{~\mathbf{final}}
\newcommand{\ival}{~\mathbf{value}}
\newcommand{\indet}{~\mathbf{indet}}
\newcommand{\istep}{~\mathbf{step}}
\newcommand{\iskip}{~\mathbf{skip}}
\newcommand{\class}[1]{\operatorname{#1}}
% \DeclareMathOperator{\Filter}{Filter}
\DeclareMathOperator{\askip}{\mathsf{skip}}
\DeclareMathOperator{\astep}{\mathsf{step}}
\DeclareMathOperator{\filter}{\mathsf{filter}}
\newcommand{\fin}{\mathrel{\mathop{\mathsf{in}}}}
\newcommand{\flet}{\operatorname{\mathsf{let}}}
\newcommand{\synth}{\mathrel{\mathop{\Rightarrow}}}
\newcommand{\analyze}{\mathrel{\mathop{\Leftarrow}}}
\newcommand{\ctype}[1]{\mathsf{#1}}
\newcommand{\inl}{\operatorname{\mathsf{injL}}}
\newcommand{\inr}{\operatorname{\mathsf{injR}}}
\newcommand{\prl}{\operatorname{\mathsf{prjL}}}
\newcommand{\prr}{\operatorname{\mathsf{prjR}}}
\newcommand{\MatchArrow}{\mathrel{\mathop{\blacktriangleright_{\rightarrow}}}}
\newcommand{\fif}{\operatorname{\mathsf{if}}}
\newcommand{\fthen}{\mathrel{\mathop{\mathsf{then}}}}
\newcommand{\felse}{\mathrel{\mathop{\mathsf{else}}}}
\newcommand{\fcase}{\operatorname{\mathsf{case}}}
\newcommand{\fcaseL}{\mathsf{L}}
\newcommand{\fcaseR}{\mathsf{R}}
\DeclareMathOperator{\instr}{\mathsf{instr}}

\begin{document}

%%
%% The "title" command has an optional parameter,
%% allowing the author to define a "short title" to be used in page headers.
\title{Stepper}

%%
%% The "author" command and its associated commands are used to define
%% the authors and their affiliations.
%% Of note is the shared affiliation of the first two authors, and the
%% "authornote" and "authornotemark" commands
%% used to denote shared contribution to the research.
\author{Ben Trovato}
\authornote{Both authors contributed equally to this research.}
\email{trovato@corporation.com}
\orcid{1234-5678-9012}
\author{G.K.M. Tobin}
\authornotemark[1]
\email{webmaster@marysville-ohio.com}
\affiliation{%
  \institution{Institute for Clarity in Documentation}
  \streetaddress{P.O. Box 1212}
  \city{Dublin}
  \state{Ohio}
  \country{USA}
  \postcode{43017-6221}
}

\author{Lars Th{\o}rv{\"a}ld}
\affiliation{%
  \institution{The Th{\o}rv{\"a}ld Group}
  \streetaddress{1 Th{\o}rv{\"a}ld Circle}
  \city{Hekla}
  \country{Iceland}}
\email{larst@affiliation.org}

\author{Valerie B\'eranger}
\affiliation{%
  \institution{Inria Paris-Rocquencourt}
  \city{Rocquencourt}
  \country{France}
}

\author{Aparna Patel}
\affiliation{%
 \institution{Rajiv Gandhi University}
 \streetaddress{Rono-Hills}
 \city{Doimukh}
 \state{Arunachal Pradesh}
 \country{India}}

\author{Huifen Chan}
\affiliation{%
  \institution{Tsinghua University}
  \streetaddress{30 Shuangqing Rd}
  \city{Haidian Qu}
  \state{Beijing Shi}
  \country{China}}

\author{Charles Palmer}
\affiliation{%
  \institution{Palmer Research Laboratories}
  \streetaddress{8600 Datapoint Drive}
  \city{San Antonio}
  \state{Texas}
  \country{USA}
  \postcode{78229}}
\email{cpalmer@prl.com}

\author{John Smith}
\affiliation{%
  \institution{The Th{\o}rv{\"a}ld Group}
  \streetaddress{1 Th{\o}rv{\"a}ld Circle}
  \city{Hekla}
  \country{Iceland}}
\email{jsmith@affiliation.org}

\author{Julius P. Kumquat}
\affiliation{%
  \institution{The Kumquat Consortium}
  \city{New York}
  \country{USA}}
\email{jpkumquat@consortium.net}

%%
%% By default, the full list of authors will be used in the page
%% headers. Often, this list is too long, and will overlap
%% other information printed in the page headers. This command allows
%% the author to define a more concise list
%% of authors' names for this purpose.
\renewcommand{\shortauthors}{Trovato and Tobin, et al.}

%%
%% The abstract is a short summary of the work to be presented in the
%% article.
\begin{abstract}
  A clear and well-documented \LaTeX\ document is presented as an
  article formatted for publication by ACM in a conference proceedings
  or journal publication. Based on the ``acmart'' document class, this
  article presents and explains many of the common variations, as well
  as many of the formatting elements an author may use in the
  preparation of the documentation of their work.
\end{abstract}

%%
%% The code below is generated by the tool at http://dl.acm.org/ccs.cfm.
%% Please copy and paste the code instead of the example below.
%%
\begin{CCSXML}
<ccs2012>
 <concept>
  <concept_id>00000000.0000000.0000000</concept_id>
  <concept_desc>Do Not Use This Code, Generate the Correct Terms for Your Paper</concept_desc>
  <concept_significance>500</concept_significance>
 </concept>
 <concept>
  <concept_id>00000000.00000000.00000000</concept_id>
  <concept_desc>Do Not Use This Code, Generate the Correct Terms for Your Paper</concept_desc>
  <concept_significance>300</concept_significance>
 </concept>
 <concept>
  <concept_id>00000000.00000000.00000000</concept_id>
  <concept_desc>Do Not Use This Code, Generate the Correct Terms for Your Paper</concept_desc>
  <concept_significance>100</concept_significance>
 </concept>
 <concept>
  <concept_id>00000000.00000000.00000000</concept_id>
  <concept_desc>Do Not Use This Code, Generate the Correct Terms for Your Paper</concept_desc>
  <concept_significance>100</concept_significance>
 </concept>
</ccs2012>
\end{CCSXML}

\ccsdesc[500]{Do Not Use This Code~Generate the Correct Terms for Your Paper}
\ccsdesc[300]{Do Not Use This Code~Generate the Correct Terms for Your Paper}
\ccsdesc{Do Not Use This Code~Generate the Correct Terms for Your Paper}
\ccsdesc[100]{Do Not Use This Code~Generate the Correct Terms for Your Paper}

%%
%% Keywords. The author(s) should pick words that accurately describe
%% the work being presented. Separate the keywords with commas.
\keywords{Do, Not, Us, This, Code, Put, the, Correct, Terms, for,
  Your, Paper}

\received{20 February 2007}
\received[revised]{12 March 2009}
\received[accepted]{5 June 2009}

%%
%% This command processes the author and affiliation and title
%% information and builds the first part of the formatted document.
\maketitle

\section{Introduction}

Debugger has been one of the most important tools that students and programmer
use to learn the behavior of their programs. Well-established debuggers such as
GDB, LLDB, and Visual Studio Debugger have been widely used in industry and
academia. These debuggers provide rich set of tools for user to inspect and even
change various aspects of a running program, such as memory, variables,
and call stacks. However, they also suffers from several limitations.

\begin{enumerate}
  \item They are tailored for imperative languages like C or C++. They are not
    well-suited for functional languages, where memory is usually managed by
    GC and never directly manipulated by the programmer.
  \item Many of them support stepping in and stepping over instructions. However,
    these two instructions can be tedious to use when the program is large and
    complex. The user has to step through many instructions that are not relevant
    to the current task.
  \item They are not well-suited for teaching. They are designed for professional
    programmers who are already familiar with the language and the program they
    are debugging. They are not well-suited for beginners who are learning the
    language and the program.
\end{enumerate}

Therefore, we want to develop a new kind of debugger that tailored for
functional languages and is well-suited for teaching in Hazel. Hazel is a pure
functional language that supports working with incomplete programs. It is used
in a upper-level programming languages course where we are teaching
substitution-based equational semantics -- we want the steps to show up as
justified proof steps, to mirror what we have previously done in lectures and
written assignments.

\TODO{(example here of a simple equational stepping proof)}

One of the main challenges in using such a debugger in lecturing is that it is
often tedious to step through a program. Students has to witness many boring
instructions that are not relevant to the current task, or that they already
well understand. Those instructions will litter the proof and obscure the key
idea. Our approach is to develop a scoped filter system that allows us to write patterns
that will be skipped or paused. For example,

\TODO{(more sophisticated example where you show side-by-side the unfiltered and
filtered version of the same stepping sequence) fib, foldright, map}

Our scoped filter system allows the user to specify sub-expressions whose full
evaluation they are not interested in, sub-expressions whose full evaluation
they are interested in, specific steps they are not interested in, and specific
steps they are interested in.

\TODO{want to be able to step programs with holes, including error holes:
abstract foldright vs foldleft}

% Prior work summary: what have people tried to do to handle this challenge in the past? why is that not quite enough

We have made the following contributes in the paper:
\begin{itemize}
  \item We have developed the hazel stepper and a companion scoped filter system
    that allows the user to specify sub-expressions whose full evaluation they
    are not interested in, sub-expressions whose full evaluation they are
    interested in, specific steps they are not interested in, and specific steps
    they are interested in.
  \item We have deployed the hazel stepper and the scoped filter system in
    lecture and to students. We have conducted qualitative and quantitative
    evaluation of the system.
  \item We have formalized the filter semantics and proved the meta-theory
    theorems of the hazel filter system so that we can be confident that the
    filter system have desired properties.
\end{itemize}

% - developed the hazel stepper and filter system, implemented it, deployed it in lecture (qualitative evaluation), deployed it to students (qualitative + quantitative evaluation), formalized the filter semantics, metatheory


\mck{content moved out of example}

We recognized some key capabilities that traditional debuggers have and will be
also helpful to a debugger for functional programming languages.
\begin{enumerate}
  \item Evaluation can be paused at some point and resumed later. The user
    should be able specify where/when the evaluation should be paused. This is
    usually implemented as the breakpoint mechanism in a traditional debugger.
  \item Paused evaluation can be stepped through. traditionally, debuggers will
    provide user with three buttons to ``step through'', ``step over'' and
    ``step out''.
\end{enumerate}

Moreover, we want some extra capabilities that are specific to functional
programming languages and a classroom settings, especially Hazel, to be a
useful tool for teaching and learning functional programming.
\begin{enumerate}
  \item It can skip the stepping process of some \emph{kind} of expression.
  \item It controls the evaluation of the program in a granularity of
  expression, instead of a granularity of line or instruction.
\end{enumerate}

To serve as a good tool for understanding the behaviour of a program, the stepper
filter should have consistent behaviour with the semantics of the language. Hazel
internally uses a environment-based big-step evaluator to evaluate the program,
but it has a consistent behaviour with the substitution-based one. Therefore,
the stepper filter shall not exhibit different behaviour when building upon a
environment-based evaluator or a substitution-based evaluator.

The evaluation order of Hazel is not specified. Therefore, an expression that is
not value might have multiple possible choice

%%% Local Variables:
%%% mode: LaTeX
%%% TeX-master: "main"
%%% End:


\section{Hazel Stepper by Example}

In this section, we introduce the Hazel stepper filter.

We recognized some key capabilities that traditional debuggers have and will be
also helpful to a debugger for functional programming languages.
\begin{enumerate}
  \item Evaluation can be paused at some point and resumed later. The user
    should be able specify where/when the evaluation should be paused. This is
    usually implemented as the breakpoint mechanism in a traditional debugger.
  \item Paused evaluation can be stepped through. traditionally, debuggers will
    provide user with three buttons to ``step through'', ``step over'' and
    ``step out''.
\end{enumerate}

Moreover, we want some extra capabilities that are specific to functional
programming languages and a classroom settings, especially Hazel, to be a
useful tool for teaching and learning functional programming.
\begin{enumerate}
  \item It can skip the stepping process of some \emph{kind} of expression.
  \item It controls the evaluation of the program in a granularity of
  expression, instead of a granularity of line or instruction.ß
\end{enumerate}

To serve as a good tool for understanding the behavior of a program, the stepper
filter should have consistent behavior with the semantics of the language. Hazel
internally uses a environment-based big-step evaluator to evaluate the program,
but it has a consistent behavior with the substitution-based one. Therefore,
the stepper filter shall not exhibit different behavior when building upon a
environment-based evaluator or a substitution-based evaluator.

The evaluation order of Hazel is not specified. Therefore, an expression that is
not value might have multiple possible choice 

\subsection{Matching}


Terms that has the same structural form after substituting all bounded variables
are considered \emph{matched}.

\begin{verbatim}
let x = 3 in
eval x + 3 in
x + 3
\end{verbatim}
shall has the same behavior as the program
\begin{verbatim}
let x = 3 in
eval 3 + 3 in
3 + 3
\end{verbatim}

\begin{verbatim}
eval $e in
let g = fun x -> x + x in
pause g($v) in # or pause (fun x -> x + x)($v) in
(fun x -> x + x)(3)
== (fun x -> x + x)(3) # or g(3) #
\end{verbatim}

\begin{figure}[h]
  \includegraphics[width=0.4\textwidth]{images/match-mod-subst.png}
\end{figure}

As in a substitution-based evaluator, the substitution is applied to the
the body of a function:
\begin{verbatim}
eval $e in
let y = fun x -> x in
let h = fun x -> (fun y -> y) in
pause (fun x -> y)($v) in
h(3)
== h(3) # or (fun x -> (fun x -> x))(3) #
\end{verbatim}

\begin{figure}[h]
  \includegraphics[width=0.4\textwidth]{images/match-recursive.png}
\end{figure}

A immediate corollary of this is that the stepper filter shows that
the filter cannot be applied to a variable, since the substitution is always
performed ahead of the matching process, and for the filter it shall not be
able to match against a variable.

\begin{verbatim}
eval $e in
let x = 3 in
pause x in
x + x
== 6
\end{verbatim}

\TODO{Discuss: whether or not to use alpha-equivalence here.}

\subsection{Four filters: Hide, Eval, Pause and Debug}

\TODO{Discuss: mention skip all, skip one, pause all and all one?}

Typical use case for the four basic filters.

\paragraph{eval} We want to use the \verb|eval| to \emph{evaluate} all sub-expressions that matches the pattern.
\begin{verbatim}
eval 1 + 2 in
1 + 2
== 3
\end{verbatim}

We want the \verb|hide| filters to hide away one step.
\begin{verbatim}
hide let   =   in   in
let x = 3 in
let y = 4 in
x + y
== 3 + 4
== 7
\end{verbatim}
while using \verb|eval| it will behave like this:
\begin{verbatim}
eval let   =   in   in
let x = 3 in
let y = 4 in
x + y
== 7
\end{verbatim}

\paragraph{hide} The \verb|hide| filter will be \emph{used up} when its body
expression is actually being evaluated.

During evaluation, we want to be able to examine every instruction
transitioning steps.

\begin{verbatim}
eval 1 + 2 + 3 + 4 in
pause 3 + 3 in
1 + 2 + 3 + 4
== 3 + 3 + 4
== 10
\end{verbatim}

This would be especially useful when unfolding a higher level
expression to its individual terms, for example I want to know how
many terms of \verb|fib(1)| I need to call to evaluate the whole
\verb|fib(5)|.

On the other hand, we want to went through all the evaluation process of a
sub-expression, for example when debugging the implementation of a function
\begin{verbatim}
let (is_even: Int -> Bool, is_odd: Int -> Bool) = ... in
debug is_odd($v) in
is_even(5)
\end{verbatim}

The difference between \verb|pause| and \verb|debug| is subtle. For
example, considering the following two piece of code:
\begin{verbatim}
eval $e in
pause (1 + 2) + (3 + 4) + (5 + 6) in
eval 3 + 7 + (5 + 6) in
(1 + 2) + (3 + 4) + (5 + 6)
== (1 + 2) + (3 + 4) + (5 + 6)
== 21
\end{verbatim}
and
\begin{verbatim}
eval $e in
debug (1 + 2) + (3 + 4) + (5 + 6) in
eval 3 + 7 + (5 + 6) in
(1 + 2) + (3 + 4) + (5 + 6)
== (1 + 2) + (3 + 4) + (5 + 6)
== 3 + (3 + 4) + (5 + 6)
== 21
\end{verbatim}
The first one will immediately evaluate to final value is because it is a \verb|pause| statement, which will be only effective once.

Filters can't be applied on values. A more significant corollary of this is they cannot be applied on variables as well.
\begin{verbatim}
eval $e in
pause x in
pause y in
let x = 3 in
let y = 4 in
x + y
== 7
\end{verbatim}

\subsection{Interaction between filter statements}

\TODO{Q: Do we need formalise these properties?}

We want nested filter statements to behave correctly, i.e.
\begin{enumerate}
\item For every pattern \verb|p|, \verb|pause p| cancels the effects
  of \verb|eval p| and \verb|hide p|, vice versa.
\item Inner filter statements take precedences.
\end{enumerate}

\begin{verbatim}
pause 1 + 2 + 3 + 4 in
eval 1 + 2 + 3 + 4 in
1 + 2 + 3 + 4
== 10
\end{verbatim}

\begin{verbatim}
pause $e in
eval 1 + 2 + 3 + 4 in
pause 1 + 2 + 3 + 4 in
1 + 2 + 3 + 4
== 1 + 2 + 3 + 4
\end{verbatim}

\begin{verbatim}
pause $e in
hide (let   =   in  ) in
pause (let   =   in  ) in
let x = 3 in
x + 4
== let x = 3 in x + 4
\end{verbatim}

The nesting properties should works across bindings.

\begin{verbatim}
eval $e in
let x = 1 in
pause 3 + 3 in
let y = 2 in
eval 3 + 3 in
x + y + 3 + 4
== 10
\end{verbatim}

\begin{verbatim}
let add = fun x, y -> pause $e in x + y in
eval $e in
add(3, 4)
== [3 + 4]
\end{verbatim}

We want the filter to recover to \emph{older} state when it finish
evaluating matched sub-expression.

\begin{verbatim}
pause $e in
eval 1 + 2 + 3 + 4 in
pause 3 + 3 in
1 + 2 + 3 + 4
== 3 + 3 + 4
== 10
\end{verbatim}

After evaluating \verb|3 + 3|, the stepper falls back to eval mode
since eval filter matches \verb|1 + 2 + 3 + 4|, so it directly
evaluates to \verb|10|.

We also want a \verb|eval| filter to automatically evaluate all sub-expression
until it cannot proceed.

\subsection{Handling inconsistency between DHExp and UExp}

\TODO{Discuss: Move to implementation.tex?}

There are inconsistencies between the surface expression and expression for
evaluation in Hazel. For example, fix-points are inserted in the expression
during elaboration. We want the filters to work with fix-points, with-out user
acknowledging that they actually needs a fix-point to implement the recursion.

\begin{verbatim}
let map : ([Int], Int -> Int) -> [Int] = fun xs, f ->
  case xs
  | [] => []
  | hd :: tl => f(hd)::map(tl, f)
  end
in
let square = fun x -> x * x in
pause map($v) in
map([1, 2, 3], square)
== 1::map([2, 3], f)
== 1::4::map([3], f)
\end{verbatim}

In the example above, we don't want to use to click twice to do unroll and apply, instead we want to merge these two transition
in one step.

\subsection{Good but unrealistic for now}

\TODO{Discuss: Is it better remove in total?}

There are also something that we think would be intuitive and useful but not possible with current implementation

\begin{verbatim}
let fib : Int -> Int =
  fun n ->
    if n <= 1 then
      n
    else
      fib(n - 1) + fib(n - 2)
in
pause fib($v) + fib($v) in
fib(5)
== 3
\end{verbatim}

Intuitively we shall see a evaluation trace like this:
\begin{verbatim}
...
== fib(4) + fib(3)
== fib(3) + fib(2) + fib(3)
== fib(3) + fib(2) + fib(2) + fib(1)
== ...
\end{verbatim}

This is no possible since this require us to traverse all possible
evaluation sequence given a program, which would be super powerful,
but at the same time super slow.

%%% Local Variables:
%%% mode: latex
%%% TeX-master: "main"
%%% End:


% (think about examples that touch on all the edge cases of the system)

% (show substitutions explicitly)

\section{Filtered Stepper Calculus}

In this section, we will discuss the semantics of the \emph{Filtered Stepper
Calculus}. We first introduce the syntax of the language, then we discuss the
the dynamics of the language. Finally, we will discuss the optional part of the
semantic, the statics of the language, and show that the calculus satisfies
properties like progression and preservation.

\subsection{Syntax}

\begin{figure}[h]
  \begin{equation*}
    \begin{array}{rcl}
      \DefAct a    &\Coloneqq& \ActSkip \mid \ActStep \\
      \DefGas g    &\Coloneqq& \GasOne \mid \GasAll \\
      \DefFilter f &\Coloneqq& (p, a, g) \\
      \DefExp e    &\Coloneqq& x \mid e(e) \mid \Lam{x}{e} \mid e + e \mid \Nat{n} \\
                   &\mid     & \Filter{f}{e} \\
                   &\mid     & \Residue{a}{g}{l}{e} \\
      \DefPat p    &\Coloneqq& x \mid p(p) \mid \Lam{x}{p} \mid p + p \mid \Nat{n} \\
                   &\mid     & \PatExpr \\
                   &\mid     & \PatValue \\
    \end{array}
  \end{equation*}
  \caption{Syntax of Filtered Stepper Calculus. Here, $n$ ranges over numbers, $x$ over variables, and $l$ over priorities, which are natural numbers.}
  \label{fig:filter-syntax}
\end{figure}

The syntax of filter stepper calculus (Fig.~\ref{fig:filter-syntax}) is a
extended version of untyped lambda calculus. Additionally, natural numbers and
addition are added to it for some more concrete demonstration. It introduce two
new expression to a untyped lambda calculus language, \texttt{Filter} and
\texttt{Residue}. The \texttt{Filter} expression is visible at user level,
indicating the expression surrounded by the expression is filtered by the
filter. The \texttt{Residue} expression is used internally by the stepper and is
not visible to the user. It is used to signal the \emph{residue} of a effect of
applying a filter to an expression.

A filter is a tuple of a pattern, an action, and a gas. The pattern
is a mutated of the term of untyped lambda calculus. Apart from constructs like
variables, applications, and lambdas, it also includes some new constructs, a
forall pattern \(\PatExpr\), a value pattern \(\PatValue\). The action is a tag indicating
the action to be taken when the pattern matches an expression. It has two
form, \(\ActSkip\) and \(\ActStep\). The gas is another tag indicating the
live time of the effect of a filter. It has two form, \(\GasOne\) and
\(\GasAll\).

The residue expression \(\Residue{a}{g}{l}{e}\) is a expression that is used to
store the residue of the effect of a filter. It has four components, an action
\(a\), a gas \(g\), a priority \(l\), and a expression \(e\). The action and the
gas are the same as the ones in a filter. The priority is a natural number
indicating the priority of the residue.

\subsection{Contextual Dynamics}

The contextual dynamics of the filtered stepper calculus builds around
the stepping judgment. This judgment establish the contextual behavior
of the stepper.

There are two main operations in our semantics, one is to \emph{skip} the
evaluation of an expression, the other is to \emph{pause} at a certain point of
evaluation. These two operations leads to two main stepping judgment in our
dynamic specifications. The \textsc{S-Step} rule specifies when do we want to
stop at a step, while the \textsc{S-Skip} rule specifies when we want to skip over
current step and proceed to next ones.

There is an extra rule in the stepping judgment. The elimination step of filter
should be always skipped. As pattern-matching against the filter statement is
not possible within current pattern definition as in
Figure.~\ref{fig:filter-syntax}, we treat them specially to make sure these
steps are hidden and is not visible to users.

We will first see two examples going through how should we or user
interpreter these two rules, the gives the formal definitions of the
judgment and all preceding judgment, including judgment for
instrumentation, decomposition/re-composition, instruction-transition,
and most importantly, stepping.

% \TODO{motivate the two rules}

\subsubsection{Examples and Motivations}

Consider the following Hazel program:
\lstset{xleftmargin=1in}
\begin{lstlisting}[language=hazel,caption={A simple Hazel program}]
debug eval(1 + 2 + 3 + 4) in
debug stop(3 + 3 + 4) in
1 + 2 + 3 + 4\end{lstlisting}

This Hazel program will go through several steps as listed below, to
be evaluated against the order specified by the user.

\begin{enumerate}
\item \emph{Instrumentation}. \label{num:simple_example_instrument} At
  this step, we transverse the expression recursively and collect all
  the filters along the path. Then for each sub-expression, we test if
  the sub-expression matches any of the collected filters. If it does,
  we insert a residue expression with the filter information around
  the matched sub-expression.

  If we start with an empty set of filters, we might end up in a
  situation where there is no sub-expression in the program that
  matches any of the filters. In that situation, the stepper won't be
  able to decide to skip over or stop at a step during evaluation.
  Therefore, we set the default behavior to be stepping through the
  evaluation of every expression.

  In our example above, we apply 3 filters to the expression
  \lstinline[language=hazel]{1 + 2 + 3 + 4}. One of them is the
  default filter that gets prepend to every user program, and the
  other two is written by the user.

  \begin{enumerate}
  \item \lstinline[language=hazel]{debug stop($e)}
  \item \lstinline[language=hazel]{debug eval(1 + 2 + 3 + 4)}
  \item \lstinline[language=hazel]{debug stop(3 + 3 + 4)}
  \end{enumerate}

  During the instrumenting of the first filter, we find out that the
  sub-expression \lstinline[language=hazel]{1 + 2 + 3 + 4} matches the
  pattern \lstinline[language=hazel]{1 + 2 + 3 + 4}, so we instrument
  the matched expression by wrapping it in a internal language
  construct called residue expression
  \(\Residue{\ActSkip}{\GasAll}{0}{\ldots}\) around the sub-expression
  (I-Add-Y). The instrumented expression is in Listing~\ref{lst:simple_example_instrument}.

  \begin{lstlisting}[language=hazel,caption={Program after instrumentation},label={lst:simple_example_instrument}]
debug stop($e) in
debug eval(1 + 2 + 3 + 4) in
debug stop(3 + 3 + 4) in
@\(
\prescript{\ActStep}{0}{\langle}
\prescript{\ActSkip}{1}{\langle}
\prescript{\ActStep}{0}{\langle}
\prescript{\ActStep}{0}{\langle}
\)@1 + 2@\({\rangle}^{\GasOne}\)@ + 3@\({\rangle}^{\GasOne}\)@ + 4@\(\rangle^{\GasAll}
\rangle^{\GasOne}\)@\end{lstlisting}

  However, during the instrumenting of the second filter, there is no
  sub-expression that matches the pattern
  \lstinline[language=hazel]{3 + 3 + 4}, so no residue expression is
  inserted around the sub-expression (I-Add-N). The instrumented
  expression remains the same.

\item \emph{Optimization}. The stepper might have insert too many
  residue expression in the instrumentation step, and this could
  probably cause a significant slowdown of the stepper. Therefore,
  some redundant residue expressions are optimized out during this
  step.

  Generally, we want to remove all immediately adjacent residue
  expressions by picking the one with higest priority, because this is
  the only one that can be picked among these residue expressions
  during the action selection process. Here, the most outward two
  residue expressions are adjacent:
  \(\Residue{\ActStep}{\GasOne}{0}{\Residue{\ActSkip}{\GasAll}{1}{\ldots}}\),
  we remove the ones with lower priorities (O-Residue-Inner) to get
  \(\Residue{\ActSkip}{\GasAll}{1}{\ldots}\).

\begin{lstlisting}[
  language=hazel,
  caption={Program after instrumentation, optimized},
]
debug stop($e) in
debug eval(1 + 2 + 3 + 4) in
debug stop(3 + 3 + 4) in
@\(
\prescript{\ActSkip}{1}{\langle}
\prescript{\ActStep}{0}{\langle}
\prescript{\ActStep}{0}{\langle}
\)@1 + 2@\({\rangle}^{\GasOne}\)@ + 3@\({\rangle}^{\GasOne}\)@ + 4@\(
\rangle^{\GasAll}
\)@\end{lstlisting}

\item \emph{Decomposition}. \label{num:simple_example_decompose} The
  expression is then decomposed into an evaluation context and a
  selected expression. The instrumented expression from last step can
  be decomposed by recursively decomposing addition (D-Add-L) and
  residue expression (D-Residue), into an evaluation context
  (Listing~\ref{lst:simple_example_decompose}) and a selected
  expression \lstinline{1 + 2}

  \begin{lstlisting}[language=hazel,caption={Decomposed evaluation context},label={lst:simple_example_decompose}]
debug stop($e) in
debug eval(1 + 2 + 3 + 4) in
debug stop(3 + 3 + 4) in
@\(
\prescript{\ActSkip}{1}{\langle}
\prescript{\ActStep}{0}{\langle}
\prescript{\ActStep}{0}{\langle}
\circ{\rangle}^{\GasOne}\)@ + 3@\({\rangle}^{\GasOne}\)@ + 4@\(
\rangle^{\GasAll}
\)@\end{lstlisting}

\item \emph{Action Selection}. \label{num:simple_example_select}
  Select an action from the evaluation context. The action is chosen
  first by their priority, then by their distance to the mark
  (\(\circ\)) in the evaluation context. In this case, we transverse
  the expression using (A-Add-L) and update the selected action
  accordingly. The action \(\ActSkip\) is selected, and the expression
  itself is kept the same.

\item \emph{Residue removal (Decaying)}. Residue expressions that
  has life-time of one is removed from the expression. This step is
  done using the \(\Decay\) function. The expression after the removal
  of the residue expression is:
  \begin{lstlisting}[language=hazel,caption={Evaluation context after decaying},label={lst:simple_example_decay}]
debug stop($e) in
debug eval(1 + 2 + 3 + 4) in
debug stop(3 + 3 + 4) in
@\(\prescript{\ActSkip}{1}{\langle}\circ\)@ + 3 + 4@\(\rangle^{\GasAll}\)@\end{lstlisting}

\item \emph{User selection}. If the selected expression produced in
  the Step~\ref{num:simple_example_decompose} is of the form
  \lstinline[language=hazel]{debug .. in ..} or
  \(\Residue{a}{g}{l}{e}\), or the selected action from
  Step~\ref{num:simple_example_select} is \(\ActSkip\), the stepper
  will skip over this step and goes into the next step. (S-Filter and
  S-Skip). Otherwise, the stepper will yield the control to the user,
  and let them to choose which next possible step to take. (S-Step)

  In this case, the selected action is \(\ActSkip\). Therefore, the stepper
  will skipping over this step and goes into the next step.

\item \emph{Instruction Transition}. The selected expression is
  instruction-transitioned to a new expression. In this case, selected
  expression \lstinline[language=hazel]{1 + 2} is
  instruction-transitioned to \lstinline[language=hazel]{3} according
  to the transition rule (T-Add).

\item \emph{Composition}: When the selected-expression is
  instruction-transitioned to a new expression, we need to compose the
  transitioned selected expression with evaluation context to get the
  updated expression. The rules for composition is the same as the
  rules for decomposition, but instead, we start from rule (D-Add-L)
  to build up the recomposed expression in a bottom-up way. The
  recomposed expression is:
  \begin{lstlisting}[language=hazel]
debug stop($e) in
debug eval(1 + 2 + 3 + 4) in
debug stop(3 + 3 + 4) in
@\(\prescript{\ActSkip}{1}{\langle}\)@3 + 3 + 4@\(\rangle^{\GasAll}\)@\end{lstlisting}

\item \emph{Recursion}. As long as we don't get a value, we will keep
  stepping through the expression, i.e. goes back to
  Step~\ref{num:simple_example_instrument}.
\end{enumerate}

In the following round of the procedure, the program is first
instrumented (Listing~\ref{lst:simple_example_1_instrument}), then get
decomposed into an evaluation context
(Listing~\ref{lst:simple_example_1_decompose}) with a selected
expression \lstinline[language=hazel]{3 + 3}.

\begin{lstlisting}[
  language=hazel,
  caption={Instrumented program in Round 2},
  label={lst:simple_example_1_instrument}
]
debug stop($e) in
debug eval(1 + 2 + 3 + 4) in
debug stop(3 + 3 + 4) in
@\(
\prescript{\ActSkip}{1}{\langle}
\prescript{\ActStep}{0}{\langle}
\prescript{\ActStep}{2}{\langle}
\prescript{\ActStep}{0}{\langle}
\)@3 + 3@\(
\rangle^{\GasOne}
\)@ + 4@\(
\rangle^{\GasOne}
\rangle^{\GasOne}
\rangle^{\GasAll}
\)@\end{lstlisting}

\begin{lstlisting}[
  language=hazel,
  caption={Decomposed evaluation context in Round 2},
  label={lst:simple_example_1_decompose}
]
debug stop($e) in
debug eval(1 + 2 + 3 + 4) in
debug stop(3 + 3 + 4) in
@\(
\prescript{\ActSkip}{1}{\langle}
\prescript{\ActStep}{0}{\langle}
\prescript{\ActStep}{2}{\langle}
\prescript{\ActStep}{0}{\langle}
\circ
\rangle^{\GasOne}
\)@ + 4@\(
\rangle^{\GasOne}
\rangle^{\GasOne}
\rangle^{\GasAll}
\)@\end{lstlisting}

Then, we computed the selected action to be \(\ActStep\). Therefore,
the stepper stops at this step and let user to choose which
sub-expression should be evaluated for the next step, by clicking the
yellow area as shown in the Figure~\ref{fig:simple_example_final_ui}

\begin{figure}[h]
  \centering
  \includegraphics[width=0.4\textwidth]{images/filter-example-user-select.png}
  \caption{User Interface for the \emph{User Selection} Step.}\label{fig:simple_example_final_ui}
\end{figure}

Let's take a look at another motivating example. Suppose a first-year
student in college is learn the concept of recursion by writing a
simple recursive function to calculate the n-th term of the Fibonacci
sequence. They would probably write a Hazel program like the one in Listing~\ref{lst:filter_example_fib_program}

\begin{lstlisting}[
  language=hazel,
  caption={A Hazel program that calculates the n-th term of the Fibonacci sequence},
  label={lst:filter_example_fib_program}
]
debug eval($e) in
let fib = fun n ->
  if n == 1 then
    1
  else
    fib(n - 1) + fib(n - 2)
in
debug stop(fib($v)) in
fib(3)
\end{lstlisting}

% The stepper will go through a similar process to determine how to
% evaluate the program and when to stop and yield the control to the
% user through the user interface. To reduce the verbosity, we only
% state 

% \begin{enumerate}
% \item \emph{Instrumentation, Decomposition, and Residue Removal}. The program can be decomposed into an
%   evaluation context (Listing~\ref{lst:filter_example_fib_decompose}) and a selected expression \lstinline[language=hazel]{fib(3)}.
% \begin{lstlisting}[
%   language=hazel,
%   caption={Decomposed evaluation context of the program},
%   label={lst:filter_example_fib_decompose}
% ]
% debug eval($e) in
% @\(\prescript{\ActSkip}{0}{\langle}\)@let fib = fun n ->
%      if n == 1 then
%        1
%      else
%        fib(n - 1) + fib(n - 2)
%    in
%    debug stop(fib($v)) in
%    @\(
% \prescript{\ActSkip}{0}{\langle}
% \prescript{\ActStep}{1}{\langle}
% \circ
% \rangle^{\GasOne}
% \rangle^{\GasAll}
% \rangle^{\GasAll}
% \)@\end{lstlisting}
% \item \emph{Action selection and User selection}. Since the stepper
%   selected the action \(\ActStep\), it will stop here and yield its
%   control to user.
% \item \emph{Instruction Transition, Composition, and Decomposition in
%     the next round}. The decomposition in Round 2 yields an evaluation
%   context as in Listing~\ref{lst:filter_example_fib_decompose} and a selected expression
%   \lstinline[language=hazel]{3 == 1}
% \begin{lstlisting}[
%   language=hazel,
%   caption={Decomposed evaluation context of the program},
%   label={lst:filter_example_fib_decompose}
% ]
% debug eval($e) in
% @\(\prescript{\ActSkip}{0}{\langle}\)@let fib = fun n ->
%      if n == 1 then
%        1
%      else
%        fib(n - 1) + fib(n - 2)
%    in
%    debug stop(fib($v)) in
%    @\(
% \prescript{\ActSkip}{0}{\langle}
% \prescript{\ActStep}{1}{\langle}
% \)@if @\(\circ\)@ then
%             1
%           else
%             fib(3 - 1) + fib(3 - 2)@\(
% \rangle^{\GasOne}
% \rangle^{\GasAll}
% \rangle^{\GasAll}
% \)@\end{lstlisting}
% \item \emph{Action Selections and User Interaction}. The selected
%   action is \(\ActSkip\). This step is skipped. 
% \item \emph{}
% \end{enumerate}

With the two motivating examples, the rules for the dynamic semantics
of the filter calculus is presented as following.

\fbox{\(\Step{e}{e'}{n}\)} Expression \(e\) steps to \(e'\) in \(n\) steps.
\begin{mathpar}
  \inferrule[S-Step]{
    \InstructPAGL{\PatExpr}{\ActStep}{\GasOne}{0}{e}{e_i} \\
    \Decompose{e_i}{\mathcal{E}_0}{e_0} \\
    \Analyze{(\ActStep, 0)}{\mathcal{E}_0}{\ActStep} \land{} \neg{} (\Strippable{e_0}) \\
    \Transition{e_0}{e_t} \\
    \Compose{e_1}{(\Decay{\mathcal{E}_1})}{e_t}
  }{
    \Step{e}{e_1}{1}
  } \\
  \inferrule[S-Skip]{
    \InstructPAGL{\PatExpr}{\ActStep}{\GasOne}{0}{e}{e_i} \\
    \Decompose{e_i}{\mathcal{E}_0}{e_0} \\
    \Analyze{(\ActStep, 0)}{\mathcal{E}_0}{\ActSkip} \lor \Strippable{e_0}\\
    \Transition{e_0}{e_t} \\
    \Compose{e_1}{(\Decay{\mathcal{E}_1})}{e_t} \\    
    \Step{e_1}{e_2}{n}
  }{
    \Step{e}{e_2}{n + 1}
  } \\
  \inferrule[S-Value]{
    \Value{v}
  }{
    \Step{v}{v}{0}
  }
\end{mathpar}

\fbox{\(\JustStep{e}{e'}{n}\)} Expression \(e\) takes \(n\) steps to \(e'\), completely ignore the behavior of the filters and residues.
\begin{mathpar}
  \inferrule[J-Step]{
    \Decompose{e}{\mathcal{E}_0}{e_0} \\
    \Transition{e_0}{e_0'} \\
    \Compose{e'}{\mathcal{E}_0}{e_0'}
  }{
    \JustStep{e}{e'}{1}
  }
\end{mathpar}

\fbox{\(\Strippable{e}\)} Expression \(e\) can be stripped (non-recursive).
\begin{mathpar}
  \inferrule[F-Filter]{
  }{
    \Strippable{(\Filter{f}{e})}
  } \qquad
  \inferrule[F-Residue]{
  }{
    \Strippable{\Residue{a}{g}{l}{e}}
  }
\end{mathpar}

\fbox{\(\InstructPAGL{p}{a}{g}{l}{e}{e'}\)} \(e\) is dynamically instructed to \(e'\).
\begin{mathpar}
  \inferrule[I-V]{
    \Value{v}
  }{
    \InstructPAGL{p}{a}{g}{l}{v}{v}
  } \qquad
  \inferrule[I-Var]{
    \Matches{p}{x}
  }{
    \InstructPAGL{p}{a}{g}{l}{x}{x}
  } \\
  \inferrule[I-Filter]{
    \InstructF{(p_0, a_0, g_0, l_0)}{e_0}{e} \\
    \InstructF{(p, a, g, l_0 + 1)}{e}{e'}
  }{
    \InstructF{(p_0, a_0, g_0, l_0)}{\Filter{(p, a, g)}{e_0}}{\Filter{(p, a, g)}{e'}}
  } \\
  \inferrule[I-Residue]{
    \InstructF{(p_0, a_0, g_0, l_0)}{e_0}{e} \\
  }{
    \InstructF{(p_0, a_0, g_0, l_0)}{\Residue{a}{g}{l}{e_0}}{\Residue{a}{g}{l}{e}}
  } \\
  \inferrule[I-Ap-Y]{
    \InstructPAGL{p}{a}{g}{l}{e_1}{e_1'} \\
    \InstructPAGL{p}{a}{g}{l}{e_2}{e_2'} \\
    \Matches{p}{e_1(e_2)}
  }{
    \InstructPAGL{p}{a}{g}{l}{e_1(e_2)}{\Residue{a}{g}{l}{e_1'(e_2')}}
  } \\
  \inferrule[I-Ap-N]{
    \InstructPAGL{p}{a}{g}{l}{e_1}{e_1'} \\
    \InstructPAGL{p}{a}{g}{l}{e_2}{e_2'} \\
    \DoesNotMatch{p}{e_1(e_2)}
  }{
    \InstructPAGL{p}{a}{g}{l}{e_1(e_2)}{e_1'(e_2')}
  } \\
  \inferrule[I-Add-Y]{
    \InstructPAGL{p}{a}{g}{l}{e_1}{e_1'} \\
    \InstructPAGL{p}{a}{g}{l}{e_2}{e_2'} \\
    \Matches{p}{e_1 + e_2}
  }{
    \InstructPAGL{p}{a}{g}{l}{e_1(e_2)}{\Residue{a}{g}{l}{e_1' + e_2'}}
  } \\
  \inferrule[I-Add-N]{
    \InstructPAGL{p}{a}{g}{l}{e_1}{e_1'} \\
    \InstructPAGL{p}{a}{g}{l}{e_2}{e_2'} \\
    \DoesNotMatch{p}{e_1 + e_2}
  }{
    \InstructPAGL{p}{a}{g}{l}{e_1(e_2)}{e_1' + e_2'}
  } \\
\end{mathpar}

% \mck{I think there's some preservation of priority property to prove here, since the numbers are re-generated on every pass.}

% \hxf{I would formalize the preservation property like this: the priority of residue/do statements \emph{produced} by the same filter have the same priorities}

% \mck{This instrumentation will regenerate the do statements on every pass - I guess it doesn't cause problems but it would be nice if it didn't?}

% \hxf{Yeah it would be much nicer to avoid the duplication of do statements. We can to some ad-hoc fusing of do-statements same effect \& priorties, probably utilmately we want the instrumentation to be Idempotent.}

\fbox{\(\Decay{\mathcal{E}} = \mathcal{E}'\)} Evaluation Context \(\mathcal{E}\) decays to context \(\mathcal{E}'\).
\[
  \begin{aligned}
    \Decay{x} &= x \\
    \Decay{\Lam{x}{e}} &= \Lam{x}{\Decay{e}} \\
    \Decay{e_1(e_2)} &= (\Decay{e_1})(\Decay{e_2}) \\
    \Decay{\Nat{n}} &= \Decay{\Nat{n}} \\
    \Decay{e_1 + e_2} &= (\Decay{e_1}) + (\Decay{e_2}) \\
    \Decay{\Filter{(p, a, g)}{e}} &= \Filter{(p, a, g)}{\Decay{e}} \\
    \Decay{\Residue{a}{\GasOne}{l}{e}} &= \Decay{e} \\
    \Decay{\Residue{a}{\GasAll}{l}{e}} &= \Residue{a}{\GasAll}{l}{\Decay{e}}
  \end{aligned}
\]

% \TODO{Correct the function to use correct evaluation context}

\fbox{\(\Analyze{(a, l)}{\mathcal{E}}{a'}\)} Under the filter environment of \((a, l)\), the context \(\mathcal{E}\) is transitioned to \(\mathcal{E}'\) and the action \(a'\) is returned.
\begin{mathpar}
  \inferrule[A-Var]{
  }{
    \Analyze{(a, l)}{\circ}{a}
  } \\
  \inferrule[A-Ap-L]{
    \Analyze{(a, l)}{\mathcal{E}_1}{a'}
  }{
    \Analyze{(a, l)}{\mathcal{E}_1(e)}{a'}
  } \qquad
  \inferrule[A-Ap-R]{
    \Analyze{(a, l)}{\mathcal{E}_2}{a'}
  }{
    \Analyze{(a, l)}{e_1(\mathcal{E}_2)}{a'}
  } \\
  \inferrule[A-Add-L]{
    \Analyze{(a, l)}{\mathcal{E}_1}{a'}
  }{
    \Analyze{(a, l)}{\mathcal{E}_1 + e}{a'}
  } \qquad
  \inferrule[A-Add-R]{
    \Analyze{(a, l)}{\mathcal{E}_2}{a'}
  }{
    \Analyze{(a, l)}{e_1 + \mathcal{E}_2}{a'}
  } \\
  \inferrule[A-Filter]{
    \Analyze{(a, l)}{\mathcal{E}}{a'}
  }{
    \Analyze{(a, l)}{\Filter{f}{\mathcal{E}}}{a'}
  } \\
  \inferrule[A-Residue-One-Old]{
    l \le l_0 \\
    \Analyze{(a_0, l_0)}{\mathcal{E}}{a'}
  }{
    \Analyze{(a_0, l_0)}{\Residue{a}{\GasOne}{l}{\mathcal{E}}}{a'}
  } \qquad
  \inferrule[A-Residue-One-New]{
    l > l_0 \\
    \Analyze{(a, l)}{\mathcal{E}}{a'}
  }{
    \Analyze{(a_0, l_0)}{\Residue{a}{\GasOne}{l}{\mathcal{E}}}{a'}
  } \\
  \inferrule[A-Residue-All-Old]{
    l \le l_0 \\
    \Analyze{(a_0, l_0)}{\mathcal{E}}{a'}
  }{
    \Analyze{(a_0, l_0)}{\Residue{a}{\GasAll}{l}{\mathcal{E}}}{a'}
  } \qquad
  \inferrule[A-Residue-All-New]{
    l > l_0 \\
    \Analyze{(a, l)}{\mathcal{E}}{a'}
  }{
    \Analyze{(a_0, l_0)}{\Residue{a}{\GasAll}{l}{\mathcal{E}}}{a'}
  } \\
\end{mathpar}

\fbox{\(\DefCtx \mathcal{E}\)} Context \(\mathcal{E}\).
\[
  \DefCtx \mathcal{E}
  = \Mark
  \mid \mathcal{E}(d)
  \mid d(\mathcal{E})
  \mid \mathcal{E} + d
  \mid d + \mathcal{E}
  \mid \Filter{f}{\mathcal{E}}
  \mid \Residue{a}{g}{l}{\mathcal{E}}
\]

\fbox{\(\Value e\)} Expression \(e\) is a value.
\begin{mathpar}
  \inferrule[V-Lam]{
  }{
    \Value{\Lam{x}{e}}
  } \qquad
  \inferrule[V-Nat]{
  }{
    \Value{\Nat{n}}
  }
\end{mathpar}

\fbox{\([v / x] e = e'\)} \(e'\) can be obtained by substitution of \(x\) for
\(v\) in expression \(e\).
\[
  \begin{aligned}
    [v / x] \Nat{n} &= \Nat{n} \\
    [v / x] x &= v \\
    [v / x] y &= y && \text{if } x \neq y \\
    [v / x] (e_1(e_2)) &= ([v / x] e_1)([v / x] e_2) \\
    [v / x] (e_1 + e_2) &= ([v / x] e_1) + ([v / x] e_2) \\
    [v / x] \Lam{y}{e} &= \Lam{y}{[v / x] e} && \text{if } x \neq y \\
    [v / x] \Lam{y}{e} &= \Lam{y}{e} && \text{if } x = y \\
    [v / x] \Filter{(p, a, g)}{e} &= \Filter{([v / x]p, a, g)}{[v / x] e} \\
    [v / x] \Residue{a}{g}{l}{e} &= \Residue{a}{g}{l}{[v / x] e} \\
    [v / x] \Fix{x}{e} &= \Fix{x}{e} \\
    [v / x] \Fix{y}{e} &= \Fix{y}{[v / x]e} && \text{if } x \neq y
  \end{aligned}
\]

\fbox{\([v / x] p = p'\)} \(p'\) can be obtained by substitution of \(x\) for
\(v\) in pattern \(p\).
\[
  \begin{aligned}
    [v / x] \$e &= \$e \\
    [v / x] \$v &= \$v \\
    [v / x] \Nat{n} &= \Nat{n} \\
    [v / x] y &= y && \text{if } x \neq y \\
    [v / x] (p_1(p_2)) &= ([v / x] p_1)([v / x] p_2) \\
    [v / x] (p_1 + p_2) &= ([v / x] p_1) + ([v / x] p_2) \\
    [v / x] \Lam{y}{e} &= \Lam{y}{[v / x] e} && \text{if } x \neq y \\
    [v / x] \Lam{y}{e} &= \Lam{y}{e} && \text{if } x = y \\
    [v / x] \Fix{x}{e} &= \Fix{x}{e} \\
    [v / x] \Fix{y}{e} &= \Fix{y}{[v / x]e} && \text{if } x \neq y
  \end{aligned}
\]

\fbox{\(\Matches{p}{e}\)} means that pattern \(p\) matches expression \(e\).
\begin{mathpar}
  \inferrule[M-All]{\ }{
    \Matches{\$e}{e}
  } \qquad
  \inferrule[M-Val]{\Value{v}}{
    \Matches{\$v}{v}
  } \qquad
  \inferrule[M-Nat]{
    \
  }{
    \Matches{\Nat{m}}{\Nat{m}}
  } \\
  \inferrule[M-Lam]{
    \Strip{e_1} \equiv \Strip{e_2}
  }{
    \Matches{\Lam{x_1}{e_1}}{\Lam{x_2}{e_2}}
  } \qquad
  \inferrule[M-Fix]{
    \Strip{e_1} \equiv \Strip{e_2}
  }{
    \Matches{\Fix{x_1}{e_1}}{\Fix{x_2}{e_2}}
  } \\
  \inferrule[M-Ap]{
    \Matches{p_1}{e_1} \\
    \Matches{p_2}{e_2}
  }{
    \Matches{p_1(p_2)}{e_1(e_2)}
  } \qquad
  \inferrule[M-Add]{
    \Matches{p_1}{e_1} \\
    \Matches{p_2}{e_2}
  }{
    \Matches{p_1 + p_2}{e_1 + e_2}
  }
\end{mathpar}

\fbox{\(\Decompose{e}{\mathcal{E}}{e'}\)} Expression \(e\) can be obtained by putting expression \(e'\) into the mark of \(\mathcal{E}\).
\begin{mathpar}
  % Residue
  \inferrule[D-Residue-T]{
    \Decompose{e}{\mathcal{E}}{e'}
  }{
    \Decompose{\Residue{a}{g}{l}{e}}{\Residue{a}{g}{l}{{\mathcal{E}}}}{e'}
  } \qquad
  \inferrule[D-Residue-E]{
    \Value{v}
  }{
    \Decompose{\Residue{a}{g}{l}{v}}{\circ}{\Residue{a}{g}{l}{v}}
  } \\
  % Filter
  \inferrule[D-Filter-T]{
    \Decompose{e}{\mathcal{E}}{e'}
  }{
    \Decompose{\Filter{f}{e}}{\Filter{f}{\mathcal{E}}}{e'}
  } \qquad
  \inferrule[D-Filter-E]{
    \Value{v}
  }{
    \Decompose{\Filter{f}{v}}{\circ}{\Filter{f}{v}}
  } \\
  % Application
  \inferrule[D-Ap-L]{
    \Decompose{e_1}{\mathcal{E}_1}{e_1'}
  }{
    \Decompose{e_1(e_2)}{\mathcal{E}_1(e_2)}{e_1'}
  } \qquad
  \inferrule[D-Ap-R]{
    \Value{e_1} \\
    \Decompose{e_2}{\mathcal{E}_2}{e_2'}
  }{
    \Decompose{e_1(e_2)}{e_1(\mathcal{E}_2)}{e_2'}
  } \qquad
  \inferrule[D-Ap-E]{
    \Value{e_1} \\
    \Value{e_2}
  }{
    \Decompose{e_1(e_2)}{\circ}{e_1(e_2)}
  } \\
  % Addition
  \inferrule[D-Add-L]{
    \Decompose{e_1}{\mathcal{E}_1}{e_1'}
  }{
    \Decompose{e_1 + e_2}{\mathcal{E}_1 + e_2}{e_1'}
  } \qquad
  \inferrule[D-Add-R]{
    \Value{e_1} \\
    \Decompose{e_2}{\mathcal{E}_2}{e_2'}
  }{
    \Decompose{e_1 + e_2}{e_1 + \mathcal{E}_2}{e_2'}
  } \qquad
  \inferrule[D-Add-E]{
    \Value{e_1} \\
    \Value{e_2}
  }{
    \Decompose{e_1 + e_2}{\circ}{e_1 + e_2}
  } \\
  \inferrule[D-Fix-E]{
    \
  }{
    \Decompose{\Fix{x}{e}}{\circ}{\Fix{x}{e}}
  }
\end{mathpar}

\fbox{\(\Compose{e}{\mathcal{E}}{e'}\)} Expression \(e\) can be obtained by putting expression \(e'\) into the mark of \(\mathcal{E}\).
\begin{mathpar}
  \inferrule[C-Top]{
  }{
    \Compose{e}{\circ}{e}
  } \\
  \inferrule[C-Ap-L]{
    \Compose{e_1}{\mathcal{E}_1}{e_1'}
  }{
    \Compose{e_1(e_2)}{\mathcal{E}_1(e_2)}{e_2'}
  } \qquad
  \inferrule[C-Ap-R]{
    \Compose{e_2}{\mathcal{E}}{e_2'}
  }{
    \Compose{e_1(e_2)}{e_1(\mathcal{E}_2)}{e_2'}
  } \\
  \inferrule[C-Add-L]{
    \Compose{e_1}{\mathcal{E}_1}{e_1'}
  }{
    \Compose{e_1 + e_2}{\mathcal{E}_1 + e_2}{e_1'}
  } \qquad
  \inferrule[C-Add-R]{
    \Compose{e_2}{\mathcal{E}_2}{e_2'}
  }{
    \Compose{e_1 + e_2}{e_1 + \mathcal{E}_2}{e_2'}
  } \\
  \inferrule[C-Filter]{
    \Compose{e}{\mathcal{E}}{e'}
  }{
    \Compose{\Filter{f}{e}}{\Filter{f}{\mathcal{E}}}{e'}
  } \qquad
  \inferrule[C-Residue]{
    \Compose{e}{\mathcal{E}}{e'}
  }{
    \Compose{\Residue{a}{g}{l}{e}}{\mathcal{E}}{e'}
  }
\end{mathpar}

\fbox{\(\Strip{e} = {e'}\)} Strip filter/do expression in expression \(e\) to get expression \(e'\).
\[
  \begin{aligned}
    \Strip{x} &= x \\
    \Strip{\Lam{x}{e}} &= \Lam{x}{\Strip{e}} \\
    \Strip{(e_1(e_2))} &= (\Strip{e_1})(\Strip{e_2}) \\
    \Strip{\Nat{n}} &= \Nat{n} \\
    \Strip{(e_1 + e_2)} &= (\Strip{e_1}) + (\Strip{e_2}) \\
    \Strip{(\Filter{f}{e})} &= \Strip{e} \\
    \Strip{\Residue{a}{g}{l}{e}} &= \Strip{e} \\
    \Strip{\Fix{x}{e}} &= \Fix{x}{\Strip{e}}
  \end{aligned}
\]

\fbox{\(\Transition{e}{e'}\)} \(e\) takes an instruction transition to \(e'\).
\begin{mathpar}
  \inferrule[T-Ap]{
    \Value{e_1} \\
    \Value{e_2}
  }{
    \Transition{(\Lam{x}{e_1})(e_2)}{\FSubst{e_2}{x}{e_1}}
  } \\
  \inferrule[T-Add]{
    \Value{n_1} \\
    \Value{n_2} \\
    n_1 + n_2 = n
  }{
    \Transition{\Nat{n_1} + \Nat{n_2}}{\Nat{n}}
  } \\
  \inferrule[T-Residue]{
    \Value{v}
  }{
    \Transition{\Residue{a}{g}{l}{v}}{v}
  } \qquad
  \inferrule[T-Filter]{
    \Value{v}
  }{
    \Transition{\Filter{f}{v}}{v}
  } \\
  \inferrule[T-Fix]{
    \
  }{
    \Transition{\Fix{x}{e}}{[\Fix{x}{e} / x] e}
  }
\end{mathpar}

\fbox{\(\IsResidue{e}\)} \(e\) is a residue.
\begin{mathpar}
  \inferrule[Is-Residue]{
    \
  }{
    \IsResidue{\Residue{a}{g}{l}{e}}
  }
\end{mathpar}

\fbox{\(\Optimize{e}{e'}\)} \(e\) is optimized to \(e'\)
\begin{mathpar}
  \inferrule[O-Var]{
    \
  }{
    \Optimize{x}{x}
  } \quad
  \inferrule[O-Val]{
    \Value{e}
  }{
    \Optimize{e}{e}
  } \\
  \inferrule[O-Ap]{
    \Optimize{e_l}{e_l'} \\
    \Optimize{e_r}{e_r'}
  }{
    \Optimize{e_l(e_r)}{e_l'(e_r')}
  } \quad
  \inferrule[O-Add]{
    \Optimize{e_l}{e_l'} \\
    \Optimize{e_r}{e_r'}
  }{
    \Optimize{e_l + e_r}{e_l' + e_r'}
  } \\
  \inferrule[O-Filter]{
    \Optimize{e}{e'}
  }{
    \Optimize{\Filter{f}{e}}{\Filter{f}{e'}}
  } \\
  \inferrule[O-Residue-Inner]{
    l_i > l_o \\
    \Optimize{\Residue{a_i}{g_i}{l_i}{e}}{e'}
  }{
    \Optimize{\Residue{a_o}{g_o}{l_o}{\Residue{a_i}{g_i}{l_i}{e}}}{e'}
  } \quad
  \inferrule[O-Residue-Outer]{
    l_i \leq l_o \\
    \Optimize{\Residue{a_o}{g_o}{l_o}{e}}{e'}
  }{
    \Optimize{\Residue{a_o}{g_o}{l_o}{\Residue{a_i}{g_i}{l_i}{e}}}{e'}
  } \quad
  \inferrule[O-Residue-Other]{
    \neg{(\IsResidue{e})} \\
    \Optimize{e}{e'}
  }{
    \Optimize{\Residue{a}{g}{l}{e}}{e'}
  } \\
  \inferrule[O-Fix]{
    \
  }{
    \Optimize{\Fix{x}{e}}{\Fix{x}{e}}
  }
\end{mathpar}

\subsection{Properties}

% \TODO{Properties}
% \\\\
% Conjecture: Idempotence of Instrumentation (Instrumenting twice should have the same effect as instrumenting once)

% $\InstructPAGL{p}{a}{g}{l}{e}{e'} \Rightarrow \InstructPAGL{p}{a}{g}{l}{e'}{e''} \Rightarrow e' = e''$

% \mck{This conjecture is actually not true, because instrumentation always adds more do statements, even if they're already there.}
% \\\\
% Conjecture : Determinism of steps

% $\RuleStep{e}{e'} \Rightarrow \RuleStep{e}{e''} \Rightarrow e' = e''$

% \mck{I think this holds in this calculus, but not in Hazel itself.}
% \\\\
% Conjecture(Unchanging Priorities)

% \(\)

% \hxf{I believe the Unchanging Priorities does holds. I would phrase it as: the residue statements inserted by the same filter statement has the same priorities (or their priorites keeps in order), even across decays and steppings.}

% Conjecture(Correctness)

\begin{conjecture}[Unchanging Priorities]
  The priorities of filter should keep the same across instrumentation, decaying and stepping.
\end{conjecture}

\mck{Since priorities are always re-numbered, we want some sort of conjecture that they remain the same. We could even possibly use this to avoid duplication of `do's?}

% \begin{conjecture}[Necessacity of Priorities]
%   It always pick the filter that is closer to the target expression.
% \end{conjecture}

% \begin{conjecture}[Scoped Filter]
%   Behavior of matched expression is specified by the user at least once, or can be traced by to a filter statement or the default.
% \end{conjecture}

\begin{theorem}[Simulation]
  If an expression \(e\) take \(n\) steps to become \(e'\), then the stripped down version of \(e\), which is \(\Strip{e}\), will takes \(m\) steps to become \(\Strip{e'}\) for some \(m\).
  \[
    \forall e, n, e', \qquad
    \Step{e}{e'}{n} \implies \exists m, \JustStep{\Strip{e}}{\Strip{e'}}{m}
  \]
\end{theorem}

\subsection{Statics}

Optionally, we want to give the filtered stepper calculus so that it can have some reasonable behavior with regard to free variables. The most notable benefit we can obtained from a typing system is that we can enforce that there is no free variable in the pattern and the expression before we test if they match.

We can extend the definition of filtered stepper calculus to include type.
\begin{figure}[h]
  \begin{equation*}
    \begin{array}{rcl}
      \ldots       &         & \ldots \\
      \DefTyp \tau &\Coloneqq& \Natural \\
                   &\mid     & \Arrow{\tau}{\tau}
    \end{array}
  \end{equation*}
  \caption{Syntax of Typed Filtered Stepper Calculus. Only definitions of types are listed, refer to Figure~\ref{fig:filter-syntax} for omitted definitions.}
  \label{fig:typed-filter-syntax}
\end{figure}

\fbox{\(\TypeEntails{\Gamma}{e}{\tau}\)} Under typing context \(\Gamma\), expression \(e\) has type \(\tau\).
\begin{mathpar}
  \inferrule[TE-Var]{
    \TypeContains{\Gamma}{x}{\tau}
  }{
    \TypeEntails{\Gamma}{x}{\tau}
  } \qquad
  \inferrule[TE-Lam]{
    \TypeEntails{\TypeExtends{\Gamma}{x}{\tau_x}}{e}{\tau_e}
  }{
    \TypeEntails{\Gamma}{\Lam{x}{e}}{\Arrow{\tau_x}{\tau_e}}
  } \qquad
  \inferrule[TE-Ap]{
    \TypeEntails{\Gamma}{e_1}{\Arrow{\tau_x}{\tau_e}} \\
    \TypeEntails{\Gamma}{e_2}{\tau_x}
  }{
    \TypeEntails{\Gamma}{e_1(e_2)}{\tau_e}
  } \\
  \inferrule[TE-Nat]{
    \
  }{
    \TypeEntails{\Gamma}{\Nat{n}}{\Natural}
  } \qquad
  \inferrule[TE-Add]{
    \TypeEntails{\Gamma}{e_1}{\Natural} \\
    \TypeEntails{\Gamma}{e_2}{\Natural}
  }{
    \TypeEntails{\Gamma}{e_1 + e_2}{\Natural}
  } \\
  \inferrule[TE-Filter]{
    \TypeEntails{\Gamma}{p}{\tau_p} \\
    \TypeEntails{\Gamma}{e}{\tau_e}
  }{
    \TypeEntails{\Gamma}{\Filter{p}{e}}{\tau_e}
  } \qquad
  \inferrule[TE-Residue]{
    \TypeEntails{\Gamma}{e}{\tau}
  }{
    \TypeEntails{\Gamma}{\Residue{a}{g}{l}{e}}{\tau}
  } \\
  \inferrule[TE-Fix]{
    \TypeEntails{\TypeExtends{\Gamma}{x}{\tau}}{e}{\tau}
  }{
    \TypeEntails{\Gamma}{\Fix{x}{e}}{\tau}
  }
\end{mathpar}

\fbox{\(\TypeEntails{\Gamma}{p}{\tau}\)} Under typing context \(\Gamma\), expression \(p\) has type \(\tau\).
\begin{mathpar}
  \inferrule[TP-Exp]{
  }{
    \TypeEntails{\Gamma}{\PatExpr}{\tau}
  } \qquad
  \inferrule[TP-Val]{
  }{
    \TypeEntails{\Gamma}{\PatValue}{\tau}
  } \\
  \inferrule[TP-Var]{
    \TypeContains{\Gamma}{x}{\tau}
  }{
    \TypeEntails{\Gamma}{x}{\tau}
  } \qquad
  \inferrule[TP-Lam]{
    \TypeEntails{\TypeExtends{\Gamma}{x}{\tau_x}}{e}{\tau_e}
  }{
    \TypeEntails{\Gamma}{\Lam{x}{e}}{\Arrow{\tau_x}{\tau_e}}
  } \qquad
  \inferrule[TP-Ap]{
    \TypeEntails{\Gamma}{p_1}{\Arrow{\tau_x}{\tau_p}} \\
    \TypeEntails{\Gamma}{p_2}{\tau_x}
  }{
    \TypeEntails{\Gamma}{p_1(p_2)}{\tau_e}
  } \\
  \inferrule[TP-Nat]{
    \
  }{
    \TypeEntails{\Gamma}{\Nat{n}}{\Natural}
  } \qquad
  \inferrule[TP-Add]{
    \TypeEntails{\Gamma}{e_1}{\Natural} \\
    \TypeEntails{\Gamma}{e_2}{\Natural}
  }{
    \TypeEntails{\Gamma}{e_1 + e_2}{\Natural}
  } \qquad
  \inferrule[TP-Fix]{
    \TypeEntails{\TypeExtends{\Gamma}{x}{\tau}}{e}{\tau}
  }{
    \TypeEntails{\Gamma}{\Fix{x}{e}}{\tau}
  }
\end{mathpar}

\begin{theorem}[Preservation]
  If a expression has type \(\tau\), then after stepping, the stepped expression also has a type \(\tau\).
  \[
    \forall e, \tau, n, \Gamma, \qquad
    \TypeEntails{\Gamma}{e}{\tau} \land \Step{e}{e'}{n} \implies \TypeEntails{\Gamma}{e'}{\tau}
  \]
\end{theorem}

\begin{theorem}[Progress]
  If an expression is well typed, then there exists an \(n\) such that the expression can be stepped for \(n\) steps.
  \[
    \forall e, \tau, \qquad
    \TypeEntails{\varnothing}{e}{\tau} \implies \exists e' . \Step{e}{e'}{1}
  \]
\end{theorem}

%%% Local Variables:
%%% mode: LaTeX
%%% TeX-master: "main"
%%% End:


\section{Implementation}

% !TEX root = ./main.tex

\subsection{Mini Stepper}

A substitution based small-stepper evaluator of simply-typed lambda
calculus with fix point. The evaluator is implemented in OCaml, and is
used to test the the correctness of the behavior as specify as the
semantic rules in the paper.

\subsection{Hazel Implementation}

Hazel internally applies an big-step, environment-based dynamic
semantics: it helps to avoid unnecessary substitutions, and can be
generally optimized using tail-call optimization. However, since Hazel
is being used as an instruction tool for a functional programming
course, it would be helpful if it can simulate the substitution-based
semantics when students are exploring this concept.

To unify the internal implementation of the stepper, we keep one copy
of the dynamic semantic rules, and having the big-step evaluator and
small-step evaluator interpret the rules. To simulate the substitution
behavior, we post-pocess the output of the stepper, so that all
variable are displayed as it they are substituted, even though the
evaluator/stepper haven't got to that part.

Small-stepper with filter-calculus takes much more computational
resources than the big-step evaluator. Therefore, we applied many
techniques to improve the performance:

\begin{enumerate}
\item During filter-match, we always perform a comparison for the
  physical equality of two OCaml object.
\item We convert an expression to its locally nameless representation
  to compare the alpha-equivalence during matching.
\item We applied some normalization rules to reduce the size of the
  intermediate expression.
\end{enumerate}

% talk about how the big step + small step evaluator abstractions

% environment-based evaluator + post-processing when showing in
% substitution mode

% Optimisation for of the stepper filter: how we remove redundant
% filters (while insure the correctness of the implementation?)

% Equivalency of function:
% \TODO{compare closure & structural equivalence of the body expression}
% Finally we will move to UUID.

%%% Local Variables:
%%% mode: LaTeX
%%% TeX-master: "main"
%%% End:


\section{Evaluation}

% !TEX root = ./main.tex

\subsection{Lecture Integration}

\subsection{Assignment Integration}

\subsubsection{Qualitative Feedback}

\subsubsection{Quantitative Feedback}

We use the stepper calculus during the lecturing of the course EECS
490 at University of Michigan. To better understanding how the stepper
calculus helps students to understand the behavior of their program,
we collected anonymous data from students during one of their
assignments. The number of steps used by studens is shown in the
Figure~\ref{fig:eval-num-steps}.

\begin{figure}[h]
  \centering
  \begin{minipage}{.40\linewidth}
    \begin{subfigure}{\linewidth}
      \includegraphics[width=\textwidth]{images/data-steps-2024-w24-a1.png}
      \caption{Winter 2024, sample size 33}
      \label{fig:eval-num-steps-w24}
    \end{subfigure}
  \end{minipage}
  \quad
  \begin{minipage}{.40\linewidth}
    \begin{subfigure}{\linewidth}
      \includegraphics[width=\textwidth]{images/data-steps-2024-f24-a1.png}
      \caption{Fall 2024, sample size 20}
      \label{fig:eval-num-steps-f24}
    \end{subfigure}
  \end{minipage}
  \caption{Number of steps used by EECS 490 students}
  \label{fig:eval-num-steps}
\end{figure}

As we can see, the number of steps used by students roughly conforms
to a exponential distribution. This implies a small portion of
students use the stepper extensively, while most of the students just
briefly tried it or not using it at all.

%%% Local Variables:
%%% mode: LaTeX
%%% TeX-master: "main"
%%% End:


% \section{Related Work}

% \TODO{other steppers}

% \TODO{Andrew's previous group's one}

% \TODO{algorithmic debugging}

\section{Discussion and Conclusion}

In the evaluation section above, we found out that the stepper is
frequently used by students to help them on their assignments,
however, we don't find many valuable usage recording on the filter
side of the stepper. We have drafted several reasons for this:

\begin{enumerate}
\item The Filtered Stepper Calculus has a rather complex user
  interface, and is more intuitive for user to get hands on.
\item Users have to modify their program to insert filter statements,
  and this usually triggers re-evaluation of their program and can
  takes a long time.
\item The implementation of the filtered stepper is not efficient and
  creates noticable delay when user switch to use a filtered stepper.
\end{enumerate}

Base on these hypothesis, we are planning to develop a more natural,
intuitive, and dynamic user interface, so that programmers can freely
add, change, remove filters during the evaluation without the need to
restart over.

%%
%% The next two lines define the bibliography style to be used, and
%% the bibliography file.
\bibliographystyle{ACM-Reference-Format}
\bibliography{references}

\end{document}
\endinput
%%
%% End of file `sample-acmsmall.tex'.

%%% Local Variables:
%%% mode: latex
%%% TeX-master: t
%%% End:
