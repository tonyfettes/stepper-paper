
Challenge: pedanticism -- lots of boring steps that, once the student understands the idea, just litter the proof and obscure the key idea. Our approach is to develop a scoped filter system that allows us to write patterns that will be skipped or paused. Example:

(more sophisticated example where you show side-by-side the unfiltered and filtered version of the same stepping sequence)
fib, foldright, map

want to be able to step programs with holes, including error holes:
abstract foldright vs foldleft

Prior work summary: what have people tried to do to handle this challenge in the past? why is that not quite enough

our contributions:
- developed the hazel stepper and filter system, implemented it, deployed it in lecture (qualitative evaluation), deployed it to students (qualitative + quantitative evaluation), formalized the filter semantics, metatheory

\section{Hazel Stepper by Example}

In this section, we introduce the Hazel stepper filter.

map example: 

  - pattern matching
  - hide the let at the top
  - eval, and pause

map + debug f
  - debug
  - substitution strikethrough
  - nondeterminism


we want:
 ordering




\mck{I think this content should be moved to the introduction}

We recognized some key capabilities that traditional debuggers have and will be
also helpful to a debugger for functional programming languages.
\begin{enumerate}
  \item Evaluation can be paused at some point and resumed later. The user
    should be able specify where/when the evaluation should be paused. This is
    usually implemented as the breakpoint mechanism in a traditional debugger.
  \item Paused evaluation can be stepped through. traditionally, debuggers will
    provide user with three buttons to ``step through'', ``step over'' and
    ``step out''.
\end{enumerate}

Moreover, we want some extra capabilities that are specific to functional
programming languages and a classroom settings, especially Hazel, to be a
useful tool for teaching and learning functional programming.
\begin{enumerate}
  \item It can skip the stepping process of some \emph{kind} of expression.
  \item It controls the evaluation of the program in a granularity of
  expression, instead of a granularity of line or instruction.
\end{enumerate}

To serve as a good tool for understanding the behaviour of a program, the stepper
filter should have consistent behaviour with the semantics of the language. Hazel
internally uses a environment-based big-step evaluator to evaluate the program,
but it has a consistent behaviour with the substitution-based one. Therefore,
the stepper filter shall not exhibit different behaviour when building upon a
environment-based evaluator or a substitution-based evaluator.

The evaluation order of Hazel is not specified. Therefore, an expression that is
not value might have multiple possible choice 

\mck{end introduction content}


% \subsection{Real life example used in Hazel (490) classroom}

As a motivating example, we will show how our features for stepping through programs can help clarify the execution of a map function.

The map function is implemented in Hazel as follows:

\begin{verbatim}
let map : ([Int], Int -> Int) -> [Int] = fun xs, f ->
  case xs
  | [] => []
  | hd :: tl => f(hd)::map(tl, f)
  end
in
let square = fun x -> x * x in
map([1, 2, 3], square)
\end{verbatim}

This code, when run, will evaluate to the list [1,4,9]. We can also run it in the hazel single stepper to get a full execution trace:

\TODO{Insert execution trace image}

The full execution trace shows each of the small-step transitions that have been taken when evaluating this program. Where a substitution has occurred, it is indicated by the variable name with a strike-through, followed by the new value. The code highlighted green at the bottom indicates where the next small-step transition can be taken.

Seeing the full length of the execution trace can be distracting however, so the Hazel stepper provides some filter primitives that can specify steps to hide. In order to more clearly show the intuition behind the map function, we can hide all steps except for those which show the application of the map function, by inserting the following filters:

\begin{verbatim}
hide $e in
let map = ... in
let square = fun x -> x * x in
pause map($v, $v) in
\end{verbatim}

\TODO{Insert execution trace image}

The \verb|hide $e| construct hides all execution steps. (The \verb|$e| wildcard pattern matches any expression.) The \verb|pause map($v, $v)| construct, since it comes later than the \verb|hide| expression, overrides the \verb|hide| expression to show any steps from an expression that contains a map applied to two values. (The \verb|$v| wildcard pattern matches any value expression.)

\TODO{Insert debug example once it has been debugged}



\subsection{Matching}

Terms that has the same structural form after substituting all bounded variables
are considered \emph{matched}.

\begin{verbatim}
let x = 3 in
eval x + 3 in
x + 3
\end{verbatim}
shall has the same behavior as the program
\begin{verbatim}
let x = 3 in
eval 3 + 3 in
3 + 3
\end{verbatim}

\begin{verbatim}
eval $e in
let g = fun x -> x + x in
pause g($v) in # or pause (fun x -> x + x)($v) in
(fun x -> x + x)(3)
== (fun x -> x + x)(3) # or g(3) #
\end{verbatim}

\begin{figure}[h]
  \includegraphics[width=0.4\textwidth]{images/match-mod-subst.png}
\end{figure}

As in a substitution-based evaluator, the substitution is applied to the
the body of a function:
\begin{verbatim}
eval $e in
let y = fun x -> x in
let h = fun x -> (fun y -> y) in
pause (fun x -> y)($v) in
h(3)
== h(3) # or (fun x -> (fun x -> x))(3) #
\end{verbatim}

\begin{figure}[h]
  \includegraphics[width=0.4\textwidth]{images/match-recursive.png}
\end{figure}

A immediate corollary of this is that the stepper filter shows that
the filter cannot be applied to a variable, since the substitution is always
performed ahead of the matching process, and for the filter it shall not be
able to match against a variable.

\begin{verbatim}
eval $e in
let x = 3 in
pause x in
x + x
== 6
\end{verbatim}

% \TODO{Discuss: whether or not to use alpha-equivalence here.}

\subsection{Four filters: Hide, Eval, Pause and Debug}

% \TODO{Discuss: mention skip all, skip one, pause all and all one?}

Typical use case for the four basic filters.

\subsubsection{Eval}

We want to use the \verb|eval| to \emph{evaluate} all sub-expressions that matches the pattern.
\begin{verbatim}
eval 1 + 2 in
1 + 2
== 3
\end{verbatim}

\subsubsection{Hide}

The \verb|hide| filter will be \emph{used up} when its body
expression is actually being evaluated.

\begin{verbatim}
hide let   =   in   in
let x = 3 in
let y = 4 in
x + y
== 3 + 4
== 7
\end{verbatim}
Compare with \verb|eval|
\begin{verbatim}
eval let   =   in   in
let x = 3 in
let y = 4 in
x + y
== 7
\end{verbatim}

\subsubsection{Pause}

During evaluation, we want to be able to examine every instruction
transitioning steps.

\begin{verbatim}
eval 1 + 2 + 3 + 4 in
pause 3 + 3 in
1 + 2 + 3 + 4
== 3 + 3 + 4
== 10
\end{verbatim}

This would be especially useful when unfolding a higher level
expression to its individual terms, for example I want to know how
many terms of \verb|fib(1)| I need to call to evaluate the whole
\verb|fib(5)|.

\subsubsection{Debug}

On the other hand, we want to went through all the evaluation process of a
sub-expression, for example when debugging the implementation of a function
\begin{verbatim}
let (is_even: Int -> Bool, is_odd: Int -> Bool) = ... in
debug is_odd($v) in
is_even(5)
\end{verbatim}

The difference between \verb|pause| and \verb|debug| is subtle. For
example, considering the following two piece of code:
\begin{verbatim}
eval $e in
pause (1 + 2) + (3 + 4) + (5 + 6) in
eval 3 + 7 + (5 + 6) in
(1 + 2) + (3 + 4) + (5 + 6)
== (1 + 2) + (3 + 4) + (5 + 6)
== 21
\end{verbatim}
and
\begin{verbatim}
eval $e in
debug (1 + 2) + (3 + 4) + (5 + 6) in
eval 3 + 7 + (5 + 6) in
(1 + 2) + (3 + 4) + (5 + 6)
== (1 + 2) + (3 + 4) + (5 + 6)
== 3 + (3 + 4) + (5 + 6)
== 21
\end{verbatim}
The first one will immediately evaluate to final value is because it is a \verb|pause| statement, which will be only effective once.

\subsection{Interaction between filter statements}

% \TODO{Q: Do we need formalise these properties?}

We want nested filter statements to behave correctly, i.e.
\begin{enumerate}
\item For every pattern \verb|p|, \verb|pause p| cancels the effects
  of \verb|eval p| and \verb|hide p|, vice versa.
\item Inner filter statements take precedences.
\end{enumerate}

\subsubsection{``Newer'' Expression overrides ``Older'' Expressions}

\begin{verbatim}
pause 1 + 2 + 3 + 4 in
eval 1 + 2 + 3 + 4 in
1 + 2 + 3 + 4
== 10
\end{verbatim}

\begin{verbatim}
pause $e in
eval 1 + 2 + 3 + 4 in
pause 1 + 2 + 3 + 4 in
1 + 2 + 3 + 4
== 1 + 2 + 3 + 4
\end{verbatim}

\begin{verbatim}
pause $e in
hide (let   =   in  ) in
pause (let   =   in  ) in
let x = 3 in
x + 4
== let x = 3 in x + 4
\end{verbatim}

\subsubsection{The nesting properties should works across bindings}

\begin{verbatim}
eval $e in
let x = 1 in
pause 3 + 3 in
let y = 2 in
eval 3 + 3 in
x + y + 3 + 4
== 10
\end{verbatim}

\begin{verbatim}
let add = fun x, y -> pause $e in x + y in
eval $e in
add(3, 4)
== [3 + 4]
\end{verbatim}


\subsubsection{We want the filter to recover to \emph{older} state when it finish
evaluating matched sub-expression.}

\begin{verbatim}
pause $e in
eval 1 + 2 + 3 + 4 in
pause 3 + 3 in
1 + 2 + 3 + 4
== 3 + 3 + 4
== 10
\end{verbatim}

After evaluating \verb|3 + 3|, the stepper falls back to eval mode
since eval filter matches \verb|1 + 2 + 3 + 4|, so it directly
evaluates to \verb|10|.

We also want a \verb|eval| filter to automatically evaluate all sub-expression
until it cannot proceed.



% \subsection{Handling inconsistency between DHExp and UExp}

% \TODO{Discuss: Move to implementation.tex?}

% There are inconsistencies between the surface expression and expression for
% evaluation in Hazel. For example, fix-points are inserted in the expression
% during elaboration. We want the filters to work with fix-points, with-out user
% acknowledging that they actually needs a fix-point to implement the recursion.

% \begin{verbatim}
% let map : ([Int], Int -> Int) -> [Int] = fun xs, f ->
%   case xs
%   | [] => []
%   | hd :: tl => f(hd)::map(tl, f)
%   end
% in
% let square = fun x -> x * x in
% pause map($v) in
% map([1, 2, 3], square)
% == 1::map([2, 3], f)
% == 1::4::map([3], f)
% \end{verbatim}

% In the example above, we don't want to use to click twice to do unroll and apply, instead we want to merge these two transition
% in one step.

% \subsection{Good but unrealistic for now}

% \TODO{Discuss: Is it better remove in total?}

% There are also something that we think would be intuitive and useful but not possible with current implementation

% \begin{verbatim}
% let fib : Int -> Int =
%   fun n ->
%     if n <= 1 then
%       n
%     else
%       fib(n - 1) + fib(n - 2)
% in
% pause fib($v) + fib($v) in
% fib(5)
% == 3
% \end{verbatim}

% Intuitively we shall see a evaluation trace like this:
% \begin{verbatim}
% ...
% == fib(4) + fib(3)
% == fib(3) + fib(2) + fib(3)
% == fib(3) + fib(2) + fib(2) + fib(1)
% == ...
% \end{verbatim}

% This is no possible since this require us to traverse all possible
% evaluation sequence given a program, which would be super powerful,
% but at the same time super slow.

%%% Local Variables:
%%% mode: latex
%%% TeX-master: "main"
%%% End:


(think about examples that touch on all the edge cases of the system) 

(show substitutions explicitly)

\section{Filter Semantic Rules}

\section{Filtered Stepper Calculus}

In this section, we will discuss the semantics of the \emph{Filtered Stepper
Calculus}. We first introduce the syntax of the language, then we discuss the
the dynamics of the language, and finally we discuss the statics of the
language. The static part is optional for understanding and implement the filtered
stepper calculus, but is useful as a foundation to prove meta-theory theorems
about the calculus, for example, \emph{progression} and \emph{preservation}.

\TODO{introduction to this section}

\subsection{Syntax}

% \TODO{put syntax into a figure}
\begin{figure}[h]
  \begin{equation*}
    \begin{array}{rcl}
      \DefAct a    &\Coloneqq& \ActSkip \mid \ActStep \\
      \DefGas g    &\Coloneqq& \GasOne \mid \GasAll \\
      \DefExp e    &\Coloneqq& x \mid e(e) \mid \Lam{x}{e} \mid e + e \mid \Nat{n} \\
                   &\mid     & \Filter{f}{e} \\
                   &\mid     & \Residue{a}{g}{l}{e} \\
      \DefPat p    &\Coloneqq& x \mid p(p) \mid \Lam{x}{e} \mid p + p \mid \Nat{n} \\
                   &\mid     & \$e \\
                   &\mid     & \$v \\
      \DefFilter f &\Coloneqq& (p, a, g) \\
    \end{array}
  \end{equation*}
  \caption{Syntax of Filtered Stepper Calculus}
  \label{fig:filter-syntax}
\end{figure}

\TODO{overview of syntax}

The syntax of filter stepper calculus (Fig.~\ref{fig:filter-syntax}) is a
extended version of untyped lambda calculus. Additionally, natural numbers and
addition are added to it for some more concrete demonstration. It introduce two
new expression to a untyped lambda calculus language, \texttt{Filter} and
\texttt{Residue}. The \texttt{Filter} expression is visible at user level,
indicating the expression surrounded by the expression is filtered by the
filter. The \texttt{Residue} expression is used internally by the stepper and is
not visible to the user. It is used to signal the \emph{residue} of a effect of
applying a filter to an expression.

A filter is a tuple of a pattern, an action, and a gas. The pattern
is a mutated of the term of untyped lambda calculus. Apart from constructs like
variables, applications, and lambdas, it also includes some new constructs, a
forall pattern \(\$e\), a value pattern \(\$v\). The action is a tag indicating
the action to be taken when the pattern matches an expression. It has two
form, \(\ActSkip\) and \(\ActStep\). The gas is another tag indicating the
live time of the effect of a filter. It has two form, \(\GasOne\) and
\(\GasAll\).

The residue expression \(\Residue{a}{g}{l}{e}\) is a expression that is used to
store the residue of the effect of a filter. It has four components, an action
\(a\), a gas \(g\), a priority \(l\), and a expression \(e\). The action and the
gas are the same as the ones in a filter. The priority is a natural number
indicating the priority of the residue.

% \fbox{\(\DefAct a\)} Action \(a\) of a filter
% \[
%   \DefAct a \Coloneqq \ActSkip \mid \ActStep
% \]

% \fbox{\(\DefGas g\)} Gas \(g\) of a filter.
% \[
%   \DefGas g \Coloneqq \GasOne \mid \GasAll
% \]

% \fbox{\(\DefExp d\)} Expression \(d\).
% \[
%   \DefExp d \Coloneqq x \mid d(d) \mid \Lam{x}{d} \mid d + d \mid \Nat{n} \mid \Filter{(p, a, g)}{d} \mid \Residue{a}{g}{l}{d}
% \]

% \fbox{\(\DefVal d\)} Value \(v\).
% \[
%   \DefVal v \Coloneqq \Lam{x}{d} \mid \Nat{n}
% \]

% \mck{I added a value thing here but we should probably actually define it as a judgement instead.}

% \mck{Maybe replace $\Residue{a}{g}{l}{\mathcal{E}}$ with $\Residue{a}{\GasAll}{l}{\mathcal{E}}$ because only $\GasAll$ is possible?}

% \fbox{\(\DefPat p\)} Pattern \(p\).
% \[
%   \DefPat p \Coloneqq \$e \mid \$v \mid x \mid p(p) \mid \Lam{x}{d} \mid p + p \mid \Nat{n}
% \]

% \mck{The $\lambda \$x. $ form actually isn't necessary because we use alpha-equivalence later.}

% \fbox{\(\DefFilter f\)} Filter \(f\).
% \[
%   \DefFilter f = (p, a, g)
% \]

%\TODO{change notation of filter to use $f$}

%\TODO{change notation for do to be more compact}

\subsection{Contextual Dynamics}

\TODO{overview of dynamics judgements, starting from the main judgement}

\TODO{motivate the two rules}

\fbox{\(\FStep{(a, g)}{d}{d'}\)} Expression \(d\) steps to expression \(d'\).
\begin{mathpar}
  \inferrule[FS-Step]{
    \FInstruct{(a, g, 0)}{d}{d_i} \\
    \Decompose{d_i}{\mathcal{E}}{d_0} \\
    \FTrans{d_0}{d_0'} \\
    \Compose{d'}{\mathcal{E}}{d_0'}
  }{
    \FStep{(p, a, g)}{d}{d'}
  } \\
  \inferrule[FS-Skip]{
    \FInstruct{(a, g, 0)}{d}{d_i} \\
    \Decompose{d_i}{\mathcal{E}}{d_0} \\
    \FTrans{d_0}{d_0'} \\
    \Compose{d'}{\mathcal{E}}{d_0'} \\
    \FStep{(a, g)}{d'}{d''}
  }{
    \FStep{(a, g)}{d}{d''}
  }
\end{mathpar}

\TODO{hard code defaults instead of making them part of the judgement}

\mck{Does this work with 0-step evaluations e.g. skip \$e in 1 + 2 + 3}


\fbox{\(\DefCtx \mathcal{E}\)} Context \(\mathcal{E}\).
\[
  \DefCtx \mathcal{E}
  = \FCMark
  \mid \mathcal{E}(d)
  \mid d(\mathcal{E})
  \mid \mathcal{E} + d
  \mid d + \mathcal{E}
  \mid \Filter{(p, a, g)}{l}{\mathcal{E}}
  \mid \Residue{a}{g}{l}{\mathcal{E}}
\]


\fbox{\(\Value e\)} Expression \(e\) is a value.
\begin{mathpar}
  \inferrule[V-Lam]{
  }{
    \Value{\Lam{x}{d}}
  } \qquad
  \inferrule[V-Nat]{
  }{
    \Value{\Nat{n}}
  }
\end{mathpar}

\fbox{\([x / v] e = e'\)} \(e'\) can be obtained by substitution of \(x\) for
\(v\) in \(e\).
\[
  \begin{aligned}
    [x / v] x &= v \\
    [x / v] y &= y && \text{if } x \neq y \\
    [x / v] (e_1(e_2)) &= ([x / v] e_1)([x / v] e_2) \\
    [x / v] \Lam{y}{\tau}{e} &= \Lam{y}{\tau}{[x / v] e} && \text{if } x \neq y \\
    [x / v] \Lam{y}{\tau}{e} &= \Lam{y}{\tau}{e} && \text{if } x = y \\
  \end{aligned}
\]

% \begin{mathpar}
%   \inferrule[S-Var-Eq]{
%   }{
%     [x / v] x = v
%   } \qquad
%   \inferrule[S-Var-Neq]{
%     x \neq y
%   }{
%     [x / v] y = y
%   } \\
%   \inferrule[S-Ap]{
%   }{
%     [x / v] (e_1(e_2)) = ([x / v] e_1)([x / v] e_2)
%   } \\
%   \inferrule[S-Lam-Eq]{
%   }{
%     [x / v] \Lam{x}{e} = \Lam{x}{e}
%   } \qquad
%   \inferrule[S-Lam-Neq]{
%     x \neq y
%   }{
%     [x / v] \Lam{y}{\tau}{e} = \Lam{y}{\tau}{[x / v] e}
%   } \\
% \end{mathpar}
\TODO{swap order to [v/x]e}

\TODO{missing filter and do cases}

\TODO{missing number literal and addition cases}

\TODO{use e instead of d everywhere}

\TODO{substitution into patterns}

\TODO{fix Lam macro}

\TODO{combine decompose and recompose?}

\TODO{add function that removes do one to formalism}

\mck{It might be clearer if we write out a separate recompose judgement too. I think decompose and recompose behave differently especially with regard to do statements}

\fbox{\(\Decompose{d}{\mathcal{E}}{d'}\)} Expression \(d\) can be obtained by putting expression \(d'\) into the mark of \(\mathcal{E}\).
\begin{mathpar}
  \inferrule[D-Var]{
  }{
    \Decompose{x}{\FCMark}{x}
  } \\
  \inferrule[D-Residue-I]{
    \Decompose{e}{\mathcal{E}}{e'}
  }{
    \Decompose{\Residue{a}{g}{l}{e}}{\mathcal{E}}{\Residue{a}{g}{l}{e'}}
  } \\
  \inferrule[D-Residue-E]{
    \Value{v}
  }{
    \Decompose{\Residue{a}{g}{l}{v}}{\circ}{\Residue{a}{g}{l}{v}}
  } \\
  \inferrule[D-Filter-I]{
    \Decompose{e}{\mathcal{E}}{e'}
  }{
    \Decompose{\Filter{(p, a, g)}{e}}{\mathcal{E}}{\Filter{(p, a, g)}{e'}}
  } \\
  \inferrule[D-Filter-E]{
    \Value{v}
  }{
    \Decompose{\Residue{p}{a}{g}{v}}{\circ}{\Residue{p}{a}{g}{v}}
  } \\
  \inferrule[D-Ap-L]{
    \Decompose{d_1}{\mathcal{E}_1}{d_1'}
  }{
    \Decompose{d_1(d_2)}{\mathcal{E}_1(d_2)}{d_1'}
  } \qquad
  \inferrule[D-Ap-R]{
    \Value{d_1} \\
    \Decompose{d_2}{\mathcal{E}_2}{d_2'}
  }{
    \Decompose{d_1(d_2)}{d_1(\mathcal{E}_2)}{d_2'}
  } \qquad
  \inferrule[D-Ap-E]{
    \Value{e_1} \\
    \Value{e_2}
  }{
    \Decompose{e_1(e_2)}{\circ}{e_1(e_2)}
  } \\
  \inferrule[D-Add-L]{
    \Decompose{d_1}{\mathcal{E}_1}{d_1'}
  }{
    \Decompose{d_1 + d_2}{\mathcal{E}_1 + d_2}{d_1'}
  } \qquad
  \inferrule[D-Add-R]{
    \Value{d_1} \\
    \Decompose{d_2}{\mathcal{E}_2}{d_2'}
  }{
    \Decompose{d_1 + d_2}{d_1 + \mathcal{E}_2}{d_2'}
  } \qquad
  \inferrule[D-Add-E]{
    \Value{e_1} \\
    \Value{e_2}
  }{
    \Decompose{e_1 + e_2}{\circ}{e_1 + e_2}
  }
\end{mathpar}
\mck{D-Filter-E rule does not have any filters in it}

\mck{The last four rules seem to specify an evaluation order - is this desired behaviour? I thought we were going to mention the nondeterminism somewhere in the text}

\mck{Todo: remove the unused `output' g from the above - I think we can also remove the `input` g too?}

\fbox{\(\Compose{d}{\mathcal{E}}{d'}\)} Expression \(d\) can be obtained by putting expression \(d'\) into the mark of \(\mathcal{E}\).
\begin{mathpar}
  \inferrule[C-Top]{
    \
  }{
    \Compose{e}{\circ}{e}
  } \qquad
  \inferrule[C-Ap-L]{
    \Compose{e_l}{\mathcal{E}}{e}
  }{
    \Compose{e_l(e_r)}{\mathcal{E}(e_r)}{e}
  } \qquad
  \inferrule[C-Ap-R]{
    \Compose{e_r}{\mathcal{E}}{e}
  }{
    \Compose{e_l(e_r)}{e_l(\mathcal{E})}{e}
  } \\
  \inferrule[C-Add-L]{
    \Compose{e_l}{\mathcal{E}}{e}
  }{
    \Compose{e_l + e_r}{\mathcal{E} + e_r}{e}
  } \qquad
  \inferrule[C-Add-R]{
    \Compose{e_r}{\mathcal{E}}{e}
  }{
    \Compose{e_l + e_r}{e_l + \mathcal{E}}{e}
  } \\
  \inferrule[C-Filter]{
    \Compose{e'}{\mathcal{E}}{e}
  }{
    \Compose{\Filter{(p, a, g)}{e'}}{\Filter{(p, a, g)}{\mathcal{E}}}{e}
  } \\
  \inferrule[C-Residue]{
    \Compose{e'}{\mathcal{E}}{e}
  }{
    \Compose{\Residue{a}{g}{l}{e'}}{\Residue{a}{g}{l}{\mathcal{E}}}{e}
  }
\end{mathpar}

\fbox{\(e_1 \equiv_\alpha e_2\)} \(e_1\) is alpha-equivalent to \(e_2\).
\begin{mathpar}
  \inferrule[\(\alpha\)-Var]{
  }{
    x \equiv_\alpha x
  } \qquad
  \inferrule[\(\alpha\)-Lam]{
    e_1 \equiv_\alpha [x_2/x_1](e_2)
  }{
    \Lam{x_1}{\tau_1}{e_1} \equiv_\alpha \Lam{x_2}{\tau_2}{e_2}
  } \\
  \inferrule[\(\alpha\)-Ap]{
    e_1 \equiv_\alpha e_3 \\
    e_2 \equiv_\alpha e_4
  }{
    e_1(e_2) \equiv_\alpha e_3(e_4)
  } \qquad
  \inferrule[\(\alpha\)-Add]{
    e_1 \equiv_\alpha e_3 \\
    e_2 \equiv_\alpha e_4
  }{
    e_1 + e_2 \equiv_\alpha e_3 + e_4
  } \\
  \inferrule[\(\alpha\)-Filter]{
    e_1 \equiv_\alpha e_2
  }{
    \Filter{(p, a, g)}{e_1} \equiv_\alpha \Filter{(p, a, g)}{e_2}
  } \\
  \inferrule[\(\alpha\)-Residue]{
    e_1 \equiv_\alpha e_2
  }{
    \Residue{a}{g}{l}{e_1} \equiv_\alpha \Residue{a}{g}{l}{e_2}
  }
\end{mathpar}

\fbox{\(\Strip{e} = {e'}\)} Strip expression \(e\) to get expression \(e'\).
\[
  \begin{aligned}
    \Strip{x} &= x \\
    \Strip{\Lam{x}{e}} &= \Lam{x}{\Strip{e}} \\
    \Strip{e_1(e_2)} &= (\Strip{e_1})(\Strip{e_2}) \\
    \Strip{\Nat{n}} &= \Strip{\Nat{b}} \\
    \Strip{e_1 + e_2} &= (\Strip{e_1}) + (\Strip{e_2}) \\
    \Strip{\Filter{(p, a, g)}{e}} &= \Strip{e} \\
    \Strip{\Residue{a}{g}{l}{e}} &= \Strip{e}
  \end{aligned}
\]

\fbox{\(\Matches{p}{d}\)} means that pattern \(p\) matches expression \(d\).
\begin{mathpar}
  \inferrule[M-All]{\ }{
    \Matches{\$e}{d}
  } \qquad
  \inferrule[M-Val]{\Value{v}}{
    \Matches{\$v}{v}
  } \\
  \inferrule[M-Nat]{
    m = n
  }{
    \Matches{\Nat{m}}{\Nat{n}}
  }\qquad
  \inferrule[M-Lam]{
    \Lam{x}{\tau_p}{e_p} \equiv_{\alpha} \Lam{y}{\tau_e}{e_e}
  }{
    \Matches{\Lam{x}{\tau_p}{e_p}}{\Lam{y}{\tau_e}{e_e}}
  } \\
  \inferrule[M-Ap]{
    \Matches{p_1}{d_1} \\
    \Matches{p_2}{d_2}
  }{
    \Matches{p_1(p_2)}{d_1(d_2)}
  } \qquad
  \inferrule[M-Add]{
    \Matches{p_1}{d_1} \\
    \Matches{p_2}{d_2}
  }{
    \Matches{p_1 + p_2}{d_1 + d_2}
  } \qquad
\end{mathpar}

\mck{M-Lam doesn't allow \$e inside function bodies - is this intentional?}

\fbox{\(\FInstruct{(p, a, g, l)}{d}{d'}\)} \(d\) is dynamically instructed to \(d'\).
\begin{mathpar}
  \inferrule[FI-V]{
    \Value{d}
  }{
    \FInstruct{(p, a, g, l)}{d}{d}
  } \\
  \inferrule[FI-Var-Y]{
    \FPatMatchesExp{p}{x}
  }{
    \FInstruct{(p, a, g, l)}{x}{\Residue{a}{g}{l}{x}}
  } \qquad
  \inferrule[FI-Var-N]{
    \FPatNotMatchesExp{p}{x}
  }{
    \FInstruct{(p, a, g, l)}{x}{x}
  } \\
  \inferrule[FI-I]{
    \FInstruct{(p_0, a_0, g_0, l_0)}{d_0}{d} \\
    \FInstruct{(p, a, g, l_0 + 1)}{d}{d'}
  }{
    \FInstruct{(p_0, a_0, g_0, l_0)}{\Filter{(p, a, g)}{d_0}}{\Filter{(p, a, g)}{d'}}
  } \\
  \inferrule[FI-T]{
    \FInstruct{(p_0, a_0, g_0, l_0)}{d_0}{d} \\
  }{
    \FInstruct{(p_0, a_0, g_0, l_0)}{\Residue{a}{g}{l}{d_0}}{\Residue{a}{g}{l}{d'}}
  } \\
  \inferrule[FI-Ap-Y]{
    \FInstruct{(p, a, g, l)}{d_1}{d_1'} \\
    \FInstruct{(p, a, g, l)}{d_2}{d_2'} \\
    \FPatMatchesExp{p}{d_1(d_2)}
  }{
    \FInstruct{(p, a, g, l)}{d_1(d_2)}{\Residue{a}{g}{l}{d_1'(d_2')}}
  } \\
  \inferrule[FI-Ap-N]{
    \FInstruct{(p, a, g, l)}{d_1}{d_1'} \\
    \FInstruct{(p, a, g, l)}{d_2}{d_2'} \\
    \FPatNotMatchesExp{p}{d_1(d_2)}
  }{
    \FInstruct{(p, a, g, l)}{d_1(d_2)}{d_1'(d_2')}
  } \\
  \inferrule[FI-Add-Y]{
    \FInstruct{(p, a, g, l)}{d_1}{d_1'} \\
    \FInstruct{(p, a, g, l)}{d_2}{d_2'} \\
    \FPatMatchesExp{p}{d_1 + d_2}
  }{
    \FInstruct{(p, a, g, l)}{d_1(d_2)}{\Residue{a}{g}{l}{d_1' + d_2'}}
  } \\
  \inferrule[FI-Add-N]{
    \FInstruct{(p, a, g, l)}{d_1}{d_1'} \\
    \FInstruct{(p, a, g, l)}{d_2}{d_2'} \\
    \FPatNotMatchesExp{p}{d_1 + d_2}
  }{
    \FInstruct{(p, a, g, l)}{d_1(d_2)}{d_1' + d_2'}
  }
\end{mathpar}

\mck{I think there's some preservation of priority property to prove here, since the numbers are re-generated on every pass.}

\mck{This instrumentation will regenerate the do statements on every pass - I guess it doesn't cause problems but it would be nice if it didn't?}

\fbox{\(\Analyze{(a, l)}{\mathcal{E}}{\mathcal{E}'}{a'}\)} Under the filter environment of \((a, l)\), the context \(\mathcal{E}\) is transitioned to \(\mathcal{E}'\) and the action \(a'\) is returned.
\begin{mathpar}
  \inferrule[A-Var]{
  }{
    \Analyze{(a, l)}{\circ}{\circ}{a}
  } \\
  \inferrule[A-Ap-L]{
    \Analyze{(a, l)}{\mathcal{E}_1}{\mathcal{E}_1'}{a'}
  }{
    \Analyze{(a, l)}{\mathcal{E}_1(d)}{\mathcal{E}_1'(d)}{a'}
  } \qquad
  \inferrule[A-Ap-R]{
    \Analyze{(a, l)}{\mathcal{E}_2}{\mathcal{E}_2'}{a'}
  }{
    \Analyze{(a, l)}{d_1(\mathcal{E}_2)}{d_1(\mathcal{E}_2')}{a'}
  } \\
  \inferrule[A-Add-L]{
    \Analyze{(a, l)}{\mathcal{E}_1}{\mathcal{E}_1'}{a'}
  }{
    \Analyze{(a, l)}{\mathcal{E}_1 + d}{\mathcal{E}_1' + d}{a'}
  } \qquad
  \inferrule[A-Add-R]{
    \Analyze{(a, l)}{\mathcal{E}_2}{\mathcal{E}_2'}{a'}
  }{
    \Analyze{(a, l)}{d_1 + \mathcal{E}_2}{d_1 + \mathcal{E}_2'}{a'}
  } \\
  \inferrule[A-Filter]{
    \Analyze{(a, l)}{\mathcal{E}}{\mathcal{E}'}{a'}
  }{
    \Analyze{(a, l)}{\Filter{(p, a, g)}{\mathcal{E}}}{\Filter{(p, a, g)}{\mathcal{E}'}}{a'}
  } \\
  \inferrule[A-Residue-One-Old]{
    l \le l_0 \\
    \Analyze{(a_0, l_0)}{\mathcal{E}}{\mathcal{E}'}{a'}
  }{
    \Analyze{(a_0, l_0)}{\Residue{a}{\GasOne}{l}{\mathcal{E}}}{\mathcal{E}'}{a'}
  } \\
  \inferrule[A-Residue-One-New]{
    l > l_0 \\
    \Analyze{(a, l)}{\mathcal{E}}{\mathcal{E}'}{a'}
  }{
    \Analyze{(a_0, l_0)}{\Residue{a}{\GasOne}{l}{\mathcal{E}}}{\mathcal{E}'}{a'}
  } \\
  \inferrule[A-Residue-All-Old]{
    l \le l_0 \\
    \Analyze{(a_0, l_0)}{\mathcal{E}}{\mathcal{E}'}{a'}
  }{
    \Analyze{(a_0, l_0)}{\Residue{a}{\GasAll}{l}{\mathcal{E}}}{\Residue{a}{\GasAll}{l}{\mathcal{E}'}}{a'}
  } \\
  \inferrule[A-Residue-All-New]{
    l > l_0 \\
    \Analyze{(a, l)}{\mathcal{E}}{\mathcal{E}'}{a'}
  }{
    \Analyze{(a_0, l_0)}{\Residue{a}{\GasAll}{l}{\mathcal{E}}}{\Residue{a}{\GasAll}{l}{\mathcal{E}'}}{a'}
  }
\end{mathpar}

\fbox{\(\FTrans{d}{d'}\)} \(d\) takes an transition to \(d'\).
\begin{mathpar}
  \inferrule[FTLam]{
    \Value{d_2}
  }{
    \FTrans{\Lam{x}{d_1}(d_2)}{\FSubst{d_2}{x}{d_1}}
  } \\
  \inferrule[FTNat]{
    n_1 + n_2 = n
  }{
    \FTrans{\Nat{n_1} + \Nat{n_2}}{\Nat{n}}
  } \\
  \text{\mck{Is FLRes-T redundant since we're using evaluation contexts?}}\\
  \inferrule[FLRes-T]{
    \FTrans{d}{d'}
  }{
    \FTrans{\Residue{a}{g}{l}{e}}{\Residue{a}{g}{l}{e'}}
  } \\
  \inferrule[\textcolor{red}{FLRes-Inst}]{
    \Value{v}
  }{
    \FTrans{\Residue{a}{g}{l}{d}}{v}
  } \qquad
  \inferrule[\textcolor{red}{FLRes-Filter}]{
    \Value{v}
  }{
    \FTrans{\Filter{a}{g}{d}}{v}
  }
\end{mathpar}

\mck{ok yeah it would be nice to have a progress property here but I guess we do need types for that...}



\TODO{Properties}
\\\\
Conjecture: Idempotence of Instrumentation (Instrumenting twice should have the same effect as instrumenting once)

$\FInstruct{(p, a, g, l)}{d}{d'} \Rightarrow \FInstruct{(p, a, g, l)}{d'}{d''} \Rightarrow d' = d''$

\mck{This conjecture is actually not true, because instrumentation always adds more do statements, even if they're already there.}
\\\\
Conjecture : Determinism of steps

$\FStep{(a, g)}{d}{d'} \Rightarrow \FStep{(a, g)}{d}{d''} \Rightarrow d' = d''$

\mck{I think this holds in this calculus, but not in Hazel itself.}
\\\\
Conjecture(Unchanging Priorities)

???

\mck{Since priorities are always re-numbered, we want some sort of conjecture that they remain the same. We could even possibly use this to avoid duplication of `do's?}
\\\\
Conjecture(Correctness)




\subsection{Statics}

\TODO{Get filter a simple type system like STLC}

\TODO{Gas is not useful in return}

\TODO{Better notation for do's statements}



%%% Local Variables:
%%% mode: latex
%%% TeX-master: "main"
%%% End:


\section{Implementation}

\subsection{Mini Stepper}

\subsection{Hazel Implementation}

talk about how the big step + small step evaluator abstractions

environment-based evaluator + post-processing when showing in substitution mode

\section{Evaluation}

\subsection{Lecture Integration}

\subsection{Assignment Integration}
\subsubsection{Qualitative Feedback}
\subsubsection{Quantitative Feedback}

\section{Related Work}
other steppers
Andrew's previous group's one
algorithmic debugging

\section{Discussion and Conclusion}

applications to agda unfolding?

%%
%% The next two lines define the bibliography style to be used, and
%% the bibliography file.
\bibliographystyle{ACM-Reference-Format}
\bibliography{references}

\end{document}
\endinput
%%
%% End of file `sample-acmsmall.tex'.

%%% Local Variables:
%%% mode: latex
%%% TeX-master: t
%%% End:
