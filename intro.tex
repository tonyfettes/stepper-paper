The debugger has been one of the most important tools that students and programmers
use to understand the behavior of their programs. Well-established debuggers such as
GDB, LLDB, and Visual Studio Debugger are widely used in industry and
academia. These debuggers provide a rich set of tools for users to inspect and even
modify various aspects of a running program, such as memory, variables,
and call stacks. However, they also suffer from several limitations.

\begin{enumerate}
  \item They are tailored for imperative languages like C or C++ and are not
    well-suited for functional languages, where memory is usually managed by
    garbage collection and is not directly manipulated by the programmer.
  \item Many of them support stepping in and stepping over instructions. However,
    these options can be tedious to use when the program is large and
    complex, as the user has to step through many instructions that are irrelevant
    to the current task.
  \item They are not well-suited for teaching. They are designed for professional
    programmers who are already familiar with the language and the program they
    are debugging. They are not ideal for beginners learning the
    language and the program.
\end{enumerate}

We recognize that some key capabilities in traditional debuggers
significantly aid programmers, and we believe it would also be useful to incorporate
these features in a debugger for functional programming languages.
\begin{enumerate}
  \item Evaluation can be paused at specific points and resumed later. Users
    should be able to specify when and where evaluation should be paused, typically
    implemented through breakpoints in traditional debuggers.
  \item Paused evaluations can be stepped through. Traditionally, debuggers provide
    users with three options: ``step through,'' ``step over,'' and ``step out.''
\end{enumerate}

Furthermore, we want to introduce additional capabilities tailored specifically to functional
programming languages and classroom settings, especially in the context of Hazel, to make it a
valuable tool for teaching and learning functional programming.
\begin{enumerate}
  \item It can skip the stepping process for certain \emph{kinds} of expressions.
  \item It controls program evaluation at the expression level rather than at
    the line or instruction level.
  \item Hazel’s evaluation order is unspecified, so expressions that are not values
    may have multiple evaluation paths. We believe it is helpful to model this property
    in the debugger.
\end{enumerate}

% To serve as a good tool for understanding program behavior, the stepper
% filter should be consistent with the semantics of the language. Hazel
% uses an environment-based big-step evaluator to evaluate programs,
% which is consistent with a substitution-based evaluator. Therefore,
% the stepper filter should not exhibit different behavior when built on
% an environment-based or a substitution-based evaluator.

We aim to develop a new kind of debugger tailored for
functional languages and suitable for teaching in Hazel. Hazel is a pure
functional language that supports working with incomplete programs and is used
in an upper-level programming languages course, where we teach
substitution-based equational semantics. We want the steps to be presented as
justified proof steps, mirroring what we cover in lectures and assignments.

% \TODO{(example here of a simple equational stepping proof)}

One of the main challenges in using such a debugger in lectures is
that it is often tedious to step through a program. Students must
witness many mundane instructions that are either irrelevant to the current
task or are already well understood. These instructions clutter the proof and
obscure key ideas. Our approach is to develop a scoped filter system that allows
us to define patterns, instructing the stepper on which expressions to skip or pause
during evaluation.

For example, suppose the user wants to debug
a program to compute the value of \(n!\), as shown in
Figure~\ref{fig:intro-fac}. Without a filtering mechanism,
they would likely have to step through the program step-by-step continuously.
(Figure~\ref{fig:intro-fac-unfiltered})
However, by applying a set of filters instructing the debugger to
``stop at specific patterns,'' the execution trace becomes more concise and
focused on steps relevant to the user. (Figure~\ref{fig:intro-fac-filtered})

\begin{figure}[h]
  \begin{center}
    \begin{subfigure}[t]{0.48\textwidth}
      \centering
      \includegraphics[width=\textwidth]{images/intro-fac-unfiltered.png}
      \caption{Program trace without filters.}
      \label{fig:intro-fac-unfiltered}
    \end{subfigure}
    \quad
    \begin{subfigure}[t]{0.48\textwidth}
      \centering
      \includegraphics[width=\textwidth]{images/intro-fac-filtered.png}
      \caption{Program trace with filters that stop at function application.}
      \label{fig:intro-fac-filtered}
    \end{subfigure}
  \end{center}
  \caption{Program traces with and without filters.}
  \label{fig:intro-fac}
\end{figure}

% \TODO{want to be able to step through programs with holes, including error holes:
% abstract foldright vs foldleft}

We have made the following contributions in this paper:
\begin{itemize}
  \item We developed the Hazel stepper and a companion scoped filter system,
    allowing users to specify sub-expressions they are uninterested in fully
    evaluating, sub-expressions they are interested in, specific steps to skip, and
    specific steps to observe.
  \item We deployed the Hazel stepper and the scoped filter system in
    lectures and with students, conducting both qualitative and quantitative
    evaluations of the system.
  \item We formalized the filter semantics and proved the meta-theoretical
    properties of the Hazel filter system to ensure that the system behaves as intended.
\end{itemize}

In the following sections, we will first show examples of the
Hazel stepper. We will then discuss the details of the semantics of
the calculus used by the stepper, called \emph{filtered stepper
  calculus}. After this theoretical discussion, we will briefly explain
how the stepper is implemented. Finally, we will present our
evaluation results based on usage statistics from assignments.

% - developed the hazel stepper and filter system, implemented it,
% deployed it in lecture (qualitative evaluation), deployed it to
% students (qualitative + quantitative evaluation), formalized the
% filter semantics, metatheory

%%% Local Variables:
%%% mode: LaTeX
%%% TeX-master: "main"
%%% End:
