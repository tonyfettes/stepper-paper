Debugger has been one of the most important tools that students and programmer
use to learn the behavior of their programs. Well-established debuggers such as
GDB, LLDB, and Visual Studio Debugger have been widely used in industry and
academia. These debuggers provide rich set of tools for user to inspect and even
change various aspects of a running program, such as memory, variables,
and call stacks. However, they also suffers from several limitations.

\begin{enumerate}
  \item They are tailored for imperative languages like C or C++. They are not
    well-suited for functional languages, where memory is usually managed by
    GC and never directly manipulated by the programmer.
  \item Many of them support stepping in and stepping over instructions. However,
    these two instructions can be tedious to use when the program is large and
    complex. The user has to step through many instructions that are not relevant
    to the current task.
  \item They are not well-suited for teaching. They are designed for professional
    programmers who are already familiar with the language and the program they
    are debugging. They are not well-suited for beginners who are learning the
    language and the program.
\end{enumerate}

We recognized some key capabilities in these traditional debuggers do
help programmers a lot, and we think it will be also helpful to have
them as well in a debugger for functional programming languages.
\begin{enumerate}
  \item Evaluation can be paused at some point and resumed later. The user
    should be able specify where/when the evaluation should be paused. This is
    usually implemented as the breakpoint mechanism in a traditional debugger.
  \item Paused evaluation can be stepped through. traditionally, debuggers will
    provide user with three buttons to ``step through'', ``step over'' and
    ``step out''.
\end{enumerate}

Moreover, we want some extra capabilities that are specific to functional
programming languages and a classroom settings, especially Hazel, to be a
useful tool for teaching and learning functional programming.
\begin{enumerate}
  \item It can skip the stepping process of some \emph{kind} of expression.
  \item It controls the evaluation of the program in a granularity of
    expression, instead of a granularity of line or instruction.
  \item The evaluation order of Hazel is not specified. Therefore, an
    expression that is not value might have multiple possible choice,
    and we think it is helpful to model this property in the debugger.
\end{enumerate}

% To serve as a good tool for understanding the behaviour of a program, the stepper
% filter should have consistent behaviour with the semantics of the language. Hazel
% internally uses a environment-based big-step evaluator to evaluate the program,
% but it has a consistent behaviour with the substitution-based one. Therefore,
% the stepper filter shall not exhibit different behaviour when building upon a
% environment-based evaluator or a substitution-based evaluator.

Therefore, we want to develop a new kind of debugger that tailored for
functional languages and is well-suited for teaching in Hazel. Hazel is a pure
functional language that supports working with incomplete programs. It is used
in a upper-level programming languages course where we are teaching
substitution-based equational semantics -- we want the steps to show up as
justified proof steps, to mirror what we have previously done in lectures and
written assignments.

% \TODO{(example here of a simple equational stepping proof)}

One of the main challenges in using such a debugger in lecturing is
that it is often tedious to step through a program. Students has to
witness many boring instructions that are not relevant to the current
task, or that they already well understand. Those instructions will
litter the proof and obscure the key idea. Our approach is to develop
a scoped filter system that allows us to write patterns that tells the
stepper the evaluation of which expressions should be skipped or paused.

For example, suppose the user want to debug through
a program with evaluate the value of \(n!\), as in
Figure~\ref{fig:intro-fac}. With a debug that comes with no filtering
mechanism, they probably will have to step though the program in
single steps all the time. (Figure~\ref{fig:intro-fac-unfiltered})
However, if we applies a set of filters to instruct the debugger to
``stop when you see the given pattern'', then the execution trace will
be much smaller, and focused on steps that the user cares
about. (Figure~\ref{fig:intro-fac-filtered})

\begin{figure}[h]
  \begin{center}
    \begin{subfigure}[t]{0.48\textwidth}
      \centering
      \includegraphics[width=\textwidth]{images/intro-fac-unfiltered.png}
      \caption{Program trace with no filter.}
      \label{fig:intro-fac-unfiltered}
    \end{subfigure}
    \quad
    \begin{subfigure}[t]{0.48\textwidth}
      \centering
      \includegraphics[width=\textwidth]{images/intro-fac-filtered.png}
      \caption{Program trace with filters that can stop at function application.}
      \label{fig:intro-fac-filtered}
    \end{subfigure}
  \end{center}
  \caption{Program traces with filters/without filters.}
  \label{fig:intro-fac}
\end{figure}

% \TODO{want to be able to step programs with holes, including error holes:
% abstract foldright vs foldleft}

We have made the following contributes in the paper:
\begin{itemize}
  \item We have developed the hazel stepper and a companion scoped filter system
    that allows the user to specify sub-expressions whose full evaluation they
    are not interested in, sub-expressions whose full evaluation they are
    interested in, specific steps they are not interested in, and specific steps
    they are interested in.
  \item We have deployed the hazel stepper and the scoped filter system in
    lecture and to students. We have conducted qualitative and quantitative
    evaluation of the system.
  \item We have formalized the filter semantics and proved the meta-theory
    theorems of the hazel filter system so that we can be confident that the
    filter system have desired properties.
\end{itemize}

In the following sections, we will first show some examples of the
Hazel stepper, then we will discuss the details of the semantics of
the calculus used by the stepper called \emph{filtered stepper
  calculus}. After such a theoretic part, we will briefly discussion
how the stepper is actually implemented. Finally, we will present our
evaluation results of our stepper by observing its usage statistics in
assignments.

% - developed the hazel stepper and filter system, implemented it,
% deployed it in lecture (qualitative evaluation), deployed it to
% students (qualitative + quantitative evaluation), formalized the
% filter semantics, metatheory

%%% Local Variables:
%%% mode: LaTeX
%%% TeX-master: "main"
%%% End:
