% !TEX root = ./main.tex

\subsection{Mini Stepper}

We use a substitution-based small-step evaluator for the simply typed lambda
calculus with a fixpoint. The evaluator is implemented in OCaml and is
used to test the correctness of the behavior specified by the
semantic rules in the paper.

\subsection{Hazel Implementation}

Hazel internally applies a big-step, environment-based dynamic
semantics: it helps avoid unnecessary substitutions and can be
optimized using tail-call optimization. However, since Hazel
is used as an instructional tool for a functional programming
course, it would be helpful if it could simulate substitution-based
semantics when students are exploring this concept.

To unify the internal implementation of the stepper, we keep one copy
of the dynamic semantics rules and have the big-step evaluator and
small-step evaluator interpret the rules. To simulate substitution
behavior, we post-process the output of the stepper, so that all
variables are displayed as if they are substituted, even though the
evaluator/stepper has not reached that part.

The small-stepper with the filter calculus takes much more
computational resources than the big-step evaluator. Therefore, we
applied many techniques to improve performance:

\begin{enumerate}
\item During filter matching, we always compare the physical
  equality of two OCaml objects.
\item We convert an expression to its locally nameless representation
  to compare the alpha-equivalence during matching.
\item We applied some normalization rules to reduce the size of the
  intermediate expression.
\end{enumerate}

% talk about how the big step + small step evaluator abstractions

% environment-based evaluator + post-processing when showing in
% substitution mode

% Optimisation for of the stepper filter: how we remove redundant
% filters (while insure the correctness of the implementation?)

% Equivalency of function:
% \TODO{compare closure & structural equivalence of the body expression}
% Finally we will move to UUID.

%%% Local Variables:
%%% mode: LaTeX
%%% TeX-master: "main"
%%% End:
