\subsection{Mini Stepper}

A substitution based small-stepper evaluator of simply-typed lambda
calculus with fix point. The evaluator is implemented in OCaml, and is
used to test the the correctness of the behaviour as specify as the
semantic rules in the paper.

\subsection{Hazel Implementation}

Hazel internally applies an big-step, environment-based dynamic
semantics: it helps to avoid unnecesarry substitutions, and can be
generally optimized using tail-call optimization. However, since Hazel
is being used as an instruction tool for a functional programming
course, it would be helpful if it can simulate the substitution-based
semantics when students are explorering this concept.

To unify the internal implementation of the stepper, we keep one copy
of the dynamic semantic rules, and having the big-step evaluator and
small-step evaluator interpret the DSL. To simulate the substitution
behavior, we postpocess the output of the stepper, so that all
variable are displayed as it they are substituted, even though the
evaluator/stepper haven't got to that part.

Small-stepper with filter-calculus takes much more computational
resources than the big-step evaluator. Therefore, we applied many
techniques to improve the performance:

\begin{enumerate}
\item During filter-match, we always perform a comparsion for the
  physical equality of two OCaml object.
\item We convert an expression to its locally nameless representation
  to compare the alpha-equivalence during matching.
\item We applied some normalization rules to reduce the size of the
  intermediate expression.
\end{enumerate}

% talk about how the big step + small step evaluator abstractions

% environment-based evaluator + post-processing when showing in
% substitution mode

% Optimisation for of the stepper filter: how we remove redundant
% filters (while insure the correctness of the implementation?)

% Equivalency of function:
% \TODO{compare closure & structural equivalence of the body expression}
% Finally we will move to UUID.
