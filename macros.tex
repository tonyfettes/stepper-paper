% !TEX root = ./main.tex

%%% Editing Convience Macros

\newif\ifdraft
\drafttrue
\newcommand{\TODO}[1]{\ifdraft{\textcolor{red}{\textbf{TODO}: {#1}}}\fi}
\newcommand{\mck}[1]{\ifdraft{\color{purple}[#1 --Matt]}\fi}
\newcommand{\hxf}[1]{\ifdraft{\color{blue}[#1 --Haoxiang]}\fi}

%%% Hazel language highlight

\lstdefinelanguage{hazel}{
  columns=fullflexible,
  keepspaces=true,
  basicstyle=\ttfamily\color{black},
  keywords={
    let, in, case, end, type,
    eval, pause, hide, debug
  },
  keywordstyle=\bfseries\color{black},
  commentstyle=\color{gray},
}

%%% Stepper/Filter Semantics

\DeclareMathOperator{\DefPat}{Pat}
\DeclareMathOperator{\DefExp}{Exp}
\DeclareMathOperator{\DefVal}{Val}
\DeclareMathOperator{\DefCtx}{Ctx}
\DeclareMathOperator{\DefAct}{Act}
\DeclareMathOperator{\DefGas}{Gas}
\DeclareMathOperator{\DefFilter}{Filter}
\DeclareMathOperator{\KeywordFilter}{filter}
\DeclareMathOperator{\KeywordResidue}{residue}
% \DeclareMathOperator{\Nat}{\mathsf{Nat}}

\newcommand{\Nat}[1]{\underline{#1}}
\newcommand{\Lam}[3]{\lambda #1 : #2 . #3}
\newcommand{\Filter}[4]{\KeywordFilter #1 ~\mathsf{do}~ #2 ~\mathsf{for}~ #3 ~\mathsf{in}~ #4}
\newcommand{\Residue}[4]{\mathsf{do}~ #1 ~\mathsf{for}~ #2 ~\mathsf{at}~ #3 ~\mathsf{in}~ #4}

\newcommand{\FValue}[1]{#1~\mathsf{value}}
\newcommand{\FEStep}[2]{#1 \mapsto #2}
\newcommand{\FSubst}[3]{[#1/#2] #3}

\newcommand{\FCMark}{\circ}

\newcommand{\FASkip}{\mathsf{skip}}
\newcommand{\FAStep}{\mathsf{step}}
\newcommand{\FGOne}{\mathsf{one}}
\newcommand{\FGAll}{\mathsf{all}}

\newcommand{\FDecompose}[5]{#1 \vdash #2 = #3 \{ #4 \}_{#5}}
\newcommand{\FCompose}[3]{#1 = #2 \{ #3 \}}

\newcommand{\FPatMatchesExpOp}{\mathrel{\mathop{\triangleright}}}
\newcommand{\FPatMatchesExp}[2]{#1 \FPatMatchesExpOp #2}
\newcommand{\FPatNotMatchesExpOp}{\mathrel{\mathop{\not\triangleright}}}
\newcommand{\FPatNotMatchesExp}[2]{#1 \FPatNotMatchesExpOp #2}

\newcommand{\FPatMatchesCtxOpL}{\mathrel{\mathop{\triangleright_\circ}}}
\newcommand{\FPatMatchesCtxOpR}{\mathrel{\mathop{\triangleleft_\circ}}}
\newcommand{\FPatMatchesCtx}[3]{#1 \FPatMatchesCtxOpL #2 \{ #3 \}}

\newcommand{\FMatchesOp}{}
\newcommand{\FMatches}[2]{#1 \FMatchesOp #2}

\newcommand{\FInstruct}[3]{#1 \vdash #2 \rightsquigarrow #3}

\newcommand{\FTrans}[2]{#1 \rightarrow #2}

\newcommand{\FStep}[3]{#1 \vdash #2 \mapsto #3}

\newcommand{\FHasType}[3]{#1 \vdash #2 : #3}
\newcommand{\FInTypeContext}[3]{#1 : #2 \in #3}
\newcommand{\FTNat}{\mathbb{N}}

%%% Local Variables:
%%% mode: latex
%%% TeX-master: "main"
%%% End:
