% !TEX root = ./main.tex

\subsection{Lecture Integration}

\subsection{Assignment Integration}

The assignments we evaluated our stepper on usually consist of several
tasks, each tasks ask students to implement or complete a function to
pass given unit tests. To help students better focus on their task, we
have provided some prelude Hazel definitions that students can
directly use in their assignments. We do not want the students to be
able to inspect the evaluation of the prelude program, as this would
break the imaginary narrative that all these prelude content already
exists in the environment. Therefore, we insert a
\lstinline[language=hazel]{debug hide($e)} at the very top of the
program, and a \lstinline[language=hazel]{debug stop($e)} before the
user program. Whenever the user turn the stepper on, it will skip the
execution of the prelude section and stop at the entrance of the user
program.

\subsubsection{Qualitative Feedback}

\subsubsection{Quantitative Feedback}

We use the stepper calculus during the lecturing of the course EECS
490 at University of Michigan. To better understanding how the stepper
calculus helps students to understand the behavior of their program,
we collected anonymous data from students during one of their
assignments. The number of steps used by studens is shown in the
Figure~\ref{fig:eval-num-steps}.

\begin{figure}[ht]
  \centering
  \begin{minipage}{.40\linewidth}
    \begin{subfigure}{\linewidth}
      \includegraphics[width=\textwidth]{images/data-steps-2024-w24-a1.png}
      \caption{Winter 2024, sample size 33}
      \label{fig:eval-num-steps-w24}
    \end{subfigure}
  \end{minipage}
  \quad
  \begin{minipage}{.40\linewidth}
    \begin{subfigure}{\linewidth}
      \includegraphics[width=\textwidth]{images/data-steps-2024-f24-a1.png}
      \caption{Fall 2024, sample size 20}
      \label{fig:eval-num-steps-f24}
    \end{subfigure}
  \end{minipage}
  \caption{Number of steps used by EECS 490 students}
  \label{fig:eval-num-steps}
\end{figure}

As we can see, the number of steps used by students roughly conforms
to a exponential distribution. This implies a small portion of
students use the stepper extensively, while most of the students just
briefly tried it or not using it at all.

%%% Local Variables:
%%% mode: LaTeX
%%% TeX-master: "main"
%%% End:
