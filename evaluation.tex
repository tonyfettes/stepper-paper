% !TEX root = ./main.tex

\subsection{Lecture Integration}

\subsection{Assignment Integration}

\subsubsection{Qualitative Feedback}

\subsubsection{Quantitative Feedback}

We use the stepper calculus during the lecturing of the course EECS
490 at University of Michigan. To better understanding how the stepper
calculus helps students to understand the behavior of their program,
we collected anonymous data from students during one of their
assignments. The number of steps used by studens is shown in the
Figure~\ref{fig:eval-num-steps}.

\begin{figure}[h]
  \centering
  \begin{minipage}{.40\linewidth}
    \begin{subfigure}{\linewidth}
      \includegraphics[width=\textwidth]{images/data-steps-2024-w24-a1.png}
      \caption{Winter 2024, sample size 33}
      \label{fig:eval-num-steps-w24}
    \end{subfigure}
  \end{minipage}
  \quad
  \begin{minipage}{.40\linewidth}
    \begin{subfigure}{\linewidth}
      \includegraphics[width=\textwidth]{images/data-steps-2024-f24-a1.png}
      \caption{Fall 2024, sample size 20}
      \label{fig:eval-num-steps-f24}
    \end{subfigure}
  \end{minipage}
  \caption{Number of steps used by EECS 490 students}
  \label{fig:eval-num-steps}
\end{figure}

As we can see, the number of steps used by students roughly conforms
to a exponential distribution. This implies a small portion of
students use the stepper extensively, while most of the students just
briefly tried it or not using it at all.

%%% Local Variables:
%%% mode: LaTeX
%%% TeX-master: "main"
%%% End:
