\section{Filtered Stepper Calculus}

In this section, we will discuss the semantics of the \emph{Filtered Stepper
Calculus}. We first introduce the syntax of the language, then we discuss the
the dynamics of the language. Finally, we will discuss the optional part of the
semantic, the statics of the language, and show that the calculus satisfies
properties like progression and preservation.

\subsection{Syntax}

\begin{figure}[h]
  \begin{equation*}
    \begin{array}{rcl}
      \DefAct a    &\Coloneqq& \ActSkip \mid \ActStep \\
      \DefGas g    &\Coloneqq& \GasOne \mid \GasAll \\
      \DefFilter f &\Coloneqq& (p, a, g) \\
      \DefExp e    &\Coloneqq& x \mid e(e) \mid \Lam{x}{e} \mid e + e \mid \Nat{n} \\
                   &\mid     & \Filter{f}{e} \\
                   &\mid     & \Residue{a}{g}{l}{e} \\
      \DefPat p    &\Coloneqq& x \mid p(p) \mid \Lam{x}{p} \mid p + p \mid \Nat{n} \\
                   &\mid     & \PatExpr \\
                   &\mid     & \PatValue \\
    \end{array}
  \end{equation*}
  \caption{Syntax of Filtered Stepper Calculus. Here, $n$ ranges over numbers, $x$ over variables, and $l$ over priorities, which are natural numbers.}
  \label{fig:filter-syntax}
\end{figure}

The syntax of filter stepper calculus (Fig.~\ref{fig:filter-syntax}) is a
extended version of untyped lambda calculus. Additionally, natural numbers and
addition are added to it for some more concrete demonstration. It introduce two
new expression to a untyped lambda calculus language, \texttt{Filter} and
\texttt{Residue}. The \texttt{Filter} expression is visible at user level,
indicating the expression surrounded by the expression is filtered by the
filter. The \texttt{Residue} expression is used internally by the stepper and is
not visible to the user. It is used to signal the \emph{residue} of a effect of
applying a filter to an expression.

A filter is a tuple of a pattern, an action, and a gas. The pattern
is a mutated of the term of untyped lambda calculus. Apart from constructs like
variables, applications, and lambdas, it also includes some new constructs, a
forall pattern \(\$e\), a value pattern \(\$v\). The action is a tag indicating
the action to be taken when the pattern matches an expression. It has two
form, \(\ActSkip\) and \(\ActStep\). The gas is another tag indicating the
live time of the effect of a filter. It has two form, \(\GasOne\) and
\(\GasAll\).

The residue expression \(\Residue{a}{g}{l}{e}\) is a expression that is used to
store the residue of the effect of a filter. It has four components, an action
\(a\), a gas \(g\), a priority \(l\), and a expression \(e\). The action and the
gas are the same as the ones in a filter. The priority is a natural number
indicating the priority of the residue.

\subsection{Contextual Dynamics}

The contextual dynamics of the filtered stepper calculus builds around the stepping judgment. This judgment establish the contextual behavior of the stepper.

There are two main operations in our semantics, one is to \emph{skip} the
evaluation of an expression, the other is to \emph{pause} at a certain point of
evaluation. These two operations leads to two main stepping judgment in our
dynamic specifications. The \textsc{S-Step} rule specifies when do we want to
stop at a step, while the \textsc{S-Skip} rule specifies when we want to skip over
current step and proceed to next ones.

There is an extra rule in the stepping judgment. The elimination step of filter
should be always skipped. As pattern-matching against the filter statement is
not possible within current pattern definition as in
Figure.~\ref{fig:filter-syntax}, we treat them specially to make sure these
steps are hidden and is not visible to users.

We will first see two examples going through how should we or user interpreter these two rules, the gives the formal definitions of the judgment and all preceding judgment, including judgment for instrumentation, decomposition/re-composition, instruction-transition, and most importantly, stepping.

% \TODO{motivate the two rules}

\subsubsection{Examples and Motivations}

Consider the example below:
\lstset{xleftmargin=1in}
\begin{lstlisting}[language=hazel]
eval 1 + 2 + 3 + 4 in
pause 3 + 3 + 4 in
1 + 2 + 3 + 4
\end{lstlisting}

It turns into the following in the calculus:
\begin{lstlisting}[language=hazel]
filter (1 + 2 + 3 + 4, skip, all) in
filter (3 + 3 + 4, step, one) in
1 + 2 + 3 + 4
\end{lstlisting}

\begin{enumerate}
  \item Instrumentation: At this step, we transverse the expression recursively and collect filters we have seen. Then for each sub-expression, we test if the sub-expression matches any of the filters we have seen. If it does, we insert a residue expression with the filter information around the matched sub-expression.

  If we start with an empty set, we might end up with a empty matching result and the stepper will have no clue what kind of action to take. Therefore, we set the default behavior to be stepping through the evaluation of every expression.

  Here, we apply two filters to the expression \(1 + 2 + 3 + 4\), one is \(\Filter{1 + 2 + 3 + 4}{\ActSkip, \GasAll}\), the other is \(\Filter{3 + 3 + 4}{\ActStep, \GasOne}\). During the application of the first filter, we find out that the sub-expression \(1 + 2 + 3 + 4\) matches the pattern \(1 + 2 + 3 + 4\), so we insert a residue expression \(\Residue{\ActSkip}{\GasAll}{0}{\ldots}\) around the sub-expression (I-Add-Y). We then get an instrumented expression as:
  \begin{lstlisting}[language=hazel]
    filter ($e, pause, one) in
    filter (1 + 2 + 3 + 4, skip, all) in
    filter (3 + 3 + 4, step, one) in
    (pause, 0)<
      (skip, 1)<
        (pause, 0)<(pause, 0)<1 + 2>^one + 3>^one + 4
      >^all
    >^one
  \end{lstlisting}

  However, during the application of the second filter, there is no sub-expression that matches the pattern \(3 + 3 + 4\), so no residue expression is inserted around the sub-expression (I-Add-N). The instrumented expression remains the same.

  \item Decomposition: The expression is decomposed into an evaluation context and selected expression. The instrumented expression from last step can be decomposed by recursively decomposing addition (D-Add-L) and residue expression (D-Residue). The decomposed evaluation context looks like this:
    \begin{lstlisting}[language=hazel]
      filter (1 + 2 + 3 + 4, skip, all) in
      filter (3 + 3 + 4, step, one) in
      (pause, 0)<
        (skip, 1)<
          (pause, 0)<(pause, 0)<(*@\(\circ\)@*)>^one + 3>^one + 4
        >^all
      >^one
    \end{lstlisting}
    and a selected expression \lstinline{1 + 2}

  \item Action Selection: Select an action from the evaluation context. The action that is closer to the selected expression, and with the highest priority is selected. In this case, we transverse the expression using (A-Add-L) and update the selected action when we retreat.The action \(\ActSkip\) is selected, and the expression itself is kept the same.

  \item Residue removal: The residue expression that has life-time of one is removed from the expression. This step is done using the \(\Decay\) function. The expression after the removal of the residue expression is:
    \begin{lstlisting}[language=hazel]
      filter (1 + 2 + 3 + 4, skip, all) in
      filter (3 + 3 + 4, step, one) in
      (skip, 1)<((*@\(\circ\)@*) + 3) + 4>^all
    \end{lstlisting}
  \item Instruction Transition: The selected expression is instruction-transitioned to a new expression. In this case, the selected expression is \(1 + 2\), which is instruction-transitioned to \(3\) according to the transition rule (T-Add).

  \item Re-composition: When the selected-expression is instruction-transitioned to a new expression, we need to recompose the transitioned selected expression with evaluation context to get the updated expression. The rules for re-composition is the same as the rules for decomposition, but instead, we start from rule (D-Add-L) to build up the recomposed expression in a bottom-up way. The recomposed expression is:
    \begin{lstlisting}[language=hazel]
      filter (1 + 2 + 3 + 4, skip, all) in
      filter (3 + 3 + 4, step, one) in
      (skip, 1)<3 + 3 + 4>^all
    \end{lstlisting}

  \item If the selected expression (before instruction-transition) is a filter expression, i.e. it is either of form \(\Filter{f}{e}\) or \(\Residue{a}{g}{l}{e}\), then the stepper will not display the step to the user, and directly go to the next step (S-Filter). This helps to hide the filter expressions from the user.

  \item Otherwise, if the action is \(\ActStep\), then the stepper pause at this step exhibiting the evaluating context and the selected expression to user. (S-Step)

  \item If the action is \(\ActSkip\), then the stepper will skip the selected expression and continue stepping. (S-Skip) In this case, the selected expression is \(1 + 2\), and the selected action is \(\ActSkip\), so the stepper will skip this step, which is calculating the value of \(1 + 2\) to be \(3\).

  \item The final result of the stepping is:
    \begin{lstlisting}[language=hazel]
      filter (1 + 2 + 3 + 4, skip, all) in
      filter (3 + 3 + 4, step, one) in
      (skip, 1)<3 + 3 + 4>^all
    \end{lstlisting}
\end{enumerate}

In the example above, we have briefly explored how instrumentation, action selection,  decomposition, and transition works, and how the stepper is able to decide what to do next. However, it does not explain why we need a priorities to get the right selected action. Here is an example demonstrating why we need priorities in the residue expressions:
\begin{lstlisting}[language=hazel]
pause 1 + 2 in
eval 1 + 2 + 3 in
1 + 2 + 3
\end{lstlisting}

Most of the steps are similar to previous example, so we keep them simple here.
\begin{enumerate}
  \item\label{agl:priority-example-instr} Instrumentation: The instrumentation process is not affected by the priorities. After instrumentation, the expression looks like:
  \begin{lstlisting}[language=hazel]
    pause 1 + 2 in
    eval 1 + 2 + 3 in
    (eval, 1)<(pause, 0)<1 + 2>^one + 3>^all
  \end{lstlisting}

  \item Decomposition: This step is also not affected by the priorities. The expression is decomposed into an evaluation context and a selected expression. The selected expression is \(1 + 2\). The evaluation context is:
  \begin{lstlisting}[language=hazel]
    pause 1 + 2 in
    eval 1 + 2 + 3 in
    (eval, 1)<(pause, 0)<(*@\(\circ\)@*)>^one + 3>^all
  \end{lstlisting}

  \item Action selection: The action selection is where the priorities come into play. The action that is closer to the selected expression, and with the highest priority is selected. In this case, the selected action is \(\ActSkip\), and the selected expression is \(1 + 2 + 3\). If we don't have priorities, we will end up selecting \(1 + 2\) with the selected action being \(\ActStep\). This is rather counter-intuitive, as we want the effect of a later filter, like \(Filter{1 + 2 + 3}{\ActSkip, \GasAll}\), to take precedence over the effect of an earlier filter, like \(\Filter{1 + 2}{\ActStep, \GasOne}\).

  \item Residue removal: The residue expression that has life-time of one is removed from the expression using function \(\Decay\). The expression after the removal of the residue expression is:
  \begin{lstlisting}[language=hazel]
    pause 1 + 2 in
    eval 1 + 2 + 3 in
    (skip, 1)<(*@\(\circ\)@*) + 3>^all
  \end{lstlisting}

  \item Instruction transition: The selected expression is instruction-transitioned to a new expression. In this case, the selected expression is \(1 + 2\), which is instruction-transitioned to \(3\).

  \item Re-composition: The instruction-transitioned expression \(3\) is recomposed with the evaluation context to get the updated expression:
  \begin{lstlisting}[language=hazel]
    pause 1 + 2 in
    eval 1 + 2 + 3 in
    (skip, 1)<3 + 3>^all
  \end{lstlisting}

  \item Because we have selected the \(\ActSkip\) action, the stepper will not stop here but instead goes to the step~\ref{agl:priority-example-instr} and continue stepping.

  \item All the following stepping steps ends up being skipped, so the final result of the stepping is the value of the expression:
  \begin{lstlisting}[language=hazel]
    6
  \end{lstlisting}
\end{enumerate}

With the two motivating examples, the rules for the dynamic semantics of the filter calculus is presented as following.

% \TODO{Put decay after instruction-transition.}

% \fbox{\(\RuleStep{e}{e'}\)} Expression \(e\) steps to expression \(e'\).
% \begin{mathpar}
%   \inferrule[S-Step]{
%     \Instruct{\PatExpr}{\ActStep}{\GasOne}{0}{e}{e_i} \\
%     \Decompose{e_i}{\mathcal{E}_0}{e_0} \\
%     \Analyze{(\ActStep, 0)}{\mathcal{E}_0}{\ActStep} \\
%     \FTrans{e_0}{e_t} \\
%     \mathcal{E}_1 = \Decay{\mathcal{E}_0} \\
%     \Compose{e_1}{\mathcal{E}_1}{e_t}
%   }{
%     \RuleStep{e}{e_1}
%   }
% \end{mathpar}

% \mck{Does this work with 0-step evaluations e.g. skip \$e in 1 + 2 + 3}

% \subsubsection{Solution 1: use separate judgments for skipping, stepping, and value (done)}

% eval = skip all
% hide = skip one
% debug = step all
% pause = step one

% e = 
% skip all 1 + 2 + 3 in 
% step one 3 + 3 in 
% 1 + 2 + 3 + 4 * 5 * 6

% ei = 
% skip all 1 + 2 + 3 in 
% step one 3 + 3 in 
% <1 + 2 + 3> + 4 * 5 * 6

% e0 = 1 + 2

% et = 3

% e1 = 
% skip all 1 + 2 + 3 in 
% step one 3 + 3 in 
% <3 + 3> + 4 * 5 * 6

% |->

% e1' = 
% skip all 1 + 2 + 3 in 
% step one 3 + 3 in 
% <6> + 4 * 5 * 6


% 0. zero click judgement
% 1. 1 click judgement
%   a. produce contexts + whether they are skipped
%   b. select one context and take a user step
%   c. run the zero click judgement

% 1. step judgement
% 2. skip judgement
% 3. value judgement

% \fbox{\(\RuleSkip{e}{e'}\)}
% \begin{mathpar}
%   \inferrule[S-Skip]{
%     \Instruct{\PatExpr}{\ActStep}{\GasOne}{0}{e}{e_i} \\
%     \Decompose{e_i}{\mathcal{E}_0}{e_0} \\
%     \Analyze{(\ActStep, 0)}{\mathcal{E}_0}{\ActSkip} \\
%     \FTrans{e_0}{e_t} \\
%     \mathcal{E}_1 = \Decay{\mathcal{E}_0} \\
%     \Compose{e_1}{\mathcal{E}_1}{e_t}
%   }{
%     \RuleSkip{e}{e_1'}
%   } \\
%   \inferrule[S-Filter]{
%     \Instruct{\PatExpr}{\ActStep}{\GasOne}{0}{e}{e_i} \\
%     \Decompose{e_i}{\mathcal{E}_0}{\Filter{f}{e_0}} \\
%     \Analyze{(\ActStep, 0)}{\mathcal{E}_0}{a} \\
%     \FTrans{e_0}{e_t} \\
%     \mathcal{E}_1 = \Decay{\mathcal{E}_0} \\
%     \Compose{e_1}{\mathcal{E}_1}{e_t} \\
%     \RuleSkip{e_1}{e_1'}
%   }{
%     \RuleSkip{e}{e_1'}
%   }
% \end{mathpar}

% \fbox{\(\RulePaused{e}{e'}\)} Evaluation of expression \(e\) is skipped-then-paused at \(e'\).
% \begin{mathpar}
%   \inferrule[S-Paused-Init]{
%     \RuleStep{e}{e'}
%   }{
%     \RulePaused{e}{e'}
%   } \qquad
%   \inferrule[S-Paused-Recur]{
%     \RuleSkip{e}{e''} \\
%     \RulePaused{e''}{e'}
%   }{
%     \RulePaused{e}{e''}
%   }
% \end{mathpar}

% \fbox{\(\RuleValue{e}{v}\)} Expression \(e\) is skipped to a value \(v\).
% \begin{mathpar}
%   \inferrule[S-Value-Init]{
%     \Value{v}
%   }{
%     \RuleValue{v}
%   } \qquad
%   \inferrule[S-Value-Recur]{
%     \RuleSkip{e}{e'}\\
%     \RuleValue{e'}{v}
%   }{
%     \RuleValue{e}{v}
%   }
% \end{mathpar}

% \begin{theorem}
% For an expression \(e\) that has type \(\tau\) under typing conting context \(\Gamma\), (or equivalently has no free variable in a untyped system), after skipping some stext \(\Gamma\), (or equivalently has no free variable in a untyped system), after skipping some steps, either it is an value, or it can be transitioned into another expression.
% \end{theorem}

% \subsubsection{Solution 2: count the number of steps}

\fbox{\(\RuleStepN{e}{e'}{n}\)} Expression \(e\) steps to \(e'\) in \(n\) steps.
\begin{mathpar}
  \inferrule[SN-Step]{
    \Instruct{\PatExpr}{\ActStep}{\GasOne}{0}{e}{e_i} \\
    \Decompose{e_i}{\mathcal{E}_0}{e_0} \\
    \Analyze{(\ActStep, 0)}{\mathcal{E}_0}{\ActStep} \\
    \FTrans{e_0}{e_t} \\
    \mathcal{E}_1 = \Decay{\mathcal{E}_0} \\
    \Compose{e_1}{\mathcal{E}_1}{e_t}
  }{
    \RuleStepN{e}{e_1}{1}
  } \\
  \inferrule[SN-Skip]{
    \Instruct{\PatExpr}{\ActStep}{\GasOne}{0}{e}{e_i} \\
    \Decompose{e_i}{\mathcal{E}_0}{e_0} \\
    \Analyze{(\ActStep, 0)}{\mathcal{E}_0}{\ActSkip} \lor e_0 = \Filter{f}{e_0'}\\
    \FTrans{e_0}{e_t} \\
    \mathcal{E}_1 = \Decay{\mathcal{E}_0} \\
    \Compose{e_1}{\mathcal{E}_1}{e_t} \\    
    \RuleStepN{e_1}{e_2}{n}
  }{
    \RuleStepN{e}{e_2}{n}
  } \\
  \inferrule[SN-Value]{
    \Value{v}
  }{
    \RuleStepN{v}{v}{0}
  }
\end{mathpar}

\begin{theorem}
For an expression \(e\) that has type \(\tau\) under typing context \(\Gamma\), (or equivalently has no free variable in a un-typed system), it can either take an \emph{1 step transition} (SN-Step, SN-Skip) into another expression, or a ``\(0\) steps transition'' into a value (SN-Skip, SN-Value).
\end{theorem}

% \hxf{After some thought, I think the reason that we can see the result of a 0-step evaluations because we are not reading the result \emph{after} the step. Instead, we are reading the result of decomposition.}

\fbox{\(\Instruct{p}{a}{g}{l}{e}{e'}\)} \(e\) is dynamically instructed to \(e'\).
\begin{mathpar}
  \inferrule[I-V]{
    \Value{v}
  }{
    \Instruct{p}{a}{g}{l}{v}{v}
  } \\
  \inferrule[I-Var-Y]{
    \FPatMatchesExp{p}{x}
  }{
    \Instruct{p}{a}{g}{l}{x}{\Residue{a}{g}{l}{x}}
  } \qquad
  \inferrule[I-Var-N]{
    \FPatNotMatchesExp{p}{x}
  }{
    \Instruct{p}{a}{g}{l}{x}{x}
  } \\
  \inferrule[I-Filter]{
    \FInstruct{(p_0, a_0, g_0, l_0)}{e_0}{e} \\
    \FInstruct{(p, a, g, l_0 + 1)}{e}{e'}
  }{
    \FInstruct{(p_0, a_0, g_0, l_0)}{\Filter{(p, a, g)}{e_0}}{\Filter{(p, a, g)}{e'}}
  } \\
  \inferrule[I-Residue]{
    \FInstruct{(p_0, a_0, g_0, l_0)}{e_0}{e} \\
  }{
    \FInstruct{(p_0, a_0, g_0, l_0)}{\Residue{a}{g}{l}{e_0}}{\Residue{a}{g}{l}{e}}
  } \\
  \inferrule[I-Ap-Y]{
    \Instruct{p}{a}{g}{l}{e_1}{e_1'} \\
    \Instruct{p}{a}{g}{l}{e_2}{e_2'} \\
    \FPatMatchesExp{p}{e_1(e_2)}
  }{
    \Instruct{p}{a}{g}{l}{e_1(e_2)}{\Residue{a}{g}{l}{e_1'(e_2')}}
  } \\
  \inferrule[I-Ap-N]{
    \Instruct{p}{a}{g}{l}{e_1}{e_1'} \\
    \Instruct{p}{a}{g}{l}{e_2}{e_2'} \\
    \FPatNotMatchesExp{p}{e_1(e_2)}
  }{
    \Instruct{p}{a}{g}{l}{e_1(e_2)}{e_1'(e_2')}
  } \\
  \inferrule[I-Add-Y]{
    \Instruct{p}{a}{g}{l}{e_1}{e_1'} \\
    \Instruct{p}{a}{g}{l}{e_2}{e_2'} \\
    \FPatMatchesExp{p}{e_1 + e_2}
  }{
    \Instruct{p}{a}{g}{l}{e_1(e_2)}{\Residue{a}{g}{l}{e_1' + e_2'}}
  } \\
  \inferrule[I-Add-N]{
    \Instruct{p}{a}{g}{l}{e_1}{e_1'} \\
    \Instruct{p}{a}{g}{l}{e_2}{e_2'} \\
    \FPatNotMatchesExp{p}{e_1 + e_2}
  }{
    \Instruct{p}{a}{g}{l}{e_1(e_2)}{e_1' + e_2'}
  }
\end{mathpar}

% \mck{I think there's some preservation of priority property to prove here, since the numbers are re-generated on every pass.}

% \hxf{I would formalize the preservation property like this: the priority of residue/do statements \emph{produced} by the same filter have the same priorities}

% \mck{This instrumentation will regenerate the do statements on every pass - I guess it doesn't cause problems but it would be nice if it didn't?}

% \hxf{Yeah it would be much nicer to avoid the duplication of do statements. We can to some ad-hoc fusing of do-statements same effect \& priorties, probably utilmately we want the instrumentation to be Idempotent.}

\fbox{\(\Decay{\mathcal{E}} = \mathcal{E}'\)} Evaluation Context \(\mathcal{E}\) decays to context \(\mathcal{E}'\).
\[
  \begin{aligned}
    \Decay{x} &= x \\
    \Decay{\Lam{x}{e}} &= \Lam{x}{\Decay{e}} \\
    \Decay{e_1(e_2)} &= (\Decay{e_1})(\Decay{e_2}) \\
    \Decay{\Nat{n}} &= \Decay{\Nat{n}} \\
    \Decay{e_1 + e_2} &= (\Decay{e_1}) + (\Decay{e_2}) \\
    \Decay{\Filter{(p, a, g)}{e}} &= \Filter{(p, a, g)}{\Decay{e}} \\
    \Decay{\Residue{a}{\GasOne}{l}{e}} &= \Decay{e} \\
    \Decay{\Residue{a}{\GasAll}{l}{e}} &= \Residue{a}{\GasAll}{l}{\Decay{e}}
  \end{aligned}
\]

% \TODO{Correct the function to use correct evaluation context}

\fbox{\(\Analyze{(a, l)}{\mathcal{E}}{a'}\)} Under the filter environment of \((a, l)\), the context \(\mathcal{E}\) is transitioned to \(\mathcal{E}'\) and the action \(a'\) is returned.
\begin{mathpar}
  \inferrule[A-Var]{
  }{
    \Analyze{(a, l)}{\circ}{a}
  } \\
  \inferrule[A-Ap-L]{
    \Analyze{(a, l)}{\mathcal{E}_1}{a'}
  }{
    \Analyze{(a, l)}{\mathcal{E}_1(e)}{a'}
  } \qquad
  \inferrule[A-Ap-R]{
    \Analyze{(a, l)}{\mathcal{E}_2}{a'}
  }{
    \Analyze{(a, l)}{e_1(\mathcal{E}_2)}{a'}
  } \\
  \inferrule[A-Add-L]{
    \Analyze{(a, l)}{\mathcal{E}_1}{a'}
  }{
    \Analyze{(a, l)}{\mathcal{E}_1 + e}{a'}
  } \qquad
  \inferrule[A-Add-R]{
    \Analyze{(a, l)}{\mathcal{E}_2}{a'}
  }{
    \Analyze{(a, l)}{e_1 + \mathcal{E}_2}{a'}
  } \\
  \inferrule[A-Filter]{
    \Analyze{(a, l)}{\mathcal{E}}{a'}
  }{
    \Analyze{(a, l)}{\Filter{f}{\mathcal{E}}}{a'}
  } \\
  \inferrule[A-Residue-One-Old]{
    l \le l_0 \\
    \Analyze{(a_0, l_0)}{\mathcal{E}}{a'}
  }{
    \Analyze{(a_0, l_0)}{\Residue{a}{\GasOne}{l}{\mathcal{E}}}{a'}
  } \qquad
  \inferrule[A-Residue-One-New]{
    l > l_0 \\
    \Analyze{(a, l)}{\mathcal{E}}{a'}
  }{
    \Analyze{(a_0, l_0)}{\Residue{a}{\GasOne}{l}{\mathcal{E}}}{a'}
  } \\
  \inferrule[A-Residue-All-Old]{
    l \le l_0 \\
    \Analyze{(a_0, l_0)}{\mathcal{E}}{a'}
  }{
    \Analyze{(a_0, l_0)}{\Residue{a}{\GasAll}{l}{\mathcal{E}}}{a'}
  } \qquad
  \inferrule[A-Residue-All-New]{
    l > l_0 \\
    \Analyze{(a, l)}{\mathcal{E}}{a'}
  }{
    \Analyze{(a_0, l_0)}{\Residue{a}{\GasAll}{l}{\mathcal{E}}}{a'}
  }
\end{mathpar}

\fbox{\(\DefCtx \mathcal{E}\)} Context \(\mathcal{E}\).
\[
  \DefCtx \mathcal{E}
  = \FCMark
  \mid \mathcal{E}(d)
  \mid d(\mathcal{E})
  \mid \mathcal{E} + d
  \mid d + \mathcal{E}
  \mid \Filter{f}{\mathcal{E}}
  \mid \Residue{a}{g}{l}{\mathcal{E}}
\]

\fbox{\(\Value e\)} Expression \(e\) is a value.
\begin{mathpar}
  \inferrule[V-Lam]{
  }{
    \Value{\Lam{x}{d}}
  } \qquad
  \inferrule[V-Nat]{
  }{
    \Value{\Nat{n}}
  }
\end{mathpar}

\fbox{\([v / x] e = e'\)} \(e'\) can be obtained by substitution of \(x\) for
\(v\) in expression \(e\).
\[
  \begin{aligned}
    [v / x] \Nat{n} &= \Nat{n} \\
    [v / x] x &= v \\
    [v / x] y &= y && \text{if } x \neq y \\
    [v / x] (e_1(e_2)) &= ([v / x] e_1)([v / x] e_2) \\
    [v / x] (e_1 + e_2) &= ([v / x] e_1) + ([v / x] e_2) \\
    [v / x] \Lam{y}{e} &= \Lam{y}{[v / x] e} && \text{if } x \neq y \\
    [v / x] \Lam{y}{e} &= \Lam{y}{e} && \text{if } x = y \\
    [v / x] \Filter{(p, a, g)}{e} &= \Filter{([v / x]p, a, g)}{[v / x] e} \\
    [v / x] \Residue{a}{g}{l}{e} &= \Residue{a}{g}{l}{[v / x] e} \\
  \end{aligned}
\]

\fbox{\([v / x] p = p'\)} \(p'\) can be obtained by substitution of \(x\) for
\(v\) in pattern \(p\).
\[
  \begin{aligned}
    [v / x] \$e &= \$e \\
    [v / x] \$v &= \$v \\
    [v / x] \Nat{n} &= \Nat{n} \\
    [v / x] y &= y && \text{if } x \neq y \\
    [v / x] (p_1(p_2)) &= ([v / x] p_1)([v / x] p_2) \\
    [v / x] (p_1 + p_2) &= ([v / x] p_1) + ([v / x] p_2) \\
    [v / x] \Lam{y}{e} &= \Lam{y}{[v / x] e} && \text{if } x \neq y \\
    [v / x] \Lam{y}{e} &= \Lam{y}{e} && \text{if } x = y \\
  \end{aligned}
\]

% \hxf{Shall we use different notation for patterns?}

% \TODO{swap order to [v/x]e}

% \TODO{missing filter and do cases}

% \TODO{missing number literal and addition cases}

% \TODO{use e instead of d everywhere}

% \TODO{substitution into patterns}

% \TODO{fix Lam macro}

% \TODO{combine decompose and recompose?}

% \TODO{add function that removes do one to formalism}

\fbox{\(\Matches{p}{e}\)} means that pattern \(p\) matches expression \(e\).
\begin{mathpar}
  \inferrule[M-All]{\ }{
    \Matches{\$e}{e}
  } \qquad
  \inferrule[M-Val]{\Value{v}}{
    \Matches{\$v}{v}
  } \\
  \inferrule[M-Nat]{
    \
  }{
    \Matches{\Nat{m}}{\Nat{m}}
  }\\
  \text{\hxf{I guess we still need to explicitly use equivalence here?}}\\
  \inferrule[M-Lam]{
    \Lam{x}{e_p} \equiv \Lam{y}{e_e}
  }{
    \Matches{\Lam{x}{e_p}}{\Lam{y}{e_e}}
  } \\
  \text{\hxf{Or we can just write as if they're structurally equivalent?}}\\
  \inferrule[M-Lam']{
    \
  }{
    \Matches{\Lam{x}{e_p}}{\Lam{y}{e_p}}
  } \\
  \inferrule[M-Ap]{
    \Matches{p_1}{e_1} \\
    \Matches{p_2}{e_2}
  }{
    \Matches{p_1(p_2)}{e_1(e_2)}
  } \qquad
  \inferrule[M-Add]{
    \Matches{p_1}{e_1} \\
    \Matches{p_2}{e_2}
  }{
    \Matches{p_1 + p_2}{e_1 + e_2}
  } \qquad
\end{mathpar}

\mck{M-Lam doesn't allow \$e inside function bodies - is this intentional?}

\hxf{This is not \emph{strictly} intentional, i.e. they can be changed. But I cannot really think of any case that we need to use \$e in function bodies.}

% \mck{It might be clearer if we write out a separate recompose judgement too. I think decompose and recompose behave differently especially with regard to do statements}

\fbox{\(\Decompose{d}{\mathcal{E}}{d'}\)} Expression \(d\) can be obtained by putting expression \(d'\) into the mark of \(\mathcal{E}\).
\fbox{\(\Decompose{d}{\mathcal{E}}{d'}\)} Expression \(d\) can be obtained by putting expression \(d'\) into the mark of \(\mathcal{E}\).
\begin{mathpar}
  \inferrule[D-Var-E]{
  }{
    \Decompose{x}{\FCMark}{x}
  } \\
  \inferrule[D-Residue-T]{
    \Decompose{e}{\mathcal{E}}{e'}
  }{
    \Decompose{\Residue{a}{g}{l}{e}}{\Residue{a}{g}{l}{{\mathcal{E}}}}{e'}
  } \\
  \inferrule[D-Residue-E]{
    \Value{v}
  }{
    \Decompose{\Residue{a}{g}{l}{v}}{\circ}{\Residue{a}{g}{l}{v}}
  } \\
  \inferrule[D-Filter-T]{
    \Decompose{e}{\mathcal{E}}{e'}
  }{
    \Decompose{\Filter{f}{e}}{\Filter{f}{\mathcal{E}}}{e'}
  } \\
  \inferrule[D-Filter-E]{
    \Value{v}
  }{
    \Decompose{\Filter{f}{v}}{\circ}{\Filter{f}{v}}
  } \\
  \inferrule[D-Ap-L]{
    \Decompose{e_1}{\mathcal{E}_1}{e_1'}
  }{
    \Decompose{e_1(e_2)}{\mathcal{E}_1(e_2)}{e_1'}
  } \qquad
  \inferrule[D-Ap-R]{
    \Value{e_1} \\
    \Decompose{e_2}{\mathcal{E}_2}{e_2'}
  }{
    \Decompose{e_1(e_2)}{e_1(\mathcal{E}_2)}{e_2'}
  } \qquad
  \inferrule[D-Ap-E]{
    \Value{e_1} \\
    \Value{e_2}
  }{
    \Decompose{e_1(e_2)}{\circ}{e_1(e_2)}
  } \\
  \inferrule[D-Add-L]{
    \Decompose{e_1}{\mathcal{E}_1}{e_1'}
  }{
    \Decompose{e_1 + e_2}{\mathcal{E}_1 + e_2}{e_1'}
  } \qquad
  \inferrule[D-Add-R]{
    \Value{e_1} \\
    \Decompose{e_2}{\mathcal{E}_2}{e_2'}
  }{
    \Decompose{e_1 + e_2}{e_1 + \mathcal{E}_2}{e_2'}
  } \qquad
  \inferrule[D-Add-E]{
    \Value{e_1} \\
    \Value{e_2}
  }{
    \Decompose{e_1 + e_2}{\circ}{e_1 + e_2}
  }
\end{mathpar}
\mck{D-Filter-E rule does not have any filters in it}
\hxf{Now they have!}

\mck{The last four rules seem to specify an evaluation order - is this desired behaviour? I thought we were going to mention the nondeterminism somewhere in the text}

\hxf{Exactly.}

\fbox{\(\Strip{e} = {e'}\)} Strip filter/do expression in expression \(e\) to get expression \(e'\).
\[
  \begin{aligned}
    \Strip{x} &= x \\
    \Strip{\Lam{x}{e}} &= \Lam{x}{\Strip{e}} \\
    \Strip{e_1(e_2)} &= (\Strip{e_1})(\Strip{e_2}) \\
    \Strip{\Nat{n}} &= \Strip{\Nat{b}} \\
    \Strip{e_1 + e_2} &= (\Strip{e_1}) + (\Strip{e_2}) \\
    \Strip{\Filter{(p, a, g)}{e}} &= \Strip{e} \\
    \Strip{\Residue{a}{g}{l}{e}} &= \Strip{e}
  \end{aligned}
\]

\fbox{\(\FTrans{e}{e'}\)} \(e\) takes an instruction transition to \(e'\).
\begin{mathpar}
  \inferrule[T-Lam]{
    \Value{e_1} \\
    \Value{e_2}
  }{
    \FTrans{(\Lam{x}{e_1})(e_2)}{\FSubst{e_2}{x}{e_1}}
  } \qquad
  \inferrule[T-Add]{
    \Value{n_1} \\
    \Value{n_2} \\
    n_1 + n_2 = n
  }{
    \FTrans{\Nat{n_1} + \Nat{n_2}}{\Nat{n}}
  } \\
  \inferrule[T-Residue]{
    \Value{v}
  }{
    \FTrans{\Residue{a}{g}{l}{v}}{v}
  } \qquad
  \inferrule[T-Filter]{
    \Value{v}
  }{
    \FTrans{\Filter{f}{v}}{v}
  }
\end{mathpar}

\mck{Is FLRes-T redundant since we're using evaluation contexts?}

\hxf{It depends how you treat instruction transition? They are less important than elimination rules}

\mck{ok yeah it would be nice to have a progress property here but I guess we do need types for that...}

\subsection{Properties}

\TODO{Properties}
\\\\
Conjecture: Idempotence of Instrumentation (Instrumenting twice should have the same effect as instrumenting once)

$\Instruct{p}{a}{g}{l}{e}{e'} \Rightarrow \Instruct{p}{a}{g}{l}{e'}{e''} \Rightarrow e' = e''$

\mck{This conjecture is actually not true, because instrumentation always adds more do statements, even if they're already there.}
\\\\
Conjecture : Determinism of steps

$\RuleStep{e}{e'} \Rightarrow \RuleStep{e}{e''} \Rightarrow e' = e''$

\mck{I think this holds in this calculus, but not in Hazel itself.}
\\\\
Conjecture(Unchanging Priorities)

\(\)

\mck{Since priorities are always re-numbered, we want some sort of conjecture that they remain the same. We could even possibly use this to avoid duplication of `do's?}

\hxf{I believe the Unchanging Priorities does holds. I would phrase it as: the residue statements inserted by the same filter statement has the same priorities (or their priorites keeps in order), even across decays and steppings.}

Conjecture(Correctness)

\subsection{Statics}

Optionally, we want to give the filtered stepper calculus so that it can have some reasonable behavior with regard to free variables. The most notable benefit we can obtained from a typing system is that we can enforce that there is no free variable in the pattern and the expression before we test if they match.

We can extend the definition of filtered stepper calculus to include type.
\begin{figure}[h]
  \begin{equation*}
    \begin{array}{rcl}
      \ldots       &         & \ldots \\
      \DefTyp \tau &\Coloneqq& \Natural \\
                   &\mid     & \Arrow{\tau}{\tau}
    \end{array}
  \end{equation*}
  \caption{Syntax of Typed Filtered Stepper Calculus. Only definitions of types are listed, refer to Figure~\ref{fig:filter-syntax} for omitted definitions.}
  \label{fig:filter-syntax}
\end{figure}

\fbox{\(\TypeEntails{\Gamma}{e}{\tau}\)} Under typing context \(\Gamma\), expression \(e\) has type \(\tau\).
\begin{mathpar}
  \inferrule[TE-Var]{
    \TypeContains{\Gamma}{x}{\tau}
  }{
    \TypeEntails{\Gamma}{x}{\tau}
  } \qquad
  \inferrule[TE-Lam]{
    \TypeEntails{\TypeExtends{\Gamma}{x}{\tau_x}}{e}{\tau_e}
  }{
    \TypeEntails{\Gamma}{\Lam{x}{e}}{\Arrow{\tau_x}{\tau_e}}
  } \qquad
  \inferrule[TE-Ap]{
    \TypeEntails{\Gamma}{e_1}{\Arrow{\tau_x}{\tau_e}} \\
    \TypeEntails{\Gamma}{e_2}{\tau_x}
  }{
    \TypeEntails{\Gamma}{e_1(e_2)}{\tau_e}
  } \\
  \inferrule[TE-Nat]{
    \
  }{
    \TypeEntails{\Gamma}{\Nat{n}}{\Natural}
  } \qquad
  \inferrule[TE-Add]{
    \TypeEntails{\Gamma}{e_1}{\Natural} \\
    \TypeEntails{\Gamma}{e_2}{\Natural}
  }{
    \TypeEntails{\Gamma}{e_1 + e_2}{\Natural}
  } \\
  \inferrule[TE-Filter]{
    \TypeEntails{\Gamma}{p}{\tau_p} \\
    \TypeEntails{\Gamma}{e}{\tau_e}
  }{
    \TypeEntails{\Gamma}{\Filter{p}{e}}{\tau_e}
  } \qquad
  \inferrule[TE-Residue]{
    \TypeEntails{\Gamma}{e}{\tau}
  }{
    \TypeEntails{\Gamma}{\Residue{a}{g}{l}{e}}{\tau}
  }
\end{mathpar}

\fbox{\(\TypeEntails{\Gamma}{p}{\tau}\)} Under typing context \(\Gamma\), expression \(p\) has type \(\tau\).
\begin{mathpar}
  \inferrule[TP-WE]{
  }{
    \TypeEntails{\Gamma}{\PatExpr}{\tau}
  } \qquad
  \inferrule[TP-WV]{
  }{
    \TypeEntails{\Gamma}{\PatValue}{\tau}
  } \\
  \inferrule[TP-Var]{
    \TypeContains{\Gamma}{x}{\tau}
  }{
    \TypeEntails{\Gamma}{x}{\tau}
  } \qquad
  \inferrule[TP-Lam]{
    \TypeEntails{\TypeExtends{\Gamma}{x}{\tau_x}}{e}{\tau_e}
  }{
    \TypeEntails{\Gamma}{\Lam{x}{e}}{\Arrow{\tau_x}{\tau_e}}
  } \qquad
  \inferrule[TP-Ap]{
    \TypeEntails{\Gamma}{p_1}{\Arrow{\tau_x}{\tau_p}} \\
    \TypeEntails{\Gamma}{p_2}{\tau_x}
  }{
    \TypeEntails{\Gamma}{p_1(p_2)}{\tau_e}
  } \\
  \inferrule[TP-Nat]{
    \
  }{
    \TypeEntails{\Gamma}{\Nat{n}}{\Natural}
  } \qquad
  \inferrule[TP-Add]{
    \TypeEntails{\Gamma}{e_1}{\Natural} \\
    \TypeEntails{\Gamma}{e_2}{\Natural}
  }{
    \TypeEntails{\Gamma}{e_1 + e_2}{\Natural}
  }
\end{mathpar}

Theorem(Preservation): If a expression has type \(\tau\), then aftere stepping, the stepped expression also has type \(\tau\).
\[
  \forall e, \tau,
  \TypeEntails{\varnothing}{e}{\tau} \land \RuleStep{e}{e'} \implies \TypeEntails{\varnothing}{e'}{\tau}
\]

Theorem(Progress): If a expression is well typed, then it can be stepped.
\[
  \forall e, \tau,
  \TypeEntails{\varnothing}{e}{\tau} \implies \RuleStep{e}{e'} \lor \Value{e}
\]

\TODO{Get filter a simple type system like STLC}

%%% Local Variables:
%%% mode: latex
%%% TeX-master: "main"
%%% End:
