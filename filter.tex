\section{Filtered Stepper Calculus}

In this section, we will discuss the semantics of the \emph{Filtered Stepper
Calculus}. We first introduce the syntax of the language, then we discuss the
the dynamics of the language. Finally, we will discuss the optional part of the
semantic, the statics of the language, and show that the calculus satisfies
properties like progression and preservation.

\subsection{Syntax}

\begin{figure}[h]
  \begin{equation*}
    \begin{array}{rcl}
      \DefAct a    &\Coloneqq& \ActSkip \mid \ActStep \\
      \DefGas g    &\Coloneqq& \GasOne \mid \GasAll \\
      \DefFilter f &\Coloneqq& (p, a, g) \\
      \DefExp e    &\Coloneqq& x \mid e(e) \mid \Lam{x}{e} \mid e + e \mid \Nat{n} \\
                   &\mid     & \Filter{f}{e} \\
                   &\mid     & \Residue{a}{g}{l}{e} \\
      \DefPat p    &\Coloneqq& x \mid p(p) \mid \Lam{x}{p} \mid p + p \mid \Nat{n} \\
                   &\mid     & \PatExpr \\
                   &\mid     & \PatValue \\
    \end{array}
  \end{equation*}
  \caption{Syntax of Filtered Stepper Calculus. Here, $n$ ranges over numbers, $x$ over variables, and $l$ over priorities, which are natural numbers.}
  \label{fig:filter-syntax}
\end{figure}

The syntax of filter stepper calculus (Fig.~\ref{fig:filter-syntax}) is a
extended version of untyped lambda calculus. Additionally, natural numbers and
addition are added to it for some more concrete demonstration. It introduce two
new expression to a untyped lambda calculus language, \texttt{Filter} and
\texttt{Residue}. The \texttt{Filter} expression is visible at user level,
indicating the expression surrounded by the expression is filtered by the
filter. The \texttt{Residue} expression is used internally by the stepper and is
not visible to the user. It is used to signal the \emph{residue} of a effect of
applying a filter to an expression.

A filter is a tuple of a pattern, an action, and a gas. The pattern
is a mutated of the term of untyped lambda calculus. Apart from constructs like
variables, applications, and lambdas, it also includes some new constructs, a
forall pattern \(\PatExpr\), a value pattern \(\PatValue\). The action is a tag indicating
the action to be taken when the pattern matches an expression. It has two
form, \(\ActSkip\) and \(\ActStep\). The gas is another tag indicating the
live time of the effect of a filter. It has two form, \(\GasOne\) and
\(\GasAll\).

The residue expression \(\Residue{a}{g}{l}{e}\) is a expression that is used to
store the residue of the effect of a filter. It has four components, an action
\(a\), a gas \(g\), a priority \(l\), and a expression \(e\). The action and the
gas are the same as the ones in a filter. The priority is a natural number
indicating the priority of the residue.

\subsection{Contextual Dynamics}

The contextual dynamics of the filtered stepper calculus builds around
the stepping judgment. This judgment establish the contextual behavior
of the stepper.

There are two main operations in our semantics, one is to \emph{skip} the
evaluation of an expression, the other is to \emph{pause} at a certain point of
evaluation. These two operations leads to two main stepping judgment in our
dynamic specifications. The \textsc{S-Step} rule specifies when do we want to
stop at a step, while the \textsc{S-Skip} rule specifies when we want to skip over
current step and proceed to next ones.

There is an extra rule in the stepping judgment. The elimination step of filter
should be always skipped. As pattern-matching against the filter statement is
not possible within current pattern definition as in
Figure.~\ref{fig:filter-syntax}, we treat them specially to make sure these
steps are hidden and is not visible to users.

We will first see two examples going through how should we or user
interpreter these two rules, the gives the formal definitions of the
judgment and all preceding judgment, including judgment for
instrumentation, decomposition/re-composition, instruction-transition,
and most importantly, stepping.

% \TODO{motivate the two rules}

\subsubsection{Examples and Motivations}

Consider the following Hazel program:
\lstset{xleftmargin=1in}
\begin{lstlisting}[language=hazel,caption={A simple Hazel program}]
debug eval(1 + 2 + 3 + 4) in
debug stop(3 + 3 + 4) in
1 + 2 + 3 + 4\end{lstlisting}

This Hazel program will go through several steps as listed below, to
be evaluated against the order specified by the user.

\begin{enumerate}
\item \emph{Instrumentation}. \label{num:simple_example_instrument} At
  this step, we transverse the expression recursively and collect all
  the filters along the path. Then for each sub-expression, we test if
  the sub-expression matches any of the collected filters. If it does,
  we insert a residue expression with the filter information around
  the matched sub-expression.

  If we start with an empty set of filters, we might end up in a
  situation where there is no sub-expression in the program that
  matches any of the filters. In that situation, the stepper won't be
  able to decide to skip over or stop at a step during evaluation.
  Therefore, we set the default behavior to be stepping through the
  evaluation of every expression.

  In our example above, we apply 3 filters to the expression
  \lstinline[language=hazel]{1 + 2 + 3 + 4}. One of them is the
  default filter that gets prepend to every user program, and the
  other two is written by the user.

  \begin{enumerate}
  \item \lstinline[language=hazel]{debug stop($e)}
  \item \lstinline[language=hazel]{debug eval(1 + 2 + 3 + 4)}
  \item \lstinline[language=hazel]{debug stop(3 + 3 + 4)}
  \end{enumerate}

  During the instrumenting of the first filter, we find out that the
  sub-expression \lstinline[language=hazel]{1 + 2 + 3 + 4} matches the
  pattern \lstinline[language=hazel]{1 + 2 + 3 + 4}, so we instrument
  the matched expression by wrapping it in a internal language
  construct called residue expression
  \(\Residue{\ActSkip}{\GasAll}{0}{\ldots}\) around the sub-expression
  (I-Add-Y). The instrumented expression is in Listing~\ref{lst:simple_example_instrument}.

  \begin{lstlisting}[language=hazel,caption={Program after instrumentation},label={lst:simple_example_instrument}]
debug stop($e) in
debug eval(1 + 2 + 3 + 4) in
debug stop(3 + 3 + 4) in
@\(
\prescript{\ActStep}{0}{\langle}
\prescript{\ActSkip}{1}{\langle}
\prescript{\ActStep}{0}{\langle}
\prescript{\ActStep}{0}{\langle}
\)@1 + 2@\({\rangle}^{\GasOne}\)@ + 3@\({\rangle}^{\GasOne}\)@ + 4@\(\rangle^{\GasAll}
\rangle^{\GasOne}\)@\end{lstlisting}

  However, during the instrumenting of the second filter, there is no
  sub-expression that matches the pattern
  \lstinline[language=hazel]{3 + 3 + 4}, so no residue expression is
  inserted around the sub-expression (I-Add-N). The instrumented
  expression remains the same.

\item \emph{Optimization}. The stepper might have insert too many
  residue expression in the instrumentation step, and this could
  probably cause a significant slowdown of the stepper. Therefore,
  some redundant residue expressions are optimized out during this
  step.

  Generally, we want to remove all immediately adjacent residue
  expressions by picking the one with higest priority, because this is
  the only one that can be picked among these residue expressions
  during the action selection process. Here, the most outward two
  residue expressions are adjacent:
  \(\Residue{\ActStep}{\GasOne}{0}{\Residue{\ActSkip}{\GasAll}{1}{\ldots}}\),
  we remove the ones with lower priorities (O-Residue-Inner) to get
  \(\Residue{\ActSkip}{\GasAll}{1}{\ldots}\).

\begin{lstlisting}[
  language=hazel,
  caption={Program after instrumentation, optimized},
]
debug stop($e) in
debug eval(1 + 2 + 3 + 4) in
debug stop(3 + 3 + 4) in
@\(
\prescript{\ActSkip}{1}{\langle}
\prescript{\ActStep}{0}{\langle}
\prescript{\ActStep}{0}{\langle}
\)@1 + 2@\({\rangle}^{\GasOne}\)@ + 3@\({\rangle}^{\GasOne}\)@ + 4@\(
\rangle^{\GasAll}
\)@\end{lstlisting}

\item \emph{Decomposition}. \label{num:simple_example_decompose} The
  expression is then decomposed into an evaluation context and a
  selected expression. The instrumented expression from last step can
  be decomposed by recursively decomposing addition (D-Add-L) and
  residue expression (D-Residue), into an evaluation context
  (Listing~\ref{lst:simple_example_decompose}) and a selected
  expression \lstinline{1 + 2}

  \begin{lstlisting}[language=hazel,caption={Decomposed evaluation context},label={lst:simple_example_decompose}]
debug stop($e) in
debug eval(1 + 2 + 3 + 4) in
debug stop(3 + 3 + 4) in
@\(
\prescript{\ActSkip}{1}{\langle}
\prescript{\ActStep}{0}{\langle}
\prescript{\ActStep}{0}{\langle}
\circ{\rangle}^{\GasOne}\)@ + 3@\({\rangle}^{\GasOne}\)@ + 4@\(
\rangle^{\GasAll}
\)@\end{lstlisting}

\item \emph{Action Selection}. \label{num:simple_example_select}
  Select an action from the evaluation context. The action is chosen
  first by their priority, then by their distance to the mark
  (\(\circ\)) in the evaluation context. In this case, we transverse
  the expression using (A-Add-L) and update the selected action
  accordingly. The action \(\ActSkip\) is selected, and the expression
  itself is kept the same.

\item \emph{Residue removal (Decaying)}. Residue expressions that
  has life-time of one is removed from the expression. This step is
  done using the \(\Decay\) function. The expression after the removal
  of the residue expression is:
  \begin{lstlisting}[language=hazel,caption={Evaluation context after decaying},label={lst:simple_example_decay}]
debug stop($e) in
debug eval(1 + 2 + 3 + 4) in
debug stop(3 + 3 + 4) in
@\(\prescript{\ActSkip}{1}{\langle}\circ\)@ + 3 + 4@\(\rangle^{\GasAll}\)@\end{lstlisting}

\item \emph{User selection}. If the selected expression produced in
  the Step~\ref{num:simple_example_decompose} is of the form
  \lstinline[language=hazel]{debug .. in ..} or
  \(\Residue{a}{g}{l}{e}\), or the selected action from
  Step~\ref{num:simple_example_select} is \(\ActSkip\), the stepper
  will skip over this step and goes into the next step. (S-Filter and
  S-Skip). Otherwise, the stepper will yield the control to the user,
  and let them to choose which next possible step to take. (S-Step)

  In this case, the selected action is \(\ActSkip\). Therefore, the stepper
  will skipping over this step and goes into the next step.

\item \emph{Instruction Transition}. The selected expression is
  instruction-transitioned to a new expression. In this case, selected
  expression \lstinline[language=hazel]{1 + 2} is
  instruction-transitioned to \lstinline[language=hazel]{3} according
  to the transition rule (T-Add).

\item \emph{Composition}: When the selected-expression is
  instruction-transitioned to a new expression, we need to compose the
  transitioned selected expression with evaluation context to get the
  updated expression. The rules for composition is the same as the
  rules for decomposition, but instead, we start from rule (D-Add-L)
  to build up the recomposed expression in a bottom-up way. The
  recomposed expression is:
  \begin{lstlisting}[language=hazel]
debug stop($e) in
debug eval(1 + 2 + 3 + 4) in
debug stop(3 + 3 + 4) in
@\(\prescript{\ActSkip}{1}{\langle}\)@3 + 3 + 4@\(\rangle^{\GasAll}\)@\end{lstlisting}

\item \emph{Recursion}. As long as we don't get a value, we will keep
  stepping through the expression, i.e. goes back to
  Step~\ref{num:simple_example_instrument}.
\end{enumerate}

In the following round of the procedure, the program is first
instrumented (Listing~\ref{lst:simple_example_1_instrument}), then get
decomposed into an evaluation context
(Listing~\ref{lst:simple_example_1_decompose}) with a selected
expression \lstinline[language=hazel]{3 + 3}.

\begin{lstlisting}[
  language=hazel,
  caption={Instrumented program in Round 2},
  label={lst:simple_example_1_instrument}
]
debug stop($e) in
debug eval(1 + 2 + 3 + 4) in
debug stop(3 + 3 + 4) in
@\(
\prescript{\ActSkip}{1}{\langle}
\prescript{\ActStep}{0}{\langle}
\prescript{\ActStep}{2}{\langle}
\prescript{\ActStep}{0}{\langle}
\)@3 + 3@\(
\rangle^{\GasOne}
\)@ + 4@\(
\rangle^{\GasOne}
\rangle^{\GasOne}
\rangle^{\GasAll}
\)@\end{lstlisting}

\begin{lstlisting}[
  language=hazel,
  caption={Decomposed evaluation context in Round 2},
  label={lst:simple_example_1_decompose}
]
debug stop($e) in
debug eval(1 + 2 + 3 + 4) in
debug stop(3 + 3 + 4) in
@\(
\prescript{\ActSkip}{1}{\langle}
\prescript{\ActStep}{0}{\langle}
\prescript{\ActStep}{2}{\langle}
\prescript{\ActStep}{0}{\langle}
\circ
\rangle^{\GasOne}
\)@ + 4@\(
\rangle^{\GasOne}
\rangle^{\GasOne}
\rangle^{\GasAll}
\)@\end{lstlisting}

Then, we computed the selected action to be \(\ActStep\). Therefore,
the stepper stops at this step and let user to choose which
sub-expression should be evaluated for the next step, by clicking the
yellow area as shown in the Figure~\ref{fig:simple_example_final_ui}

\begin{figure}[h]
  \centering
  \includegraphics[width=0.4\textwidth]{images/filter-example-user-select.png}
  \caption{User Interface for the \emph{User Selection} Step.}\label{fig:simple_example_final_ui}
\end{figure}

Let's take a look at another motivating example. Suppose a first-year
student in college is learn the concept of recursion by writing a
simple recursive function to calculate the n-th term of the Fibonacci
sequence. They would probably write a Hazel program like the one in Listing~\ref{lst:filter_example_fib_program}

\begin{lstlisting}[
  language=hazel,
  caption={A Hazel program that calculates the n-th term of the Fibonacci sequence},
  label={lst:filter_example_fib_program}
]
debug eval($e) in
let fib = fun n ->
  if n == 1 then
    1
  else
    fib(n - 1) + fib(n - 2)
in
debug stop(fib($v)) in
fib(3)
\end{lstlisting}

% The stepper will go through a similar process to determine how to
% evaluate the program and when to stop and yield the control to the
% user through the user interface. To reduce the verbosity, we only
% state 

% \begin{enumerate}
% \item \emph{Instrumentation, Decomposition, and Residue Removal}. The program can be decomposed into an
%   evaluation context (Listing~\ref{lst:filter_example_fib_decompose}) and a selected expression \lstinline[language=hazel]{fib(3)}.
% \begin{lstlisting}[
%   language=hazel,
%   caption={Decomposed evaluation context of the program},
%   label={lst:filter_example_fib_decompose}
% ]
% debug eval($e) in
% @\(\prescript{\ActSkip}{0}{\langle}\)@let fib = fun n ->
%      if n == 1 then
%        1
%      else
%        fib(n - 1) + fib(n - 2)
%    in
%    debug stop(fib($v)) in
%    @\(
% \prescript{\ActSkip}{0}{\langle}
% \prescript{\ActStep}{1}{\langle}
% \circ
% \rangle^{\GasOne}
% \rangle^{\GasAll}
% \rangle^{\GasAll}
% \)@\end{lstlisting}
% \item \emph{Action selection and User selection}. Since the stepper
%   selected the action \(\ActStep\), it will stop here and yield its
%   control to user.
% \item \emph{Instruction Transition, Composition, and Decomposition in
%     the next round}. The decomposition in Round 2 yields an evaluation
%   context as in Listing~\ref{lst:filter_example_fib_decompose} and a selected expression
%   \lstinline[language=hazel]{3 == 1}
% \begin{lstlisting}[
%   language=hazel,
%   caption={Decomposed evaluation context of the program},
%   label={lst:filter_example_fib_decompose}
% ]
% debug eval($e) in
% @\(\prescript{\ActSkip}{0}{\langle}\)@let fib = fun n ->
%      if n == 1 then
%        1
%      else
%        fib(n - 1) + fib(n - 2)
%    in
%    debug stop(fib($v)) in
%    @\(
% \prescript{\ActSkip}{0}{\langle}
% \prescript{\ActStep}{1}{\langle}
% \)@if @\(\circ\)@ then
%             1
%           else
%             fib(3 - 1) + fib(3 - 2)@\(
% \rangle^{\GasOne}
% \rangle^{\GasAll}
% \rangle^{\GasAll}
% \)@\end{lstlisting}
% \item \emph{Action Selections and User Interaction}. The selected
%   action is \(\ActSkip\). This step is skipped. 
% \item \emph{}
% \end{enumerate}

With the two motivating examples, the rules for the dynamic semantics
of the filter calculus is presented as following.

\fbox{\(\Step{e}{e'}{n}\)} Expression \(e\) steps to \(e'\) in \(n\) steps.
\begin{mathpar}
  \inferrule[S-Step]{
    \InstructPAGL{\PatExpr}{\ActStep}{\GasOne}{0}{e}{e_i} \\
    \Decompose{e_i}{\mathcal{E}_0}{e_0} \\
    \Analyze{(\ActStep, 0)}{\mathcal{E}_0}{\ActStep} \land{} \neg{} (\Strippable{e_0}) \\
    \Transition{e_0}{e_t} \\
    \Compose{e_1}{(\Decay{\mathcal{E}_1})}{e_t}
  }{
    \Step{e}{e_1}{1}
  } \\
  \inferrule[S-Skip]{
    \InstructPAGL{\PatExpr}{\ActStep}{\GasOne}{0}{e}{e_i} \\
    \Decompose{e_i}{\mathcal{E}_0}{e_0} \\
    \Analyze{(\ActStep, 0)}{\mathcal{E}_0}{\ActSkip} \lor \Strippable{e_0}\\
    \Transition{e_0}{e_t} \\
    \Compose{e_1}{(\Decay{\mathcal{E}_1})}{e_t} \\    
    \Step{e_1}{e_2}{n}
  }{
    \Step{e}{e_2}{n + 1}
  } \\
  \inferrule[S-Value]{
    \Value{v}
  }{
    \Step{v}{v}{0}
  }
\end{mathpar}

\fbox{\(\JustStep{e}{e'}{n}\)} Expression \(e\) takes \(n\) steps to \(e'\), completely ignore the behavior of the filters and residues.
\begin{mathpar}
  \inferrule[J-Step]{
    \Decompose{e}{\mathcal{E}_0}{e_0} \\
    \Transition{e_0}{e_0'} \\
    \Compose{e'}{\mathcal{E}_0}{e_0'}
  }{
    \JustStep{e}{e'}{1}
  }
\end{mathpar}

\fbox{\(\Strippable{e}\)} Expression \(e\) can be stripped (non-recursive).
\begin{mathpar}
  \inferrule[F-Filter]{
  }{
    \Strippable{(\Filter{f}{e})}
  } \qquad
  \inferrule[F-Residue]{
  }{
    \Strippable{\Residue{a}{g}{l}{e}}
  }
\end{mathpar}

\fbox{\(\InstructPAGL{p}{a}{g}{l}{e}{e'}\)} \(e\) is dynamically instructed to \(e'\).
\begin{mathpar}
  \inferrule[I-V]{
    \Value{v}
  }{
    \InstructPAGL{p}{a}{g}{l}{v}{v}
  } \qquad
  \inferrule[I-Var]{
    \Matches{p}{x}
  }{
    \InstructPAGL{p}{a}{g}{l}{x}{x}
  } \\
  \inferrule[I-Filter]{
    \InstructF{(p_0, a_0, g_0, l_0)}{e_0}{e} \\
    \InstructF{(p, a, g, l_0 + 1)}{e}{e'}
  }{
    \InstructF{(p_0, a_0, g_0, l_0)}{\Filter{(p, a, g)}{e_0}}{\Filter{(p, a, g)}{e'}}
  } \\
  \inferrule[I-Residue]{
    \InstructF{(p_0, a_0, g_0, l_0)}{e_0}{e} \\
  }{
    \InstructF{(p_0, a_0, g_0, l_0)}{\Residue{a}{g}{l}{e_0}}{\Residue{a}{g}{l}{e}}
  } \\
  \inferrule[I-Ap-Y]{
    \InstructPAGL{p}{a}{g}{l}{e_1}{e_1'} \\
    \InstructPAGL{p}{a}{g}{l}{e_2}{e_2'} \\
    \Matches{p}{e_1(e_2)}
  }{
    \InstructPAGL{p}{a}{g}{l}{e_1(e_2)}{\Residue{a}{g}{l}{e_1'(e_2')}}
  } \\
  \inferrule[I-Ap-N]{
    \InstructPAGL{p}{a}{g}{l}{e_1}{e_1'} \\
    \InstructPAGL{p}{a}{g}{l}{e_2}{e_2'} \\
    \DoesNotMatch{p}{e_1(e_2)}
  }{
    \InstructPAGL{p}{a}{g}{l}{e_1(e_2)}{e_1'(e_2')}
  } \\
  \inferrule[I-Add-Y]{
    \InstructPAGL{p}{a}{g}{l}{e_1}{e_1'} \\
    \InstructPAGL{p}{a}{g}{l}{e_2}{e_2'} \\
    \Matches{p}{e_1 + e_2}
  }{
    \InstructPAGL{p}{a}{g}{l}{e_1(e_2)}{\Residue{a}{g}{l}{e_1' + e_2'}}
  } \\
  \inferrule[I-Add-N]{
    \InstructPAGL{p}{a}{g}{l}{e_1}{e_1'} \\
    \InstructPAGL{p}{a}{g}{l}{e_2}{e_2'} \\
    \DoesNotMatch{p}{e_1 + e_2}
  }{
    \InstructPAGL{p}{a}{g}{l}{e_1(e_2)}{e_1' + e_2'}
  } \\
\end{mathpar}

% \mck{I think there's some preservation of priority property to prove here, since the numbers are re-generated on every pass.}

% \hxf{I would formalize the preservation property like this: the priority of residue/do statements \emph{produced} by the same filter have the same priorities}

% \mck{This instrumentation will regenerate the do statements on every pass - I guess it doesn't cause problems but it would be nice if it didn't?}

% \hxf{Yeah it would be much nicer to avoid the duplication of do statements. We can to some ad-hoc fusing of do-statements same effect \& priorties, probably utilmately we want the instrumentation to be Idempotent.}

\fbox{\(\Decay{\mathcal{E}} = \mathcal{E}'\)} Evaluation Context \(\mathcal{E}\) decays to context \(\mathcal{E}'\).
\[
  \begin{aligned}
    \Decay{x} &= x \\
    \Decay{\Lam{x}{e}} &= \Lam{x}{\Decay{e}} \\
    \Decay{e_1(e_2)} &= (\Decay{e_1})(\Decay{e_2}) \\
    \Decay{\Nat{n}} &= \Decay{\Nat{n}} \\
    \Decay{e_1 + e_2} &= (\Decay{e_1}) + (\Decay{e_2}) \\
    \Decay{\Filter{(p, a, g)}{e}} &= \Filter{(p, a, g)}{\Decay{e}} \\
    \Decay{\Residue{a}{\GasOne}{l}{e}} &= \Decay{e} \\
    \Decay{\Residue{a}{\GasAll}{l}{e}} &= \Residue{a}{\GasAll}{l}{\Decay{e}}
  \end{aligned}
\]

% \TODO{Correct the function to use correct evaluation context}

\fbox{\(\Analyze{(a, l)}{\mathcal{E}}{a'}\)} Under the filter environment of \((a, l)\), the context \(\mathcal{E}\) is transitioned to \(\mathcal{E}'\) and the action \(a'\) is returned.
\begin{mathpar}
  \inferrule[A-Var]{
  }{
    \Analyze{(a, l)}{\circ}{a}
  } \\
  \inferrule[A-Ap-L]{
    \Analyze{(a, l)}{\mathcal{E}_1}{a'}
  }{
    \Analyze{(a, l)}{\mathcal{E}_1(e)}{a'}
  } \qquad
  \inferrule[A-Ap-R]{
    \Analyze{(a, l)}{\mathcal{E}_2}{a'}
  }{
    \Analyze{(a, l)}{e_1(\mathcal{E}_2)}{a'}
  } \\
  \inferrule[A-Add-L]{
    \Analyze{(a, l)}{\mathcal{E}_1}{a'}
  }{
    \Analyze{(a, l)}{\mathcal{E}_1 + e}{a'}
  } \qquad
  \inferrule[A-Add-R]{
    \Analyze{(a, l)}{\mathcal{E}_2}{a'}
  }{
    \Analyze{(a, l)}{e_1 + \mathcal{E}_2}{a'}
  } \\
  \inferrule[A-Filter]{
    \Analyze{(a, l)}{\mathcal{E}}{a'}
  }{
    \Analyze{(a, l)}{\Filter{f}{\mathcal{E}}}{a'}
  } \\
  \inferrule[A-Residue-One-Old]{
    l \le l_0 \\
    \Analyze{(a_0, l_0)}{\mathcal{E}}{a'}
  }{
    \Analyze{(a_0, l_0)}{\Residue{a}{\GasOne}{l}{\mathcal{E}}}{a'}
  } \qquad
  \inferrule[A-Residue-One-New]{
    l > l_0 \\
    \Analyze{(a, l)}{\mathcal{E}}{a'}
  }{
    \Analyze{(a_0, l_0)}{\Residue{a}{\GasOne}{l}{\mathcal{E}}}{a'}
  } \\
  \inferrule[A-Residue-All-Old]{
    l \le l_0 \\
    \Analyze{(a_0, l_0)}{\mathcal{E}}{a'}
  }{
    \Analyze{(a_0, l_0)}{\Residue{a}{\GasAll}{l}{\mathcal{E}}}{a'}
  } \qquad
  \inferrule[A-Residue-All-New]{
    l > l_0 \\
    \Analyze{(a, l)}{\mathcal{E}}{a'}
  }{
    \Analyze{(a_0, l_0)}{\Residue{a}{\GasAll}{l}{\mathcal{E}}}{a'}
  } \\
\end{mathpar}

\fbox{\(\DefCtx \mathcal{E}\)} Context \(\mathcal{E}\).
\[
  \DefCtx \mathcal{E}
  = \Mark
  \mid \mathcal{E}(d)
  \mid d(\mathcal{E})
  \mid \mathcal{E} + d
  \mid d + \mathcal{E}
  \mid \Filter{f}{\mathcal{E}}
  \mid \Residue{a}{g}{l}{\mathcal{E}}
\]

\fbox{\(\Value e\)} Expression \(e\) is a value.
\begin{mathpar}
  \inferrule[V-Lam]{
  }{
    \Value{\Lam{x}{e}}
  } \qquad
  \inferrule[V-Nat]{
  }{
    \Value{\Nat{n}}
  }
\end{mathpar}

\fbox{\([v / x] e = e'\)} \(e'\) can be obtained by substitution of \(x\) for
\(v\) in expression \(e\).
\[
  \begin{aligned}
    [v / x] \Nat{n} &= \Nat{n} \\
    [v / x] x &= v \\
    [v / x] y &= y && \text{if } x \neq y \\
    [v / x] (e_1(e_2)) &= ([v / x] e_1)([v / x] e_2) \\
    [v / x] (e_1 + e_2) &= ([v / x] e_1) + ([v / x] e_2) \\
    [v / x] \Lam{y}{e} &= \Lam{y}{[v / x] e} && \text{if } x \neq y \\
    [v / x] \Lam{y}{e} &= \Lam{y}{e} && \text{if } x = y \\
    [v / x] \Filter{(p, a, g)}{e} &= \Filter{([v / x]p, a, g)}{[v / x] e} \\
    [v / x] \Residue{a}{g}{l}{e} &= \Residue{a}{g}{l}{[v / x] e} \\
    [v / x] \Fix{x}{e} &= \Fix{x}{e} \\
    [v / x] \Fix{y}{e} &= \Fix{y}{[v / x]e} && \text{if } x \neq y
  \end{aligned}
\]

\fbox{\([v / x] p = p'\)} \(p'\) can be obtained by substitution of \(x\) for
\(v\) in pattern \(p\).
\[
  \begin{aligned}
    [v / x] \$e &= \$e \\
    [v / x] \$v &= \$v \\
    [v / x] \Nat{n} &= \Nat{n} \\
    [v / x] y &= y && \text{if } x \neq y \\
    [v / x] (p_1(p_2)) &= ([v / x] p_1)([v / x] p_2) \\
    [v / x] (p_1 + p_2) &= ([v / x] p_1) + ([v / x] p_2) \\
    [v / x] \Lam{y}{e} &= \Lam{y}{[v / x] e} && \text{if } x \neq y \\
    [v / x] \Lam{y}{e} &= \Lam{y}{e} && \text{if } x = y \\
    [v / x] \Fix{x}{e} &= \Fix{x}{e} \\
    [v / x] \Fix{y}{e} &= \Fix{y}{[v / x]e} && \text{if } x \neq y
  \end{aligned}
\]

\fbox{\(\Matches{p}{e}\)} means that pattern \(p\) matches expression \(e\).
\begin{mathpar}
  \inferrule[M-All]{\ }{
    \Matches{\$e}{e}
  } \qquad
  \inferrule[M-Val]{\Value{v}}{
    \Matches{\$v}{v}
  } \qquad
  \inferrule[M-Nat]{
    \
  }{
    \Matches{\Nat{m}}{\Nat{m}}
  } \\
  \inferrule[M-Lam]{
    \Strip{e_1} \equiv \Strip{e_2}
  }{
    \Matches{\Lam{x_1}{e_1}}{\Lam{x_2}{e_2}}
  } \qquad
  \inferrule[M-Fix]{
    \Strip{e_1} \equiv \Strip{e_2}
  }{
    \Matches{\Fix{x_1}{e_1}}{\Fix{x_2}{e_2}}
  } \\
  \inferrule[M-Ap]{
    \Matches{p_1}{e_1} \\
    \Matches{p_2}{e_2}
  }{
    \Matches{p_1(p_2)}{e_1(e_2)}
  } \qquad
  \inferrule[M-Add]{
    \Matches{p_1}{e_1} \\
    \Matches{p_2}{e_2}
  }{
    \Matches{p_1 + p_2}{e_1 + e_2}
  }
\end{mathpar}

\fbox{\(\Decompose{e}{\mathcal{E}}{e'}\)} Expression \(e\) can be obtained by putting expression \(e'\) into the mark of \(\mathcal{E}\).
\begin{mathpar}
  % Residue
  \inferrule[D-Residue-T]{
    \Decompose{e}{\mathcal{E}}{e'}
  }{
    \Decompose{\Residue{a}{g}{l}{e}}{\Residue{a}{g}{l}{{\mathcal{E}}}}{e'}
  } \qquad
  \inferrule[D-Residue-E]{
    \Value{v}
  }{
    \Decompose{\Residue{a}{g}{l}{v}}{\circ}{\Residue{a}{g}{l}{v}}
  } \\
  % Filter
  \inferrule[D-Filter-T]{
    \Decompose{e}{\mathcal{E}}{e'}
  }{
    \Decompose{\Filter{f}{e}}{\Filter{f}{\mathcal{E}}}{e'}
  } \qquad
  \inferrule[D-Filter-E]{
    \Value{v}
  }{
    \Decompose{\Filter{f}{v}}{\circ}{\Filter{f}{v}}
  } \\
  % Application
  \inferrule[D-Ap-L]{
    \Decompose{e_1}{\mathcal{E}_1}{e_1'}
  }{
    \Decompose{e_1(e_2)}{\mathcal{E}_1(e_2)}{e_1'}
  } \qquad
  \inferrule[D-Ap-R]{
    \Value{e_1} \\
    \Decompose{e_2}{\mathcal{E}_2}{e_2'}
  }{
    \Decompose{e_1(e_2)}{e_1(\mathcal{E}_2)}{e_2'}
  } \qquad
  \inferrule[D-Ap-E]{
    \Value{e_1} \\
    \Value{e_2}
  }{
    \Decompose{e_1(e_2)}{\circ}{e_1(e_2)}
  } \\
  % Addition
  \inferrule[D-Add-L]{
    \Decompose{e_1}{\mathcal{E}_1}{e_1'}
  }{
    \Decompose{e_1 + e_2}{\mathcal{E}_1 + e_2}{e_1'}
  } \qquad
  \inferrule[D-Add-R]{
    \Value{e_1} \\
    \Decompose{e_2}{\mathcal{E}_2}{e_2'}
  }{
    \Decompose{e_1 + e_2}{e_1 + \mathcal{E}_2}{e_2'}
  } \qquad
  \inferrule[D-Add-E]{
    \Value{e_1} \\
    \Value{e_2}
  }{
    \Decompose{e_1 + e_2}{\circ}{e_1 + e_2}
  } \\
  \inferrule[D-Fix-E]{
    \
  }{
    \Decompose{\Fix{x}{e}}{\circ}{\Fix{x}{e}}
  }
\end{mathpar}

\fbox{\(\Compose{e}{\mathcal{E}}{e'}\)} Expression \(e\) can be obtained by putting expression \(e'\) into the mark of \(\mathcal{E}\).
\begin{mathpar}
  \inferrule[C-Top]{
  }{
    \Compose{e}{\circ}{e}
  } \\
  \inferrule[C-Ap-L]{
    \Compose{e_1}{\mathcal{E}_1}{e_1'}
  }{
    \Compose{e_1(e_2)}{\mathcal{E}_1(e_2)}{e_2'}
  } \qquad
  \inferrule[C-Ap-R]{
    \Compose{e_2}{\mathcal{E}}{e_2'}
  }{
    \Compose{e_1(e_2)}{e_1(\mathcal{E}_2)}{e_2'}
  } \\
  \inferrule[C-Add-L]{
    \Compose{e_1}{\mathcal{E}_1}{e_1'}
  }{
    \Compose{e_1 + e_2}{\mathcal{E}_1 + e_2}{e_1'}
  } \qquad
  \inferrule[C-Add-R]{
    \Compose{e_2}{\mathcal{E}_2}{e_2'}
  }{
    \Compose{e_1 + e_2}{e_1 + \mathcal{E}_2}{e_2'}
  } \\
  \inferrule[C-Filter]{
    \Compose{e}{\mathcal{E}}{e'}
  }{
    \Compose{\Filter{f}{e}}{\Filter{f}{\mathcal{E}}}{e'}
  } \qquad
  \inferrule[C-Residue]{
    \Compose{e}{\mathcal{E}}{e'}
  }{
    \Compose{\Residue{a}{g}{l}{e}}{\mathcal{E}}{e'}
  }
\end{mathpar}

\fbox{\(\Strip{e} = {e'}\)} Strip filter/do expression in expression \(e\) to get expression \(e'\).
\[
  \begin{aligned}
    \Strip{x} &= x \\
    \Strip{\Lam{x}{e}} &= \Lam{x}{\Strip{e}} \\
    \Strip{(e_1(e_2))} &= (\Strip{e_1})(\Strip{e_2}) \\
    \Strip{\Nat{n}} &= \Nat{n} \\
    \Strip{(e_1 + e_2)} &= (\Strip{e_1}) + (\Strip{e_2}) \\
    \Strip{(\Filter{f}{e})} &= \Strip{e} \\
    \Strip{\Residue{a}{g}{l}{e}} &= \Strip{e} \\
    \Strip{\Fix{x}{e}} &= \Fix{x}{\Strip{e}}
  \end{aligned}
\]

\fbox{\(\Transition{e}{e'}\)} \(e\) takes an instruction transition to \(e'\).
\begin{mathpar}
  \inferrule[T-Ap]{
    \Value{e_1} \\
    \Value{e_2}
  }{
    \Transition{(\Lam{x}{e_1})(e_2)}{\FSubst{e_2}{x}{e_1}}
  } \\
  \inferrule[T-Add]{
    \Value{n_1} \\
    \Value{n_2} \\
    n_1 + n_2 = n
  }{
    \Transition{\Nat{n_1} + \Nat{n_2}}{\Nat{n}}
  } \\
  \inferrule[T-Residue]{
    \Value{v}
  }{
    \Transition{\Residue{a}{g}{l}{v}}{v}
  } \qquad
  \inferrule[T-Filter]{
    \Value{v}
  }{
    \Transition{\Filter{f}{v}}{v}
  } \\
  \inferrule[T-Fix]{
    \
  }{
    \Transition{\Fix{x}{e}}{[\Fix{x}{e} / x] e}
  }
\end{mathpar}

\fbox{\(\IsResidue{e}\)} \(e\) is a residue.
\begin{mathpar}
  \inferrule[Is-Residue]{
    \
  }{
    \IsResidue{\Residue{a}{g}{l}{e}}
  }
\end{mathpar}

\fbox{\(\Optimize{e}{e'}\)} \(e\) is optimized to \(e'\)
\begin{mathpar}
  \inferrule[O-Var]{
    \
  }{
    \Optimize{x}{x}
  } \quad
  \inferrule[O-Val]{
    \Value{e}
  }{
    \Optimize{e}{e}
  } \\
  \inferrule[O-Ap]{
    \Optimize{e_l}{e_l'} \\
    \Optimize{e_r}{e_r'}
  }{
    \Optimize{e_l(e_r)}{e_l'(e_r')}
  } \quad
  \inferrule[O-Add]{
    \Optimize{e_l}{e_l'} \\
    \Optimize{e_r}{e_r'}
  }{
    \Optimize{e_l + e_r}{e_l' + e_r'}
  } \\
  \inferrule[O-Filter]{
    \Optimize{e}{e'}
  }{
    \Optimize{\Filter{f}{e}}{\Filter{f}{e'}}
  } \\
  \inferrule[O-Residue-Inner]{
    l_i > l_o \\
    \Optimize{\Residue{a_i}{g_i}{l_i}{e}}{e'}
  }{
    \Optimize{\Residue{a_o}{g_o}{l_o}{\Residue{a_i}{g_i}{l_i}{e}}}{e'}
  } \quad
  \inferrule[O-Residue-Outer]{
    l_i \leq l_o \\
    \Optimize{\Residue{a_o}{g_o}{l_o}{e}}{e'}
  }{
    \Optimize{\Residue{a_o}{g_o}{l_o}{\Residue{a_i}{g_i}{l_i}{e}}}{e'}
  } \quad
  \inferrule[O-Residue-Other]{
    \neg{(\IsResidue{e})} \\
    \Optimize{e}{e'}
  }{
    \Optimize{\Residue{a}{g}{l}{e}}{e'}
  } \\
  \inferrule[O-Fix]{
    \
  }{
    \Optimize{\Fix{x}{e}}{\Fix{x}{e}}
  }
\end{mathpar}

\subsection{Properties}

% \TODO{Properties}
% \\\\
% Conjecture: Idempotence of Instrumentation (Instrumenting twice should have the same effect as instrumenting once)

% $\InstructPAGL{p}{a}{g}{l}{e}{e'} \Rightarrow \InstructPAGL{p}{a}{g}{l}{e'}{e''} \Rightarrow e' = e''$

% \mck{This conjecture is actually not true, because instrumentation always adds more do statements, even if they're already there.}
% \\\\
% Conjecture : Determinism of steps

% $\RuleStep{e}{e'} \Rightarrow \RuleStep{e}{e''} \Rightarrow e' = e''$

% \mck{I think this holds in this calculus, but not in Hazel itself.}
% \\\\
% Conjecture(Unchanging Priorities)

% \(\)

% \hxf{I believe the Unchanging Priorities does holds. I would phrase it as: the residue statements inserted by the same filter statement has the same priorities (or their priorites keeps in order), even across decays and steppings.}

% Conjecture(Correctness)

\begin{conjecture}[Unchanging Priorities]
  The priorities of filter should keep the same across instrumentation, decaying and stepping.
\end{conjecture}

\mck{Since priorities are always re-numbered, we want some sort of conjecture that they remain the same. We could even possibly use this to avoid duplication of `do's?}

% \begin{conjecture}[Necessacity of Priorities]
%   It always pick the filter that is closer to the target expression.
% \end{conjecture}

% \begin{conjecture}[Scoped Filter]
%   Behavior of matched expression is specified by the user at least once, or can be traced by to a filter statement or the default.
% \end{conjecture}

\begin{theorem}[Simulation]
  If an expression \(e\) take \(n\) steps to become \(e'\), then the stripped down version of \(e\), which is \(\Strip{e}\), will takes \(m\) steps to become \(\Strip{e'}\) for some \(m\).
  \[
    \forall e, n, e', \qquad
    \Step{e}{e'}{n} \implies \exists m, \JustStep{\Strip{e}}{\Strip{e'}}{m}
  \]
\end{theorem}

\subsection{Statics}

Optionally, we want to give the filtered stepper calculus so that it can have some reasonable behavior with regard to free variables. The most notable benefit we can obtained from a typing system is that we can enforce that there is no free variable in the pattern and the expression before we test if they match.

We can extend the definition of filtered stepper calculus to include type.
\begin{figure}[h]
  \begin{equation*}
    \begin{array}{rcl}
      \ldots       &         & \ldots \\
      \DefTyp \tau &\Coloneqq& \Natural \\
                   &\mid     & \Arrow{\tau}{\tau}
    \end{array}
  \end{equation*}
  \caption{Syntax of Typed Filtered Stepper Calculus. Only definitions of types are listed, refer to Figure~\ref{fig:filter-syntax} for omitted definitions.}
  \label{fig:typed-filter-syntax}
\end{figure}

\fbox{\(\TypeEntails{\Gamma}{e}{\tau}\)} Under typing context \(\Gamma\), expression \(e\) has type \(\tau\).
\begin{mathpar}
  \inferrule[TE-Var]{
    \TypeContains{\Gamma}{x}{\tau}
  }{
    \TypeEntails{\Gamma}{x}{\tau}
  } \qquad
  \inferrule[TE-Lam]{
    \TypeEntails{\TypeExtends{\Gamma}{x}{\tau_x}}{e}{\tau_e}
  }{
    \TypeEntails{\Gamma}{\Lam{x}{e}}{\Arrow{\tau_x}{\tau_e}}
  } \qquad
  \inferrule[TE-Ap]{
    \TypeEntails{\Gamma}{e_1}{\Arrow{\tau_x}{\tau_e}} \\
    \TypeEntails{\Gamma}{e_2}{\tau_x}
  }{
    \TypeEntails{\Gamma}{e_1(e_2)}{\tau_e}
  } \\
  \inferrule[TE-Nat]{
    \
  }{
    \TypeEntails{\Gamma}{\Nat{n}}{\Natural}
  } \qquad
  \inferrule[TE-Add]{
    \TypeEntails{\Gamma}{e_1}{\Natural} \\
    \TypeEntails{\Gamma}{e_2}{\Natural}
  }{
    \TypeEntails{\Gamma}{e_1 + e_2}{\Natural}
  } \\
  \inferrule[TE-Filter]{
    \TypeEntails{\Gamma}{p}{\tau_p} \\
    \TypeEntails{\Gamma}{e}{\tau_e}
  }{
    \TypeEntails{\Gamma}{\Filter{p}{e}}{\tau_e}
  } \qquad
  \inferrule[TE-Residue]{
    \TypeEntails{\Gamma}{e}{\tau}
  }{
    \TypeEntails{\Gamma}{\Residue{a}{g}{l}{e}}{\tau}
  } \\
  \inferrule[TE-Fix]{
    \TypeEntails{\TypeExtends{\Gamma}{x}{\tau}}{e}{\tau}
  }{
    \TypeEntails{\Gamma}{\Fix{x}{e}}{\tau}
  }
\end{mathpar}

\fbox{\(\TypeEntails{\Gamma}{p}{\tau}\)} Under typing context \(\Gamma\), expression \(p\) has type \(\tau\).
\begin{mathpar}
  \inferrule[TP-Exp]{
  }{
    \TypeEntails{\Gamma}{\PatExpr}{\tau}
  } \qquad
  \inferrule[TP-Val]{
  }{
    \TypeEntails{\Gamma}{\PatValue}{\tau}
  } \\
  \inferrule[TP-Var]{
    \TypeContains{\Gamma}{x}{\tau}
  }{
    \TypeEntails{\Gamma}{x}{\tau}
  } \qquad
  \inferrule[TP-Lam]{
    \TypeEntails{\TypeExtends{\Gamma}{x}{\tau_x}}{e}{\tau_e}
  }{
    \TypeEntails{\Gamma}{\Lam{x}{e}}{\Arrow{\tau_x}{\tau_e}}
  } \qquad
  \inferrule[TP-Ap]{
    \TypeEntails{\Gamma}{p_1}{\Arrow{\tau_x}{\tau_p}} \\
    \TypeEntails{\Gamma}{p_2}{\tau_x}
  }{
    \TypeEntails{\Gamma}{p_1(p_2)}{\tau_e}
  } \\
  \inferrule[TP-Nat]{
    \
  }{
    \TypeEntails{\Gamma}{\Nat{n}}{\Natural}
  } \qquad
  \inferrule[TP-Add]{
    \TypeEntails{\Gamma}{e_1}{\Natural} \\
    \TypeEntails{\Gamma}{e_2}{\Natural}
  }{
    \TypeEntails{\Gamma}{e_1 + e_2}{\Natural}
  } \qquad
  \inferrule[TP-Fix]{
    \TypeEntails{\TypeExtends{\Gamma}{x}{\tau}}{e}{\tau}
  }{
    \TypeEntails{\Gamma}{\Fix{x}{e}}{\tau}
  }
\end{mathpar}

\begin{theorem}[Preservation]
  If a expression has type \(\tau\), then after stepping, the stepped expression also has a type \(\tau\).
  \[
    \forall e, \tau, n, \Gamma, \qquad
    \TypeEntails{\Gamma}{e}{\tau} \land \Step{e}{e'}{n} \implies \TypeEntails{\Gamma}{e'}{\tau}
  \]
\end{theorem}

\begin{theorem}[Progress]
  If an expression is well typed, then there exists an \(n\) such that the expression can be stepped for \(n\) steps.
  \[
    \forall e, \tau, \qquad
    \TypeEntails{\varnothing}{e}{\tau} \implies \exists e' . \Step{e}{e'}{1}
  \]
\end{theorem}

%%% Local Variables:
%%% mode: LaTeX
%%% TeX-master: "main"
%%% End:
